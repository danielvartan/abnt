% Author: Daniel Vartanian.
% Licence: GNU General Public License v3 or later
%
% Based on template.tex, developed by the Quarto team and
% abtex2-modelo-trabalho-academico.tex, v-1.9.7, developed by
% Lauro César Araujo and the team behind abnt2tex, with additional guidance
% from the theses and dissertations regulations of the University of São Paulo
% (USP). For more information, please visit <http://www.abntex.net.br/>.

% For help, see:
%
% - <https://quarto.org/docs/reference/formats/pdf.html>
% - <https://github.com/abntex/abntex2/wiki/ComoCustomizar>
% - <https://www.ctan.org/pkg/abntex2>
% - <https://www.ctan.org/pkg/memoir>
% - <https://www.ctan.org/pkg/hyperref>

% -----
% Preamble
% -----

% Don´t move the `\PassOptionsToPackage` macros.
% See why: <https://tex.stackexchange.com/a/433605/234832>.
\PassOptionsToPackage{
unicode,bookmarksnumbered
}{hyperref}

\PassOptionsToPackage{hyphens}{url}

\PassOptionsToPackage{dvipsnames,svgnames,x11names}{xcolor}


\documentclass[
12pt,
openright,
oneside,
a4paper,
chapter=TITLE,
section=TITLE,
french,
spanish,
brazil,
english
]{abntex2}
% \usepackage{showframe}

% The order in which the packages are loaded is important!

\usepackage{array}
\usepackage{calc}
\usepackage{caption}
\usepackage{color}
\usepackage{colortbl}
\usepackage{amsmath}
\usepackage{amssymb}
\usepackage{booktabs}
\usepackage{enumitem}
\usepackage{etoolbox}
\usepackage{epigraph}
\usepackage{float}
\usepackage[T1]{fontenc}
\usepackage[hang,multiple]{footmisc}
\usepackage{graphicx}
\usepackage{iftex}
\usepackage{indentfirst}
\usepackage[hyphenation,lastparline,nosingleletter]{impnattypo}
\usepackage[utf8]{inputenc}
\usepackage{lastpage}
\usepackage{lipsum}
\usepackage{longtable}
\usepackage{luacode}
\usepackage{luatexbase}
\usepackage{microtype}
\usepackage{multirow}
\usepackage{parskip}
\usepackage{pdfpages}
\usepackage{titlesec}
\usepackage[table]{xcolor}
\usepackage{xparse}
\usepackage{xstring}
\usepackage{hyperref}
\usepackage{url}

\ifPDFTeX
  \usepackage{textcomp} % provide euro and other symbols
\else % if luatex or xetex
\usepackage{unicode-math}
\fi
% Set Lengths -----

\newlength{\microskipamount}
\newlength{\tinyskipamount}
\newlength{\hugeskipamount}

\setlength{\microskipamount}{0.25\baselineskip} % Arial/12pt/1.5 == 5.4375pt
\setlength{\tinyskipamount}{0.5\baselineskip} % Arial/12pt/1.5 == 10.875pt
\setlength{\smallskipamount}{0.75\baselineskip} % Arial/12pt/1.5 == 16.3125pt
\setlength{\medskipamount}{1\baselineskip} % Arial/12pt/1.5 == 21.75pt
\setlength{\bigskipamount}{1.5\baselineskip} % Arial/12pt/1.5 == 32.625pt
\setlength{\hugeskipamount}{2\baselineskip}% Arial/12pt/1.5 == 43.5pt

\newcommand{\microskip}{\vspace{\microskipamount}}
\newcommand{\tinyskip}{\vspace{\tinyskipamount}}
\newcommand{\hugeskip}{\vspace{\hugeskipamount}}

% Set Skips -----

\setlength{\beforechapskip}{\bigskipamount}
\setlength{\afterchapskip}{\smallskipamount}

\titlespacing*{\chapter}{0pt}{\beforechapskip}{\afterchapskip}
\titlespacing*{\section}{0pt}{\medskipamount}{\smallskipamount}
\titlespacing*{\subsection}{0pt}{\medskipamount}{\smallskipamount}
\titlespacing*{\subsubsection}{0pt}{\medskipamount}{\smallskipamount}
\titlespacing*{\paragraph}{0pt}{\medskipamount}{\smallskipamount}

% Set Epigraph -----

\setlength\epigraphwidth{0.6\textwidth}
\setlength\epigraphrule{0pt}
% Set Page -----

\setlength{\headsep}{1cm}
\setlength{\footskip}{1cm}
\checkandfixthelayout[fixed]

% Set Text Spacing -----

\renewcommand{\familydefault}{\rmdefault}

\renewcommand{\baselinestretch}{1.5}

\setlength{\parindent}{1cm}

\setlength{\parskip}{0ex}

% Set Text Font -----


\ifPDFTeX\else
    % xetex/luatex font selection
  \setmainfont[]{Noto-Sans}
  \setsansfont[]{Noto-Sans}
  \setmonofont[Scale=0.75]{Noto-Sans-Mono}




\fi


% Set Footnote -----

\setlength{\footnotemargin}{0.5em} % Equal to `\footmarkwidth`
\let\svfootnoterule\footnoterule % Equal to `\footmarksep`
\renewcommand\footnoterule{\vspace{1ex}\svfootnoterule\vspace{1ex}}
\newcommand{\capaname}{Capa}
\newcommand{\fichacatalograficaname}{Ficha catalográfica}
\newcommand{\resumoestrangeironame}{Resumo}
\newcommand{\glossarioname}{Glossário}

\addto\captionsenglish{
  \renewcommand{\capaname}{Cover}
  \renewcommand{\folhaderostoname}{Title Page}
  \renewcommand{\fichacatalograficaname}{Cataloging Record}
  \renewcommand{\errataname}{Errata}
  \renewcommand{\folhadeaprovacaoname}{Approval Sheet}
  \renewcommand{\dedicatorianame}{Inscription}
  \renewcommand{\agradecimentosname}{Acknowledgements}
  \renewcommand{\epigraphname}{Epigraph}
  \renewcommand{\resumoname}{Abstract}
  \renewcommand{\resumoestrangeironame}{Resumo}
  \renewcommand{\listfigurename}{List of Figures}
  \renewcommand{\listtablename}{List of Tables}
  \renewcommand{\listadesiglasname}{List of Abbreviations and Acronyms}
  \renewcommand{\listadesimbolosname}{List of Symbols}
  \renewcommand{\contentsname}{Contents}
  \renewcommand{\bibname}{References}
  \renewcommand{\glossarioname}{Glossary}
  \renewcommand{\apendicename}{APPENDIX}
  \renewcommand{\apendicesname}{Appendices}
  \renewcommand{\anexoname}{ANNEX}
  \renewcommand{\anexosname}{Annexes}
  \renewcommand{\indexname}{Index}
  \renewcommand{\orientadorname}{Supervisor:}
  \renewcommand{\coorientadorname}{Co-supervisor:}
  \renewcommand{\fontename}{Source}
  \renewcommand{\notaname}{Note}
  \renewcommand{\pageautorefname}{page}
  \renewcommand{\sectionautorefname}{section}
  \renewcommand{\subsectionautorefname}{subsection}
  \renewcommand{\subsubsectionautorefname}{subsubsection}
  \renewcommand{\paragraphautorefname}{subsubsubsection}
}

\addto\captionsbrazil{
  \renewcommand{\capaname}{Capa}
  \renewcommand{\folhaderostoname}{Folha de Rosto}
  \renewcommand{\fichacatalograficaname}{Ficha Catalográfica}
  \renewcommand{\errataname}{Errata}
  \renewcommand{\folhadeaprovacaoname}{Folha de Aprovação}
  \renewcommand{\dedicatorianame}{Dedicatória}
  \renewcommand{\agradecimentosname}{Agradecimentos}
  \renewcommand{\epigraphname}{Epígrafe}
  \renewcommand{\resumoname}{Resumo}
  \renewcommand{\resumoestrangeironame}{Abstract}
  \renewcommand{\listfigurename}{Lista de Figuras}
  \renewcommand{\listtablename}{Lista de Tabelas}
  \renewcommand{\listadesiglasname}{Lista de Abreviaturas e Siglas}
  \renewcommand{\listadesimbolosname}{Lista de Símbolos}
  \renewcommand{\contentsname}{Sumário}
  \renewcommand{\bibname}{Referências}
  \renewcommand{\glossarioname}{Glossário}
  \renewcommand{\apendicename}{APÊNDICE}
  \renewcommand{\apendicesname}{Apêndices}
  \renewcommand{\anexoname}{ANEXO}
  \renewcommand{\anexosname}{Anexos}
  \renewcommand{\indexname}{Índice}
  \renewcommand{\orientadorname}{Orientador:}
  \renewcommand{\coorientadorname}{Coorientador:}
  \renewcommand{\fontename}{Fonte}
  \renewcommand{\notaname}{Nota}
  \renewcommand{\pageautorefname}{página}
  \renewcommand{\sectionautorefname}{seção}
  \renewcommand{\subsectionautorefname}{subseção}
  \renewcommand{\subsubsectionautorefname}{subsubseção}
  \renewcommand{\paragraphautorefname}{subsubsubseção}
}

\addto\captionsspanish{
  \renewcommand{\capaname}{Portada}
  \renewcommand{\folhaderostoname}{Página de título}
  \renewcommand{\fichacatalograficaname}{Ficha Catalográfica}
  \renewcommand{\errataname}{Errata}
  \renewcommand{\folhadeaprovacaoname}{Hoja de aprobación}
  \renewcommand{\dedicatorianame}{Dedicatoria}
  \renewcommand{\agradecimentosname}{Agradecimientos}
  \renewcommand{\epigraphname}{Epígrafe}
  \renewcommand{\resumoname}{Resumen}
  \renewcommand{\resumoestrangeironame}{Resumo}
  \renewcommand{\listfigurename}{Lista de Figuras}
  \renewcommand{\listtablename}{Lista de Tablas}
  \renewcommand{\listadesiglasname}{Lista de Abreviaturas y Siglas}
  \renewcommand{\listadesimbolosname}{Lista de Símbolos}
  \renewcommand{\contentsname}{Sumario}
  \renewcommand{\bibname}{Referencias}
  \renewcommand{\glossarioname}{Glosario}
  \renewcommand{\apendicename}{APÉNDICE}
  \renewcommand{\apendicesname}{Apéndices}
  \renewcommand{\anexoname}{ANEXO}
  \renewcommand{\anexosname}{Anexos}
  \renewcommand{\indexname}{Índice}
  \renewcommand{\orientadorname}{Asesor:}
  \renewcommand{\coorientadorname}{Coasesor:}
  \renewcommand{\fontename}{Fuente}
  \renewcommand{\notaname}{Nota}
  \renewcommand{\pageautorefname}{página}
  \renewcommand{\sectionautorefname}{sección}
  \renewcommand{\subsectionautorefname}{subsección}
  \renewcommand{\subsubsectionautorefname}{subsubsección}
  \renewcommand{\paragraphautorefname}{subsubsubsección}
}

\addto\captionsfrench{
  \renewcommand{\capaname}{Couverture}
  \renewcommand{\folhaderostoname}{Page de Titre}
  \renewcommand{\fichacatalograficaname}{Fiche Cataloguée}
  \renewcommand{\errataname}{Errata}
  \renewcommand{\folhadeaprovacaoname}{Feuille d'Approbation}
  \renewcommand{\dedicatorianame}{Dédiace
  \renewcommand{\agradecimentosname}{Remerciements}}
  \renewcommand{\epigraphname}{Épigraphe}
  \renewcommand{\resumoname}{Résumé}
  \renewcommand{\resumoestrangeironame}{Resumo}
  \renewcommand{\listfigurename}{Liste des Figures}
  \renewcommand{\listtablename}{Liste des Tableaux}
  \renewcommand{\listadesiglasname}{Liste des Abréviations et Sigles}
  \renewcommand{\listadesimbolosname}{Liste des Symboles}
  \renewcommand{\contentsname}{Sommaire}
  \renewcommand{\bibname}{Références}
  \renewcommand{\glossarioname}{Glossaire}
  \renewcommand{\apendicename}{APPENDICE}
  \renewcommand{\apendicesname}{Appendices}
  \renewcommand{\anexoname}{ANNEXE}
  \renewcommand{\anexosname}{Annexes}
  \renewcommand{\indexname}{Index}
  \renewcommand{\orientadorname}{Conseiller:}
  \renewcommand{\coorientadorname}{Co-conseiller:}
  \renewcommand{\fontename}{Source}
  \renewcommand{\notaname}{Note}
  \renewcommand{\pageautorefname}{page}
  \renewcommand{\sectionautorefname}{section}
  \renewcommand{\subsectionautorefname}{sous-section}
  \renewcommand{\subsubsectionautorefname}{sous-sous-section}
  \renewcommand{\paragraphautorefname}{sous-sous-sous-section}
}
% See `babel.tex` for language changes.
% See `toc.text` for changes related to the ToC.

% Set Page Numbering -----

\makepagestyle{abntheadings}
\makeevenhead{abntheadings}{\ABNTEXfontereduzida\thepage}{}{}
\makeoddhead{abntheadings}{}{}{\ABNTEXfontereduzida\thepage}

% Set Text Variables -----

\renewcommand{\ABNTEXpartfont}{\sffamily\bfseries}
\renewcommand{\ABNTEXpartfontsize}{\normalsize}
\renewcommand{\ABNTEXchapterfont}{\sffamily\bfseries}
\renewcommand{\ABNTEXchapterfontsize}{\normalsize}
\renewcommand{\ABNTEXsectionfont}{\sffamily}
\renewcommand{\ABNTEXsectionfontsize}{\normalsize}
\renewcommand{\ABNTEXsubsectionfont}{\sffamily}
\renewcommand{\ABNTEXsubsectionfontsize}{\normalsize}
\renewcommand{\ABNTEXsubsubsectionfont}{\sffamily}
\renewcommand{\ABNTEXsubsubsectionfontsize}{\normalsize}
\renewcommand{\ABNTEXsubsubsubsectionfont}{\sffamily}
\renewcommand{\ABNTEXsubsubsubsectionfontsize}{\normalsize\itshape}
\renewcommand{\ABNTEXfontereduzida}{\footnotesize}
\renewcommand{\ABNTEXcaptiondelim}{~\textendash~}
\renewcommand{\ABNTEXcaptionfontedelim}{:~}

\renewcommand{\captiontitlefont}{\ABNTEXfontereduzida}

% Set New Commands -----

\providecommand{\imprimiruniversidade}{}
\newcommand{\universidade}[1]{\renewcommand{\imprimiruniversidade}{#1}}

\providecommand{\imprimirescola}{}
\newcommand{\escola}[1]{\renewcommand{\imprimirescola}{#1}}

\providecommand{\imprimirprograma}{}
\newcommand{\programa}[1]{\renewcommand{\imprimirprograma}{#1}}

\newcommand{\imprimirtipodetrabalho}{\imprimirtipotrabalho}

\providecommand{\imprimirtipodetituloacademico}{}
\newcommand{\tipodetituloacademico}[1]{\renewcommand{\imprimirtipodetituloacademico}{#1}}

\providecommand{\imprimirtituloacademico}{}
\newcommand{\tituloacademico}[1]{\renewcommand{\imprimirtituloacademico}{#1}}

\providecommand{\imprimirareadeconcentracao}{}
\newcommand{\areadeconcentracao}[1]{\renewcommand{\imprimirareadeconcentracao}{#1}}

\providecommand{\imprimirnotadeversao}{}
\newcommand{\notadeversao}[1]{\renewcommand{\imprimirnotadeversao}{#1}}

% Set Chapter Style -----

\renewcommand{\chapnamefont}{\ABNTEXchapterfont\ABNTEXchapterfontsize\mdseries}
\renewcommand{\chapnumfont}{\ABNTEXchapterfont\ABNTEXchapterfontsize\mdseries}

\setsecnumformat{\chapnumfont\csname the#1\endcsname\quad}

\renewcommand{\printchaptername}{
  \ifthenelse{\boolean{abntex@apendiceousecao}}{
    \vspace*{\medskipamount}
    \chapnamefont \ABNTEXchapterupperifneeded{\appendixname} % [Changed]
  }{}
}

% Open an issue about it (`\hspace{-1em}`) - Title stretching.
\renewcommand{\chapternamenum}{
  \ifthenelse{\boolean{abntex@apendiceousecao}}{
    \hspace{-2em} \space
  }{}
}

\renewcommand{\printchapternum}{
  \tocprintchapter
  \setboolean{abntex@innonumchapter}{false}
  \chapnumfont
  \thechapter % [Changed]
  % \ifthenelse{\boolean{abntex@apendiceousecao}}{ % [Removed]
  %   \tocinnonumchapter
  %   \ABNTEXcaptiondelim
  % }{}
}

\renewcommand{\afterchapternum}{
  \ifthenelse{\boolean{abntex@apendiceousecao}}{ % [Added]
    \ABNTEXchapterfont\mdseries \hspace{-1em} \space\ABNTEXcaptiondelim\space \hspace{-1.5em}
  }{
    \hspace{-0.875em}
  }
}

\renewcommand{\printchapternonum}{
  \tocprintchapternonum
  \setlength{\afterchapskip}{\hugeskipamount} % [Added]
  \setboolean{abntex@innonumchapter}{true}
}

\renewcommand{\printchaptertitle}[1]{
  \chaptitlefont
  \ifthenelse{\boolean{abntex@innonumchapter}}{
    \centering \ABNTEXchapterupperifneeded{#1}
  }{
    \ifthenelse{\boolean{abntex@apendiceousecao}}{
      \ABNTEXchapterfont\mdseries\ABNTEXchapterupperifneeded{#1}
    }{
      \ABNTEXchapterupperifneeded{#1}
    }
  }
}

% Set `\textual` -----

\renewcommand{\textual}{
  \pagestyle{abntheadings}
  \aliaspagestyle{chapter}{abntheadings}
}

% Set Cover -----

\renewcommand{\imprimircapa}{
  \phantomsection\pdfbookmark[0]{\capaname}{}
  \begin{capa}%
  \begin{adjustwidth}{-1cm}{0cm}
  \center
  \imprimirinstituicao

  \vfill
  \imprimirautor

  \vfill
  {\ABNTEXchapterfont\imprimirtitulo}

  \vfill
  \vspace{6.5cm}
  \imprimirlocal

  \imprimirdata
  \vspace{1.5cm}
  \end{adjustwidth}
  \end{capa}
}

% Set Title Page -----

\makeatletter
\renewcommand{\folhaderostocontent}{
  \begin{center}
  \imprimirautor

  \vfill
  {\ABNTEXchapterfont\imprimirtitulo}

  \vfill
  \textbf{\imprimirnotadeversao}

  \vfill
  \abntex@ifnotempty{
    \imprimirpreambulo
  }{
    \hspace{0.35\textwidth}
    \begin{minipage}{.6\textwidth}
    \SingleSpacing
    \imprimirpreambulo
    \end{minipage}
  }

  \vfill
  \imprimirlocal

  \imprimirdata
  \vspace{1cm}
  \end{center}
}
\makeatother

% Set Cataloging Record -----

\renewenvironment{fichacatalografica}{
  \PRIVATEbookmarkthis{\fichacatalograficaname}
  \setlength{\parindent}{0cm}
  \begin{SingleSpacing}
}{
  \end{SingleSpacing}
}

% Set Errata -----

\renewenvironment{errata}[1][\errataname]{
  \newpage
  \phantomsection
  \pretextualchapter{#1}
}{
  \cleardoublepage
}

% Set Approval Sheet -----

\renewenvironment{folhadeaprovacao}[1][\folhadeaprovacaoname]{
  \clearpage
  \PRIVATEbookmarkthis{#1}
  \setlength\parindent{0cm}
  \AtBeginEnvironment{tabular}{\normalsize}
  \begin{SingleSpace}
}{
  \end{SingleSpace}
  \cleardoublepage
}

% Set Abstract -----

\newenvironment{resumoenv}[1][\resumoname]{
  \pretextualchapter{#1}
  \begingroup
  \setlength{\parindent}{0cm}
  \setlength{\parskip}{\smallskipamount} % The troublemaker.
  \AtBeginEnvironment{tabular}{\normalsize}
  \renewcommand{\arraystretch}{1}
  \setlength{\aboverulesep}{0ex}
  \setlength{\belowrulesep}{0ex}
  \setlength{\arrayrulewidth}{0pt}
  \setlength{\tabcolsep}{0cm}
  \vspace{-\smallskipamount} % !
  \begin{SingleSpace}
}{
  \end{SingleSpace}
  \cleardoublepage
  \endgroup
}

% Set List of Abbreviations and Acronyms -----

\renewenvironment{siglas}{
  \pretextualchapter{\listadesiglasname}
}{
  \cleardoublepage
}

% Set List of Symbols -----

\renewenvironment{simbolos}{
  \pretextualchapter{\listadesimbolosname}
}{
  \cleardoublepage
}

% Set Glossary -----

\newenvironment{glossario}{
  \tocprintchapternonum
}{
  \cleardoublepage
}

% Set Appendices and Annexes -----

\renewcommand{\PRIVATEapendiceconfig}[2]{
  \setboolean{abntex@apendiceousecao}{true}
  \renewcommand{\appendixname}{#1}
  %\renewcommand{\apendicesname}{#1}

  \ifthenelse{\boolean{ABNTEXsumario-abnt-6027-2012}}{
    \renewcommand{\appendixtocname}{\uppercase{#2}}
  }{
    \renewcommand{\appendixtocname}{#2}
  }

  \renewcommand{\appendixpagename}{#2}
  \renewcommand{\appendixtocname}{#2}
  % \switchchapname{#1} % [Altered]
  \renewcommand{\cftappendixname}{} % [Altered]
  \tocpartapendices % [Added]

  % Note:
  %
  % \cleardoublepage
  % \phantomsection
  % \addcontentsline{toc}{part}{Appendices}
  % \appendix
  %
  % is automatically add by the Quarto render.
}

\newcommand{\PRIVATEapendiceconfigafter}[1]{
    \chapterstyle{apendice}
    %\begingroup\centering\bfseries
    %\ABNTEXchapterupperifneeded{#1}
    %\par\endgroup
    %\vspace{\medskipamount}
    \pretextualchapter{#1}
    \let\clearpage\relax
}

\renewcommand{\apendices}{
  \clearpage
  \PRIVATEapendiceconfig{\apendicename}{\apendicesname}
  \appendix
  \PRIVATEapendiceconfigafter{\apendicesname}
}

\renewenvironment{apendicesenv}{
  \clearpage
  \PRIVATEapendiceconfig{\apendicename}{\apendicesname}
  \begin{appendix}
  \PRIVATEapendiceconfigafter{\apendicesname}
}{
  \end{appendix}
  \setboolean{abntex@apendiceousecao}{false}
  \bookmarksetup{startatroot}
}

\renewcommand{\anexos}{
  \clearpage
  % \cftinserthook{toc}{AAA} [Removed]
  \PRIVATEapendiceconfig{\anexoname}{\anexosname}

  \newpage % [Added]
  \phantomsection % [Added]
  \addcontentsline{toc}{part}{\appendixtocname} % [Added]

  \appendix
  \renewcommand\theHchapter{anexochapback.\arabic{chapter}}
  \PRIVATEapendiceconfigafter{\anexosname}
}

\renewenvironment{anexosenv}{
  \clearpage
  \PRIVATEapendiceconfig{\anexoname}{\anexosname}

  \newpage % [Added]
  \phantomsection % [Added]
  \addcontentsline{toc}{part}{\appendixtocname} % [Added]

  \begin{appendix}
  \renewcommand\theHchapter{anexochapback.\arabic{chapter}}
  \PRIVATEapendiceconfigafter{\anexosname}
}{
  \end{appendix}
  \setboolean{abntex@apendiceousecao}{false}
  \bookmarksetup{startatroot}
}
% -----
% New Colors
% -----

\definecolor{blue}{HTML}{2905C3}

% See <https://getbootstrap.com/docs/5.0/utilities/colors/>.
\definecolor{quarto-blue}{HTML}{2780E3}
\definecolor{quarto-lighter-blue}{HTML}{ECF4FC}
\definecolor{quarto-orange}{HTML}{FF7518}
\definecolor{quarto-ligther-orange}{HTML}{FFF3EB}
\definecolor{quarto-red}{HTML}{D9534F}
\definecolor{quarto-ligther-red}{HTML}{FCF1F1}
\definecolor{quarto-green}{HTML}{3FB618}
\definecolor{quarto-ligther-green}{HTML}{EFF9EB}
\definecolor{quarto-purple}{HTML}{7D12BA}
\definecolor{quarto-gray}{HTML}{A3A3A3}
\definecolor{quarto-medium-gray}{HTML}{CFD0D1}
\definecolor{quarto-ligther-gray}{HTML}{F1F3F5}

\definecolor{bs-link-color}{HTML}{39729E}

% -----
% Body Color
% -----


% Quarto's Default Settings -----

% \usepackage{graphicx} % Already loaded in `packages.tex`.
\makeatletter
\newsavebox\pandoc@box
\newcommand*\pandocbounded[1]{% scales image to fit in text height/width
  \sbox\pandoc@box{#1}%
  \Gscale@div\@tempa{\textheight}{\dimexpr\ht\pandoc@box+\dp\pandoc@box\relax}%
  \Gscale@div\@tempb{\linewidth}{\wd\pandoc@box}%
  \ifdim\@tempb\p@<\@tempa\p@\let\@tempa\@tempb\fi% select the smaller of both
  \ifdim\@tempa\p@<\p@\scalebox{\@tempa}{\usebox\pandoc@box}%
  \else\usebox{\pandoc@box}%
  \fi%
}
% Set default figure placement to `htbp`
\def\fps@figure{htbp}
\makeatother

% Set Distance from Top of the Page to First Float -----

\makeatletter
\setlength{\@fptop}{5pt}
\makeatother

% Set Captions and Legends -----

\DeclareCaptionFont{ABNTEXfontereduzida}{\ABNTEXfontereduzida}

% For customization, see `\DeclareCaptionFormat` in the `caption` package.
\captionsetup{
  font=ABNTEXfontereduzida
  ,justification=justified
}

\renewcommand{\abovecaptionskip}{\smallskipamount}
\renewcommand{\belowcaptionskip}{\smallskipamount}

\renewcommand{\legend}[1]{
  \hyphenpenalty=100000
  \ABNTEXfontereduzida
  \addvspace{\smallskipamount}
  #1
}

% Credits: <https://tex.stackexchange.com/a/611556/234832>.
\AddToHook{cmd/caption/before}{\hyphenpenalty=100000}

% Set figure environment -----

\AtBeginEnvironment{figure}{
  \ABNTEXfontereduzida
  \addvspace{\tinyskipamount}
}

\AtEndEnvironment{figure}{
  \addvspace{\smallskipamount}
}
\renewcommand{\arraystretch}{1.5}
\setlength{\aboverulesep}{0ex}
\setlength{\belowrulesep}{0ex}

% Correct order of tables after \paragraph or \subparagraph
% \usepackage{etoolbox}
% \makeatletter
% \patchcmd\longtable{\par}{\if@noskipsec\mbox{}\fi\par}{}{}
% \makeatother

% Allow footnotes in `longtable` head/foot
\IfFileExists{footnotehyper.sty}{\usepackage{footnotehyper}}{\usepackage{footnote}}
\makesavenoteenv{longtable}

% Set Tabular Environment -----

\AtBeginEnvironment{table}{\ABNTEXfontereduzida}
\AtBeginEnvironment{tabular}{\ABNTEXfontereduzida}

\AtBeginEnvironment{longtable}{\ABNTEXfontereduzida \addvspace{\tinyskipamount}}
\AtBeginEnvironment{longtable*}{\ABNTEXfontereduzida \addvspace{\tinyskipamount}}

\floatplacement{table}{H}

% Set Theorem Environment -----

\AtEndEnvironment{theorem}{\vspace{\bigskipamount}}
\providecommand{\tightlist}{
\setlength{\itemsep}{0ex}\setlength{\parskip}{0\baselineskip}}

% \setlist[enumerate]{leftmargin=1cm)}
% \setlist[itemize]{leftmargin=2cm}
\makeatletter
\newcommand*{\getlength}[1]{\strip@pt#1}
\makeatother
\title{
abnt: Quarto Format for ABNT Theses and Dissertations

}

\titulo{
abnt: Quarto Format for ABNT Theses and Dissertations
}


\author{Daniel Vartanian}
\autor{Daniel Vartanian}

\local{{[}City{]}}

\date{2026}
\data{2026}

\orientador{{[}Supervisor's full name{]}}

\coorientador{{[}Co-supervisor's full name{]}}

\tipodetituloacademico{{[}Master/PhD{]}}

\tituloacademico{{[}Master of Science/Doctor of Science{]}}

\tipotrabalho{{[}Dissertation/Thesis{]}}

\areadeconcentracao{{[}Area of concentration{]}}

\instituicao{\MakeUppercase{{[}University{]}}}
\universidade{{[}University{]}}

\instituicao{
  \MakeUppercase{{[}University{]}}
  \par
  \MakeUppercase{{[}School/Department{]}}
}

\escola{{[}School/Department{]}}

\instituicao{
  \MakeUppercase{{[}University{]}}
  \par
  \MakeUppercase{{[}School/Department{]}}
  \par
  \MakeUppercase{{[}Graduate program{]}}
}

\programa{{[}Graduate program{]}}

\notadeversao{{[}Original/Revised version{]}}

\hypersetup{
pdftitle={abnt: Quarto Format for ABNT Theses and Dissertations},
pdfauthor={Daniel Vartanian},
pdflang={en},
pdfsubject={{[}Dissertation/Thesis{]}},
linktoc={section},
colorlinks=true,
linkcolor={blue},
filecolor={blue},
citecolor={blue},
urlcolor={blue},
pdfcreator={LaTeX via pandoc},
bookmarksdepth=5
}
% Set Sections Skips (`\cftchapterpresnum`)

\setlength{\cftbeforebookskip}{0\baselineskip}
\setlength{\cftbeforepartskip}{\bigskipamount}
\setlength{\cftbeforechapterskip}{\microskipamount}
\setlength{\cftbeforesectionskip}{0\baselineskip}
\setlength{\cftbeforesubsectionskip}{0\baselineskip}
\setlength{\cftbeforesubsubsectionskip}{0\baselineskip}
\setlength{\cftbeforeparagraphskip}{0\baselineskip}

% Set Section Numbers Fonts (`\cftchapterpresnum`)

\renewcommand{\cftchapterpresnum}{\normalfont}
\renewcommand{\cftsectionpresnum}{\normalfont}
\renewcommand{\cftsubsectionpresnum}{\normalfont}
\renewcommand{\cftsubsubsectionpresnum}{\normalfont}
\renewcommand{\cftparagraphpresnum}{\normalfont}

% Set Section Names Fonts (`\cftpartfont`)

\renewcommand{\cftpartfont}[1]{
  \ABNTEXchapterupperifneeded{\normalfont\bfseries #1}
}

\renewcommand{\cftchapterfont}[1]{
  \ABNTEXchapterupperifneeded{\normalfont\bfseries #1}
}

\renewcommand{\cftsectionfont}[1]{
  \ABNTEXsectionupperifneeded{\normalfont #1}
}

\renewcommand{\cftsubsectionfont}[1]{
  \ABNTEXsubsectionupperifneeded{\normalfont\bfseries #1}
}

\renewcommand{\cftsubsubsectionfont}[1]{
  \ABNTEXsubsubsectionupperifneeded{\normalfont #1}
}

\renewcommand{\cftparagraphfont}[1]{
  \ABNTEXsubsubsubsectionupperifneeded{\normalfont\itshape #1}
}

% Set Section Page Numbers Fonts (`\cftpartpagefont`)

\renewcommand{\cftpartpagefont}{\normalfont}
\renewcommand{\cftchapterpagefont}{\normalfont}
\renewcommand{\cftsectionpagefont}{\normalfont}
\renewcommand{\cftsubsectionpagefont}{\normalfont}
\renewcommand{\cftsubsubsectionpagefont}{\normalfont}
\renewcommand{\cftparagraphpagefont}{\normalfont}
\renewcommand{\cftfigurepagefont}{\normalfont}
\renewcommand{\cfttablepagefont}{\normalfont}

% Renew `abntex2` ToC Commands -----

\cftinsertcode{A}{} % [Changed]

% This is not right. Create an issue about it.
\renewcommand{\tocprintchapternonum}{
  \addtocontents{toc}{\setlength{\cftchapterindent}{5.65em}}
  \addtocontents{toc}{\setlength{\cftchapternumwidth}{0em}}
}

\renewcommand{\tocpartapendices}{
  \addtocontents{toc}{\setlength{\cftpartindent}{5.65em}}
  \addtocontents{toc}{\setlength{\cftpartnumwidth}{0em}}
}

% Set ToC Skip

\newcommand{\tocskipone}{
  \addtocontents{toc}{\protect\vspace{\smallskipamount}}
}

% \setlength{\cftbeforepartskip}{\bigskipamount}
\newcommand{\tocskiptwo}{
  % \addtocontents{toc}{\protect\vspace{\tinyskipamount}}
}
\usepackage[
,backend=biber,url=true,useprefix=false,giveninits=true,style=abnt
]{biblatex}

\usepackage{csquotes}

\addbibresource{references.bib}

\renewcommand{\bibname}{REFERENCES}
\newcommand{\newbibname}{REFERENCES}

\newcommand{\bibnamewithfootnote}{
  \newbibname\protect\footnote{According to the Brazilian Association of
Technical Standards (ABNT NBR 6023).}
}

\setlength{\bibhang}{0cm}

\setlength{\bibparsep}{-2ex}


\defbibheading{bibheading}[\bibnamewithfootnote]{
  \ifthenelse{\boolean{ABNTEXupperchapter}}{
    \setboolean{ABNTEXupperchapter}{false}
    \chapter*{#1}
    \markboth{#1}{#1}
    \setboolean{ABNTEXupperchapter}{true}
  }{
    \chapter*{#1}
    \markboth{#1}{#1}
  }
}

\AtBeginBibliography{\vspace{0.5\baselineskip}}
\AtEveryBibitem{\clearfield{annotation}}
\renewcommand{\bibfont}{\ABNTEXfontereduzida}
\usepackage{makeidx}
\makeindex

% Set sections skips (`\cftchapterpresnum`)

% Credits: https://tex.stackexchange.com/a/28361/234832
% \begin{luacode}
% local PENALTY=node.id("penalty")
% last_line_twice_parindent = function (head)
%   while head do
%     local _w,_h,_d = node.dimensions(head)
%     if head.id == PENALTY and head.subtype ~= 15 and (_w < 2 * tex.parindent) then
%
%         -- we are at a glue and have less than 2*\parindent to go
%         local p = node.new("penalty")
%         p.penalty = 10000
%         p.next = head
%         head.prev.next = p
%         p.prev = head.prev
%         head.prev = p
%     end
%
%     head = head.next
%   end
%   return true
% end
%
% luatexbase.add_to_callback("pre_linebreak_filter",last_line_twice_parindent,"Raphink")
% \end{luacode}
% Use upquote if available, for straight quotes in verbatim environments
\IfFileExists{upquote.sty}{\usepackage{upquote}}{}
\IfFileExists{microtype.sty}{% use microtype if available
  \usepackage[]{microtype}
  \UseMicrotypeSet[protrusion]{basicmath} % disable protrusion for tt fonts
}{}





\setlength{\emergencystretch}{3em} % Prevent overfull lines

\setcounter{secnumdepth}{5}

% Make \paragraph and \subparagraph free-standing
\ifx\paragraph\undefined\else
  \let\oldparagraph\paragraph
  \renewcommand{\paragraph}[1]{\oldparagraph{#1}\mbox{}}
\fi
\ifx\subparagraph\undefined\else
  \let\oldsubparagraph\subparagraph
  \renewcommand{\subparagraph}[1]{\oldsubparagraph{#1}\mbox{}}
\fi

\usepackage{color}
\usepackage{fancyvrb}
\newcommand{\VerbBar}{|}
\newcommand{\VERB}{\Verb[commandchars=\\\{\}]}
\DefineVerbatimEnvironment{Highlighting}{Verbatim}{commandchars=\\\{\}}
% Add ',fontsize=\small' for more characters per line
\usepackage{framed}
\definecolor{shadecolor}{RGB}{241,243,245}
\newenvironment{Shaded}{\begin{snugshade}}{\end{snugshade}}
\newcommand{\AlertTok}[1]{\textcolor[rgb]{0.68,0.00,0.00}{#1}}
\newcommand{\AnnotationTok}[1]{\textcolor[rgb]{0.37,0.37,0.37}{#1}}
\newcommand{\AttributeTok}[1]{\textcolor[rgb]{0.40,0.45,0.13}{#1}}
\newcommand{\BaseNTok}[1]{\textcolor[rgb]{0.68,0.00,0.00}{#1}}
\newcommand{\BuiltInTok}[1]{\textcolor[rgb]{0.00,0.23,0.31}{#1}}
\newcommand{\CharTok}[1]{\textcolor[rgb]{0.13,0.47,0.30}{#1}}
\newcommand{\CommentTok}[1]{\textcolor[rgb]{0.37,0.37,0.37}{#1}}
\newcommand{\CommentVarTok}[1]{\textcolor[rgb]{0.37,0.37,0.37}{\textit{#1}}}
\newcommand{\ConstantTok}[1]{\textcolor[rgb]{0.56,0.35,0.01}{#1}}
\newcommand{\ControlFlowTok}[1]{\textcolor[rgb]{0.00,0.23,0.31}{\textbf{#1}}}
\newcommand{\DataTypeTok}[1]{\textcolor[rgb]{0.68,0.00,0.00}{#1}}
\newcommand{\DecValTok}[1]{\textcolor[rgb]{0.68,0.00,0.00}{#1}}
\newcommand{\DocumentationTok}[1]{\textcolor[rgb]{0.37,0.37,0.37}{\textit{#1}}}
\newcommand{\ErrorTok}[1]{\textcolor[rgb]{0.68,0.00,0.00}{#1}}
\newcommand{\ExtensionTok}[1]{\textcolor[rgb]{0.00,0.23,0.31}{#1}}
\newcommand{\FloatTok}[1]{\textcolor[rgb]{0.68,0.00,0.00}{#1}}
\newcommand{\FunctionTok}[1]{\textcolor[rgb]{0.28,0.35,0.67}{#1}}
\newcommand{\ImportTok}[1]{\textcolor[rgb]{0.00,0.46,0.62}{#1}}
\newcommand{\InformationTok}[1]{\textcolor[rgb]{0.37,0.37,0.37}{#1}}
\newcommand{\KeywordTok}[1]{\textcolor[rgb]{0.00,0.23,0.31}{\textbf{#1}}}
\newcommand{\NormalTok}[1]{\textcolor[rgb]{0.00,0.23,0.31}{#1}}
\newcommand{\OperatorTok}[1]{\textcolor[rgb]{0.37,0.37,0.37}{#1}}
\newcommand{\OtherTok}[1]{\textcolor[rgb]{0.00,0.23,0.31}{#1}}
\newcommand{\PreprocessorTok}[1]{\textcolor[rgb]{0.68,0.00,0.00}{#1}}
\newcommand{\RegionMarkerTok}[1]{\textcolor[rgb]{0.00,0.23,0.31}{#1}}
\newcommand{\SpecialCharTok}[1]{\textcolor[rgb]{0.37,0.37,0.37}{#1}}
\newcommand{\SpecialStringTok}[1]{\textcolor[rgb]{0.13,0.47,0.30}{#1}}
\newcommand{\StringTok}[1]{\textcolor[rgb]{0.13,0.47,0.30}{#1}}
\newcommand{\VariableTok}[1]{\textcolor[rgb]{0.07,0.07,0.07}{#1}}
\newcommand{\VerbatimStringTok}[1]{\textcolor[rgb]{0.13,0.47,0.30}{#1}}
\newcommand{\WarningTok}[1]{\textcolor[rgb]{0.37,0.37,0.37}{\textit{#1}}}

\newcolumntype{P}[1]{>{\centering\arraybackslash}p{#1}}

\clubpenalty10000
\widowpenalty10000
\displaywidowpenalty10000

\ifLuaTeX
  \usepackage{selnolig}  % disable illegal ligatures
\fi


\IfFileExists{xurl.sty}{\usepackage{xurl}}{} % add URL line breaks if available
\urlstyle{same} % disable monospaced font for URLs


% -----
% Custom Functions
% -----

% Credits: <https://tex.stackexchange.com/a/300215/234832>.

\usepackage{xparse}

\ExplSyntaxOn
\NewExpandableDocumentCommand{\repeatntimes}{O{}mm}
 {
  \int_compare:nT { #2 > 0 }
   {
    #3 \prg_replicate:nn { #2 - 1 } { #1#3 }
   }
 }
\ExplSyntaxOff
% -----
% Title Page
% -----

\preambulo{
\hyphenpenalty=100000

%:::% title-page body begin %:::%
{\imprimirtipotrabalho} presented to the {\imprimirescola} at the {\imprimiruniversidade}, as a requirement for the degree of {\imprimirtituloacademico} by the {\imprimirprograma}.

\smallskip

Area of concentration: {\imprimirareadeconcentracao}

\smallskip

Supervisor: Prof. Dr. {\imprimirorientador}

\vspace{\tinyskipamount}

Co-Supervisor: Prof. Dr. {\imprimircoorientador}
%:::% title-page body end %:::%
}
\usepackage{booktabs}
\usepackage{caption}
\usepackage{longtable}
\usepackage{colortbl}
\usepackage{array}
\usepackage{anyfontsize}
\usepackage{multirow}
\makeatletter
\@ifpackageloaded{tcolorbox}{}{\usepackage[skins,breakable]{tcolorbox}}
\@ifpackageloaded{fontawesome5}{}{\usepackage{fontawesome5}}
\definecolor{quarto-callout-color}{HTML}{909090}
\definecolor{quarto-callout-note-color}{HTML}{0758E5}
\definecolor{quarto-callout-important-color}{HTML}{CC1914}
\definecolor{quarto-callout-warning-color}{HTML}{EB9113}
\definecolor{quarto-callout-tip-color}{HTML}{00A047}
\definecolor{quarto-callout-caution-color}{HTML}{FC5300}
\definecolor{quarto-callout-color-frame}{HTML}{acacac}
\definecolor{quarto-callout-note-color-frame}{HTML}{4582ec}
\definecolor{quarto-callout-important-color-frame}{HTML}{d9534f}
\definecolor{quarto-callout-warning-color-frame}{HTML}{f0ad4e}
\definecolor{quarto-callout-tip-color-frame}{HTML}{02b875}
\definecolor{quarto-callout-caution-color-frame}{HTML}{fd7e14}
\makeatother
\makeatletter
\@ifpackageloaded{bookmark}{}{\usepackage{bookmark}}
\makeatother
\makeatletter
\@ifpackageloaded{caption}{}{\usepackage{caption}}
\AtBeginDocument{%
\ifdefined\contentsname
  \renewcommand*\contentsname{Table of Contents}
\else
  \newcommand\contentsname{Table of Contents}
\fi
\ifdefined\listfigurename
  \renewcommand*\listfigurename{List of Figures}
\else
  \newcommand\listfigurename{List of Figures}
\fi
\ifdefined\listtablename
  \renewcommand*\listtablename{List of Tables}
\else
  \newcommand\listtablename{List of Tables}
\fi
\ifdefined\figurename
  \renewcommand*\figurename{Figure}
\else
  \newcommand\figurename{Figure}
\fi
\ifdefined\tablename
  \renewcommand*\tablename{Table}
\else
  \newcommand\tablename{Table}
\fi
}
\@ifpackageloaded{float}{}{\usepackage{float}}
\floatstyle{ruled}
\@ifundefined{c@chapter}{\newfloat{codelisting}{h}{lop}}{\newfloat{codelisting}{h}{lop}[chapter]}
\floatname{codelisting}{Listing}
\newcommand*\listoflistings{\listof{codelisting}{List of Listings}}
\usepackage{amsthm}
\theoremstyle{plain}
\newtheorem{theorem}{Theorem}[chapter]
\theoremstyle{remark}
\AtBeginDocument{\renewcommand*{\proofname}{Proof}}
\newtheorem*{remark}{Remark}
\newtheorem*{solution}{Solution}
\newtheorem{refremark}{Remark}[chapter]
\newtheorem{refsolution}{Solution}[chapter]
\makeatother
\makeatletter
\makeatother
\makeatletter
\@ifpackageloaded{caption}{}{\usepackage{caption}}
\@ifpackageloaded{subcaption}{}{\usepackage{subcaption}}
\makeatother
\makeatletter
\@ifpackageloaded{tcolorbox}{}{\usepackage[skins,breakable]{tcolorbox}}
\makeatother
\makeatletter
\@ifundefined{shadecolor}{\definecolor{shadecolor}{HTML}{CFD0D1}}{}
\makeatother
\makeatletter
\@ifundefined{codebgcolor}{\definecolor{codebgcolor}{HTML}{F1F3F5}}{}
\makeatother
\makeatletter
\ifdefined\Shaded\renewenvironment{Shaded}{\begin{tcolorbox}[enhanced, borderline west={3pt}{0pt}{shadecolor}, sharp corners, colback={codebgcolor}, frame hidden, breakable, boxrule=0pt]}{\end{tcolorbox}}\fi
\makeatother

% -----
% Body
% -----

\begin{document}

% Top matter -----

\pretextual

\frenchspacing

\selectlanguage{english}

%:::% class attribute begin/end %:::%

% -----
% Cover (Mandatory)
% -----

%:::% cover begin %:::%
\imprimircapa
%:::% cover end %:::%

% -----
% Title Page (Mandatory)
% -----

%:::% approval-sheet begin %:::%
\imprimirfolhaderosto
%:::% approval-sheet end %:::%

% -----
% Cataloging Record (Mandatory)
% -----

%:::% cataloging-record begin %:::%
\begin{fichacatalografica}
\hyphenpenalty=100000
%:::% cataloging-record body begin %:::%
I authorize the full or partial reproduction of this work by any conventional or electronic means for the purposes of study and research, provided that the source is cited.

\vfill

\begin{center}
Cataloging in publication

[University. School. Library]

[CRB-8]

\medskip
\ABNTEXfontereduzida

\setlength{\fboxsep}{1cm}
\fbox{\begin{minipage}[c][6.5cm]{12.5cm}
[Author's surname], [Author's forename(s)]

\smallskip

\hspace{0.5cm} {\imprimirtitulo}  / {\imprimirautor}; supervisor, {\imprimirorientador}; co-supervisor {\imprimircoorientador}. {\imprimirlocal}, {\imprimirdata}

\smallskip

{\thelastpage}p. : il

\smallskip

\hspace{0.5cm} {\imprimirtipotrabalho} (\imprimirtituloacademico) -- {\imprimirprograma}, {\imprimirescola}, {\imprimiruniversidade}, {\imprimirdata}.

\smallskip

\hspace{0.5cm} {\imprimirnotadeversao}

\smallskip

\hspace{0.5cm} 1. [Subject A]. 2. [Subject B]. 3. [Subject C]. I. [Supervisor's surname], [Supervisor's forename(s)], super. II. [Co-supervisor's surname], [Co-supervisor's forename(s)] co-super. III. Title.
\end{minipage}}
\end{center}
\vspace{\hugeskipamount}
%:::% cataloging-record body end %:::%
\end{fichacatalografica}
%:::% cataloging-record end %:::%

% -----
% Errata (Optional)
% -----

%:::% errata begin %:::%
\begin{errata}[\errataname]
%:::% errata reference begin %:::%
\noindent [SURNAME], [Forename(s) initial(s)]. \textbf{\imprimirtitulo}. {\imprimirdata}. {\thelastpage}p. {\imprimirtipotrabalho}  ({\imprimirtituloacademico}) -- {\imprimirescola}, {\imprimiruniversidade}, {\imprimirlocal}, {\imprimirdata}.
%:::% errata reference end %:::%
\smallskip
%:::% errata body begin %:::%

This is the preliminary version of this thesis (version \textless1.0.0).
Any required corrections will be listed here upon approval.

%:::% errata body end %:::%
\end{errata}
%:::% errata end %:::%

% -----
% Approval Sheet (Mandatory)
% -----

%:::% approval-sheet begin %:::%
\begin{folhadeaprovacao}[\folhadeaprovacaoname]
\hyphenpenalty=100000
%:::% approval-sheet body begin %:::%
{\imprimirtipotrabalho} by {\imprimirautor}, under the title \textbf{\imprimirtitulo}, presented to the {\imprimirescola} at the {\imprimiruniversidade}, as a requirement for the degree of {\imprimirtituloacademico} by the {\imprimirprograma}, in the concentration area of {\imprimirareadeconcentracao}.

\vspace{\hugeskipamount}
Approved on [Month] [Day], [Year].

\vspace{\hugeskipamount}
\begin{center}
  Examination Committee
\end{center}

\vspace{\smallskipamount}
Committee Chair:

\vspace{\tinyskipamount}
\begingroup

\AtBeginEnvironment{tabular}{
  \normalsize\raggedright
  \renewcommand{\arraystretch}{2}
}

\setlength{\arrayrulewidth}{0pt}
\setlength{\tabcolsep}{0cm}
\begin{tabular}{m{2.5cm} m{13.5cm}}
  Prof. Dr. & [Full name] \\
  Institution & [School], [University] \\
\end{tabular}

\vspace{\bigskipamount}
Examiners:

\vspace{\tinyskipamount}
\begin{tabular}{m{2.5cm} m{13.5cm}}
  Prof. Dr. & [Full name] \\
  Institution & [School], [University] \\
  Evaluation & [Approved/Rejected] \\
\end{tabular}

\vspace{\smallskipamount}
\begin{tabular}{m{2.5cm} m{13.5cm}}
  Prof. Dr. & [Full name] \\
  Institution & [School], [University] \\
  Evaluation & [Approved/Rejected] \\
\end{tabular}

\vspace{\smallskipamount}
\begin{tabular}{m{2.5cm} m{13.5cm}}
  Prof. Dr. & [Full name] \\
  Institution & [School], [University] \\
  Evaluation & [Approved/Rejected] \\
\end{tabular}
\endgroup
%:::% approval-sheet body end %:::%
\end{folhadeaprovacao}
%:::% approval-sheet end %:::%

% -----
% Inscription (Optional)
% -----

%:::% inscription begin %:::%
\begin{dedicatoria}[] % \dedicatorianame | Keep #1 empty.
\vspace*{\fill} % Don't change it.
\centering
%:::% inscription body begin %:::%
\textit{To the worm that first gnawed on the cold flesh of my corpse,}

\textit{I dedicate, as a fond remembrance, these posthumous memories.}\footnotemark{}

\footnotetext{
	ASSIS, M. \textbf{Memórias póstumas de Brás Cubas} [The Posthumous Memoirs of Brás Cubas]. São Paulo: Companhia das Letras, 2014.
}
%:::% inscription body end %:::%
\vspace*{\fill} % Don't change it.
\vspace{4.5cm}
% \vspace{13cm}
\end{dedicatoria}
%:::% inscription end %:::%

% -----
% Acknowledgments (Optional)
% -----

%:::% acknowledgments begin %:::%
\begin{agradecimentos}[\agradecimentosname]
  %:::% acknowledgments body begin %:::%

I would like to acknowledge this awesome
\href{https://github.com/danielvartan/abnt}{Quarto format}! :)

  %:::% acknowledgments body end %:::%
\end{agradecimentos}
%:::% acknowledgments end %:::%

% -----
% Epigraph (Optional)
% -----

%:::% epigraph begin %:::%
\begin{epigrafe}[] % \epigraphname | Keep #1 empty.
\vspace*{\fill} % Don't change it.
\begin{flushright}
%:::% epigraph body begin %:::%
\textit{Nullius in verba}\footnotemark{}

\footnotetext{
	THE ROYAL SOCIETY. \textbf{History of the Royal Society}. Available from: <\href{https://royalsociety.org/about-us/history/}{https://royalsociety.org/about-us/history/}>. Visited on: 9 Sept. 2023.
}
%:::% epigraph body end %:::%
\end{flushright}
\end{epigrafe}
%:::% epigraph end %:::%

% -----
% Abstract in Vernacular Language (Mandatory)
% -----

%:::% vernacular-abstract begin %:::%
\begin{resumoenv}[\resumoname]
 %:::% vernacular-abstract reference begin %:::%
[SURNAME], [Forename(s) initial(s)]. \textbf{\imprimirtitulo}. {\imprimirdata}. {\thelastpage}p. {\imprimirtipotrabalho}  (\imprimirtituloacademico) -- {\imprimirescola}, {\imprimiruniversidade}, {\imprimirlocal}, {\imprimirdata}.
%:::% vernacular-abstract reference end %:::%

%:::% vernacular-abstract body begin %:::%

\href{https://github.com/danielvartan/abnt}{abnt} is a
\href{https://quarto.org}{Quarto} format designed for creating theses
and dissertations that comply with guidelines established by the
Brazilian Association of Technical Standards
(\href{https://www.abnt.org.br/}{ABNT}). It's based on the
\href{https://www.abntex.net.br/}{abntex2} LaTeX class, which belongs to
the \href{https://www.ctan.org/pkg/memoir}{memoir} class family.

%:::% vernacular-abstract body end %:::%

%:::% vernacular-abstract keywords begin %:::%
\begin{tabular}{p{2.3cm} p{13.6cm}}
  \textbf{Keywords}: & [Keyword 1]. [Keyword 2]. [Keyword 3].
\end{tabular}
%:::% vernacular-abstract keywords end %:::%
\end{resumoenv}
%:::% vernacular-abstract end %:::%

% -----
% Abstract in Foreign Language (Mandatory)
% -----

%:::% foreign-abstract begin %:::%
\begin{resumoenv}[\resumoestrangeironame]
\begin{otherlanguage*}{brazil}
%:::% foreign-abstract reference begin %:::%
[SOBRENOME], [Inicial(is) do(s) prenome(s)].  \textbf{[Título]}. {\imprimirdata}. {\thelastpage}p. [Dissertação/Tese]  ([Título acadêmico]) -- [Escola/Faculdade], [Universidade], [Cidade/Local], {\imprimirdata}.
%:::% foreign-abstract reference end %:::%

%:::% foreign-abstract body begin %:::%

\href{https://github.com/danielvartan/abnt}{abnt} é um formato
\href{https://quarto.org}{Quarto} projetado para criar teses e
dissertações que atendem às diretrizes estabelecidas pela Associação
Brasileira de Normas Técnicas (\href{https://www.abnt.org.br/}{ABNT}).
Ele é baseado na classe LaTeX
\href{https://www.abntex.net.br/}{abntex2}, que pertence à família de
classes \href{https://www.ctan.org/pkg/memoir}{memoir}.

%:::% foreign-abstract body end %:::%

%:::% foreign-abstract keywords begin %:::%
\begin{tabular}{p{3.6cm} p{12.3cm}}
  \textbf{Palavras-chaves}: & [Palavra-chave 1]. [Palavra-chave 2]. [Palavra-chave 3].
\end{tabular}
%:::% foreign-abstract keywords end %:::%
\end{otherlanguage*}
\end{resumoenv}
%:::% foreign-abstract end %:::%

% -----
% List of Figures (Optional)
% -----

%:::% list-of-figures begin %:::%
\pdfbookmark[0]{\listfigurename}{lof}
\listoffigures*
\cleardoublepage
%:::% list-of-figures end %:::%

% -----
% List of Tables (Optional)
% -----

%:::% list-of-tables begin %:::%
\pdfbookmark[0]{\listtablename}{lot}
\listoftables*
\cleardoublepage
%:::% list-of-tables end %:::%

% -----
% List of Abbreviations and Acronyms (Optional)
% -----

%:::% list-of-abbreviations begin %:::%
\begin{siglas}
%:::% list-of-abbreviations body begin %:::%

\begin{description}
\item[\textsubscript{F}]
\hspace{20cm}

Subscript indicating a relation with work-free days.
\item[\textsubscript{W}]
\hspace{20cm}

Subscript indicating a relation with workdays.
\item[MCTQ]
\hspace{20cm}

Munich ChronoType Questionnaire.
\item[MCTQ\textsuperscript{PT}]
\hspace{20cm}

Portuguese version of the MCTQ.
\item[MEQ]
\hspace{20cm}

Morningness-Eveningness Questionnaire.
\item[MSF]
\hspace{20cm}

Local time of mid-sleep on work-free days.
\item[MSF\textsubscript{sc}]
\hspace{20cm}

Chronotype proxy. The midpoint between sleep onset and sleep end on
work-free days. A sleep correction (\textsubscript{SC}) is made when a
possible sleep compensation related to a lack of sleep on workdays is
identified.
\item[MSW]
\hspace{20cm}

Local time of mid-sleep on workdays.
\end{description}

%:::% list-of-abbreviations body end %:::%
\end{siglas}
%:::% list-of-abbreviations end %:::%

% -----
% List of Symbols (Optional)
% -----

%:::% list-of-symbols begin %:::%
\begin{simbolos}
%:::% list-of-symbols body begin %:::%

For an extensive list of chronobiology related symbols, please refer to
\textcite{aschoff1965} and \textcite{marques2012a}.

\begin{description}
\item[\(\tau\)]
\hspace{20cm}

Period of a rhythm in free flow; only revealed under constant
environmental conditions.
\item[\(T\)]
\hspace{20cm}

Zeitgeber period.
\item[\(\phi\)]
\hspace{20cm}

Phase.
\item[\(\Delta\phi\)]
\hspace{20cm}

Phase shift.
\item[\(+\Delta\phi\)]
\hspace{20cm}

Phase advance.
\item[\(-\Delta\phi\)]
\hspace{20cm}

Phase delay.
\item[\(\Psi\)]
\hspace{20cm}

Phase relation.
\end{description}

%:::% list-of-symbols body end %:::%
\end{simbolos}
%:::% list-of-symbols end %:::%

% -----
% Table of Contents (Mandatory)
% -----

%:::% table-of-contents begin %:::%
\pdfbookmark[0]{\contentsname}{toc}
\tableofcontents*
\cleardoublepage
%:::% table-of-contents end %:::%

% -----
% Other Additions
% -----


% Main and back matter -----

\textual
\bookmarksetup{startatroot}

\chapter{{[}Showcase{]} Introduction}\label{sec-introduction}

See Figure~\ref{fig-karl-popper}.

``The activity can be represented by a \emph{general schema of
problem-solving by the method of imaginative conjectures and criticism},
or, as I have often called it, by \emph{the method of conjecture and
refutation}. The schema (in its simplest form) is this

\[
\text{P}_{1} \to \text{TT} \to \text{EE} \to \text{P}_{2}
\]

\vspace{1em}
\index{KarlPopper}

Here \(\text{P}_{1}\) is the \emph{problem} from which we start,
\(\text{TT}\) (the `tentative theory') is the imaginative conjectural
solution which we first reach, for example our first \emph{tentative
interpretation}. \(\text{EE}\) (\emph{`error- elimination'}) consists of
a severe critical examination of our conjecture, our tentative
interpretation: it consists, for example, of the critical use of
documentary evidence and, if we have at this early stage more than one
conjecture at our disposal, it will also consist of a critical
discussion and comparative evaluation of the competing conjectures.
\(\text{P}_{2}\) is the problem situation as it emerges from our first
critical attempt to solve our problems.

It leads up to our second attempt (\emph{and so on}). A satisfactory
understanding will be reached if the interpretation, the conjectural
theory, finds support in the fact that it can throw new light on new
problems --- on more problems than we expected; or if it finds support
in the fact that it explains many sub-problems, some of which were not
seen to start with. Thus we may say that we can gauge the progress we
have made by comparing \(\text{P}_{1}\) with some of our later problems
(\(\text{P}_{n}\), say).''

\vspace{15pt}
\noindent \hspace*{\fill} \autocite[164]{popper1979a}
\vspace{15pt}

Ipsum tempor laborum pariatur amet occaecat et culpa nisi eiusmod
laborum ad culpa reprehenderit. Esse adipisicing et consectetur eu elit
reprehenderit irure veniam. Quis ipsum ullamco id ea cillum. Fugiat elit
ut laboris do ex culpa enim adipisicing id deserunt ullamco consectetur
commodo nostrud. Veniam incididunt Lorem velit pariatur. Ipsum esse ut
irure pariatur consequat pariatur sunt ad quis. Est ipsum dolore id
pariatur ea commodo eiusmod non minim veniam. Est officia cupidatat
ullamco elit cillum magna excepteur nostrud. Commodo anim qui elit
occaecat anim veniam ipsum dolor consequat nostrud velit consequat.

\index{figures}

\begin{figure}[H]

\caption{\label{fig-karl-popper}Karl Popper (July 25, 1902 -- September
17, 1994).\\
One of the 20th century's most influential philosophers of science.}

\centering{

\includegraphics[width=0.5\linewidth,height=\textheight,keepaspectratio]{images/karl-popper.png}

\legend{Source:
\href{https://www.npg.org.uk/collections/search/portrait/mw08238/Sir-Karl-Raimund-Popper?}{Steve
Pyke}.}

}

\end{figure}%

\section{Secondary Section}\label{secondary-section}

See Table~\ref{tbl-penguins}.

Irure cillum ut nisi dolore voluptate esse ullamco deserunt. Mollit
nulla ad eu amet fugiat velit mollit. Ea irure quis veniam sint ut
cillum aliqua ipsum mollit dolore. Labore fugiat ullamco reprehenderit
cupidatat adipisicing in pariatur dolor consectetur adipisicing proident
culpa. Minim non deserunt sit occaecat.

\index{tables}

\begin{table}

\caption{\label{tbl-penguins}A sample of the penguins dataset}

\centering{

\fontsize{12.0pt}{14.0pt}\selectfont
\begin{tabular*}{\linewidth}{@{\extracolsep{\fill}}ccrrr}
\toprule
Species & Island & Bill length (mm) & Bill depth (mm) & Flipper length (mm) \\ 
\midrule\addlinespace[2.5pt]
Gentoo & Biscoe & 43.5 & 14.2 & 220 \\ 
Adelie & Torgersen & 36.2 & 17.2 & 187 \\ 
Chinstrap & Dream & 58.0 & 17.8 & 181 \\ 
Adelie & Dream & 37.0 & 16.9 & 185 \\ 
Adelie & Biscoe & 38.1 & 16.5 & 198 \\ 
\bottomrule
\end{tabular*}

\legend{Source: Based on \textcite{horst2020} penguin dataset.}

}

\end{table}%

\subsection{Tertiary Section}\label{tertiary-section}

Ullamco duis irure ipsum duis minim incididunt. Elit duis non qui sit
sit. Labore consectetur laboris magna reprehenderit sint magna eiusmod
occaecat. Sit aute ipsum laborum exercitation mollit cillum nulla Lorem.
Quis adipisicing aliquip labore aute excepteur ut voluptate in proident
culpa. Laborum minim quis tempor mollit excepteur excepteur.

Culpa ex ut excepteur velit sunt et quis eu magna duis laboris ullamco
id. Laboris officia exercitation elit incididunt amet exercitation.
Consectetur eiusmod excepteur do Lorem et velit id. Aute dolore ipsum
eiusmod ipsum veniam pariatur cillum ex adipisicing quis exercitation
deserunt esse ea. Ut reprehenderit deserunt consectetur occaecat aute
Lorem et voluptate ullamco. Ipsum aliquip qui elit id proident
adipisicing nostrud in anim.

\index{figures!chart}

\begin{figure}[H]

\caption{\label{fig-eruption}Relation between \emph{waiting time to next
eruption} (minutes) and \emph{eruption time} (minutes) at Old Faithful
Geyser, Yellowstone National Park, Wyoming, USA}

\centering{

\pandocbounded{\includegraphics[keepaspectratio]{index_files/figure-pdf/unnamed-chunk-5-1.png}}

\legend{Source: Reproduced from the
\href{https://ggplot2.tidyverse.org/reference/geom_density_2d.html}{ggplot2
R package documentation} \autocite{wickham2016a}.}

}

\end{figure}%

\clearpage

\subsubsection{Quaternary Section}\label{quaternary-section}

\begin{itemize}
\tightlist
\item
  Bullet point

  \begin{itemize}
  \tightlist
  \item
    Bullet point

    \begin{itemize}
    \tightlist
    \item
      Bullet point
    \end{itemize}
  \end{itemize}
\end{itemize}

\paragraph{Quinary Section}\label{quinary-section}

\begin{enumerate}
\def\labelenumi{\arabic{enumi}.}
\tightlist
\item
  List
\item
  List
\item
  List
\end{enumerate}

\section{Another Secondary Section}\label{another-secondary-section}

See Figure~\ref{fig-mpg}.

Laboris eu quis in ut ipsum. Laborum qui tempor laboris ad cillum velit
ea minim adipisicing magna labore mollit officia. Mollit cillum id do
sint irure ut velit amet aliqua ipsum mollit adipisicing. Reprehenderit
deserunt mollit consequat qui laborum occaecat laboris. Deserunt
pariatur Lorem est reprehenderit excepteur aliqua esse in nostrud aute
velit. Veniam cupidatat aute id adipisicing aute proident ut
reprehenderit. Cillum non et est deserunt quis commodo.

Tempor sint dolore ea adipisicing id voluptate est duis est consectetur
Lorem. Dolore tempor dolor esse sit nulla fugiat aute labore aute
laboris. Aliquip duis duis voluptate do et consectetur ad culpa sint
proident. Id in elit deserunt dolore est est ullamco laboris commodo
labore consectetur qui. Officia ut do nisi excepteur ex. Proident
aliquip laboris culpa esse excepteur nisi. Adipisicing ex qui cillum ad
commodo dolor quis ipsum ea anim. Deserunt culpa adipisicing ea ex
adipisicing ad id. Ullamco tempor voluptate fugiat qui Lorem est eu
reprehenderit tempor irure aute consequat eiusmod aliqua. Ex Lorem
exercitation anim elit in ut consectetur ad.

Consectetur consequat mollit esse laborum aliquip laborum veniam in
duis. Deserunt nostrud duis ipsum adipisicing pariatur nulla et
consectetur qui laborum deserunt irure duis. Duis nulla ad quis sint do
esse reprehenderit. Enim nulla exercitation sunt excepteur Lorem labore
adipisicing anim elit nisi officia velit ut ut. Sit reprehenderit
officia quis sit ut elit duis enim. Sit irure anim deserunt laboris
proident ea labore in sit culpa labore. Aliqua ea sit magna dolore.

Sint do ad minim officia. Magna proident commodo ipsum nostrud pariatur
exercitation eiusmod. Laboris et ut eiusmod quis ex. Laborum non magna
ex cupidatat voluptate ullamco. Enim cillum ipsum minim sint velit
veniam magna nostrud sit. Dolor eiusmod ut et incididunt commodo
proident sint laborum laborum eiusmod amet. Laboris in incididunt non
sit dolore amet laborum do ad esse enim. Laboris laboris cupidatat ad
aliqua dolore nostrud nostrud consectetur proident. Ut fugiat qui velit
quis cillum enim mollit anim quis. Lorem et do excepteur aliqua ad.

\index{figures!chart}

\begin{figure}[H]

\caption{\label{fig-mpg}Relation between \emph{weight (1000lbs)} and
\emph{miles per gallon} for combustion engine vehicles}

\centering{

\pandocbounded{\includegraphics[keepaspectratio]{index_files/figure-pdf/unnamed-chunk-7-1.png}}

\legend{Source: Data extracted from the 1974 Motor Trend magazine,
published by \textcite{henderson1981}. Visualization by
\textcite{holtz}, available at
\href{https://r-graph-gallery.com/277-marginal-histogram-for-ggplot2.html}{The
R Graph Gallery}.}

}

\end{figure}%

\bookmarksetup{startatroot}

\chapter{{[}Showcase{]} Development}\label{sec-development}

Aliqua aute non qui sint reprehenderit ullamco. Sunt fugiat voluptate
laborum adipisicing est ad. Nisi irure ex nostrud officia officia
laboris veniam. In occaecat non amet do veniam. Amet officia pariatur
sint consequat dolor deserunt elit quis ad. Velit sint ut in cupidatat
eiusmod velit voluptate voluptate culpa exercitation veniam. Minim
proident mollit quis deserunt officia cillum consectetur ipsum.

Fugiat enim ut consequat eu enim cupidatat pariatur labore adipisicing
quis irure pariatur esse voluptate. Elit sunt amet cillum quis
consectetur ad cillum incididunt officia quis. Culpa in deserunt aliquip
quis. Reprehenderit veniam voluptate nisi ullamco culpa. Amet incididunt
Lorem ut nulla ad. Ipsum pariatur cupidatat irure sit amet minim qui
excepteur aliquip est cillum labore quis commodo.

\section{Secondary Section}\label{secondary-section-1}

Tempor do consequat tempor minim elit fugiat aute sit ea aliqua laborum
ullamco occaecat reprehenderit. Laboris ex commodo enim excepteur ut
consectetur elit ullamco eu qui pariatur ullamco sint. Excepteur
cupidatat do esse nisi sint nulla duis dolor et consectetur mollit. Sunt
cillum eiusmod excepteur eiusmod nostrud nisi magna. Do anim elit ex
irure cillum qui proident dolor labore cillum incididunt duis culpa
voluptate. Incididunt do sint labore dolore eiusmod veniam proident
consequat commodo aliquip aliqua excepteur ex.

Ex officia minim laborum adipisicing dolor elit velit. Minim dolore anim
pariatur sint veniam adipisicing aute est sint magna exercitation magna.
Dolore deserunt officia dolore quis cupidatat minim cupidatat excepteur
Lorem nisi aliquip elit do cupidatat. Lorem voluptate deserunt do culpa
dolor id qui et tempor eu esse reprehenderit fugiat. Ullamco in aliquip
nulla ex eu fugiat consequat aliqua et adipisicing in laborum. Est
laborum occaecat amet exercitation cillum proident irure. Id irure elit
id occaecat qui aliqua nisi veniam irure cupidatat Lorem sint fugiat
qui.

\bookmarksetup{startatroot}

\chapter{{[}Showcase{]} Conclusion}\label{sec-conclusion}

In anim magna nostrud incididunt proident esse esse. Consequat velit
consequat ex consequat veniam occaecat. Duis nisi Lorem adipisicing
dolore dolore cupidatat esse eiusmod culpa cupidatat deserunt ex.
Pariatur aliquip ea enim aliquip. Labore ut esse incididunt veniam
voluptate pariatur laborum ea nostrud dolor labore labore deserunt
deserunt. Culpa dolore in cillum elit veniam irure officia irure magna.
Ipsum minim culpa ipsum ea ut dolore ex est.

In ex Lorem voluptate fugiat. Aliquip qui ex fugiat quis non quis ut. Id
nostrud proident ex aliqua Lorem aliqua laboris. Et ea eu ullamco aute
cillum ullamco eiusmod excepteur culpa tempor adipisicing veniam
adipisicing. Deserunt quis cillum dolore proident minim. Quis
exercitation in minim irure. Enim anim esse et mollit. Commodo ipsum sit
nulla Lorem eiusmod aliqua commodo elit magna incididunt. Ea cupidatat
excepteur exercitation eiusmod esse mollit officia.

\section{Secondary Section}\label{secondary-section-2}

Aute laboris eiusmod ea sunt excepteur in nisi ipsum. Nostrud ea quis ex
ullamco aliqua enim. Laborum excepteur fugiat anim voluptate nulla
fugiat sint sint minim laborum elit excepteur. Exercitation eiusmod
irure cillum elit nulla ad sint ad enim non enim do consectetur anim.
Dolore nostrud esse proident magna dolore elit duis commodo laboris
excepteur esse.

Excepteur ipsum ex qui occaecat minim dolore elit sunt tempor. Nulla
occaecat est nisi irure voluptate id mollit sunt nisi. In qui qui
adipisicing et. Ex ipsum dolore aute minim. Ad aliquip fugiat amet
exercitation ea aliquip aliqua pariatur. Labore minim dolor ex eu veniam
amet laboris proident esse. Id et elit ex enim eu ullamco est
adipisicing nostrud. Elit adipisicing consequat ullamco minim mollit sit
quis minim sit esse quis. Cillum cupidatat cillum reprehenderit
voluptate ex deserunt ut est quis aute irure esse velit. Aliquip ad elit
labore ea amet amet et excepteur voluptate veniam tempor id nostrud
sint.

\postextual

\begingroup
\renewcommand{\baselinestretch}{1}
\setcounter{footnote}{0}
\renewcommand{\thefootnote}{\fnsymbol{footnote}}
\printbibliography[heading=bibheading]
\endgroup

\tocskipone
\tocprintchapternonum
\addcontentsline{toc}{chapter}{\newbibname}

\begin{glossario}

\bookmarksetup{startatroot}

\chapter*{Glossary}\label{glossary}
\addcontentsline{toc}{chapter}{Glossary}

\markboth{Glossary}{Glossary}

For an extensive list of chronobiology related terms and definitions,
please refer to \textcite{aschoff1965} and \textcite{marques2012a}.

\begin{description}
\item[Chronotype]
\hspace{20cm}

Any kind of temporal phenotype \autocite{ehret1974,pittendrigh1993}.
Usually, it refers to circadian phenotypes in a spectrum that goes from
morningness to eveningness \autocite{roenneberg2003b}. It can also be
seen as an organism's phase of entrainment \autocite{roenneberg2012a}.
\item[Circadian rhythm]
\hspace{20cm}

A rhythm with a period close to a day/24h, an approximation to the
period of the earth's rotation \autocite{pittendrigh1960}. From the
Latin \emph{circā}, around, and \emph{dĭes}, day \autocite{latinitium}.
Example: the sleep-wake cycle.
\item[Complex system]
\hspace{20cm}

There are several definitions. Here are some that I found to be of use:
\end{description}

\begin{itemize}
\tightlist
\item
  ``Systems that don't yield to compact forms of representation or
  description'' (David Krakauer apud \textcite{mitchell2013})
\item
  ``A system of many interacting parts where the system is more than
  just the sum of its parts'' (Mark Newman apud \textcite{mitchell2013})
\end{itemize}

\begin{description}
\item[Entrainment]
\hspace{20cm}

A shift and alignment of biological rhythms induced by a zeitgeber input
\autocite{kuhlman2018}. For example: a shift/alignment of an organism's
circadian rhythm when exposed to light.
\end{description}

\end{glossario}

\begin{apendicesenv}

\cleardoublepage
\phantomsection
\addcontentsline{toc}{part}{Appendices}
\appendix

\chapter{{[}Showcase{]}}\label{sec-appendix-a}

Ipsum do culpa sit mollit Lorem voluptate ad exercitation proident
aliqua. Anim enim quis consequat magna aliquip nulla. Commodo culpa
voluptate aute magna ea anim aute adipisicing elit eu do consectetur.
Tempor mollit cillum voluptate reprehenderit ullamco officia enim
voluptate elit. Laboris anim consectetur culpa ex id irure et cillum
labore qui sint. Reprehenderit officia reprehenderit incididunt aliquip
aute aliqua proident ipsum dolore ipsum. Do tempor culpa occaecat ipsum
consequat quis. Officia et laborum dolor eu minim id quis elit non
labore amet anim.

Non quis proident et reprehenderit proident. Proident sint labore tempor
incididunt quis deserunt ex incididunt nostrud qui elit pariatur.
Proident cupidatat quis commodo magna cupidatat cupidatat. Ut quis eu ea
Lorem velit laborum dolor laborum esse. Consequat velit in laboris
ullamco aliqua cupidatat duis consequat. Eu amet tempor enim
exercitation aliquip esse fugiat nostrud culpa. Voluptate quis officia
Lorem deserunt elit fugiat ullamco amet duis proident. Sint eu eiusmod
proident magna.

\section{Secondary Section}\label{secondary-section-3}

Anim ut est ea aliqua culpa officia ut fugiat. Duis laboris mollit non
irure velit fugiat tempor enim. Ad reprehenderit laborum adipisicing
occaecat adipisicing excepteur deserunt eu ea irure enim deserunt. Et
commodo laborum nisi aliqua consequat dolore occaecat minim. Ea dolore
culpa consequat incididunt mollit aute magna ipsum Lorem veniam pariatur
occaecat nisi sit. Occaecat esse do sint minim ut elit minim consectetur
dolor Lorem labore anim exercitation. Dolor qui labore exercitation non
magna dolore duis Lorem aute esse culpa ad ad nisi.

\end{apendicesenv}

\begin{anexosenv}

\chapter{{[}Showcase{]}}\label{sec-annex-a}

\clearpage
\includepdf{images/annex.pdf}

\end{anexosenv}

\chapter{Settings}\label{sec-settings}

\begin{tcolorbox}[enhanced jigsaw, title=\textcolor{quarto-callout-important-color}{\faExclamation}\hspace{0.5em}{Important}, colbacktitle=quarto-callout-important-color!10!white, toprule=.15mm, left=2mm, colframe=quarto-callout-important-color-frame, leftrule=.75mm, bottomrule=.15mm, coltitle=black, titlerule=0mm, breakable, colback=white, bottomtitle=1mm, arc=.35mm, toptitle=1mm, rightrule=.15mm, opacitybacktitle=0.6, opacityback=0]

You are reading the work-in-progress of this manual.

\microskip

This chapter is undergoing heavy restructuring and may be confusing or
incomplete.

\end{tcolorbox}

\section{Sections}\label{sections}

If you won't use a section of this document, you must remove it from
\texttt{\_quarto.yml}. Empty sections will produce an error.

At the moment, you also need to remove them in
\texttt{tex/include-before-body.tex} and \texttt{R/.pre-render.R}.

\section{Typography}\label{typography}

\index{tipography}

\subsection{Typeface}\label{typeface}

To change typefaces, simply use the
\href{https://quarto.org/docs/reference/formats/pdf.html}{Quarto
options}, such as \texttt{mainfont}, \texttt{monofont} and
\texttt{sansfont} in your \texttt{quarto-{[}format{]}.yml} file.

\begin{Shaded}
\begin{Highlighting}[numbers=left,,]
\FunctionTok{format}\KeywordTok{:}
\AttributeTok{  }\FunctionTok{abnt{-}pdf}\KeywordTok{:}
\AttributeTok{    }\FunctionTok{mainfont}\KeywordTok{:}\AttributeTok{ Arial}
\end{Highlighting}
\end{Shaded}

The ABNT NBR 14724:2011 norm does not specify the use of any specific
font. You have the freedom to choose any font you prefer, but it's
important to note that the selected font must be installed on your
computer.

\subsection{Font Size}\label{font-size}

To adjust the font size, utilize the \texttt{fontsize} option in in the
\texttt{quarto-{[}format{]}.yml} file.

\begin{Shaded}
\begin{Highlighting}[numbers=left,,]
\FunctionTok{format}\KeywordTok{:}
\AttributeTok{  }\FunctionTok{abnt{-}pdf}\KeywordTok{:}
\AttributeTok{    }\FunctionTok{fontsize}\KeywordTok{:}\AttributeTok{ 12pt}
\end{Highlighting}
\end{Shaded}

It's important to note that the third paragraph of Section 5.1 of ABNT
NBR 14724:2011 norm establishes that the font size should be 12pt for
the entire document, including the cover, except for quotations longer
than three lines, footnotes, pagination, cataloging data, captions, and
sources of illustrations and tables, which should be in a smaller and
uniform size.

The smaller font is set to \texttt{\textbackslash{}footnotesize}, which
corresponds to a 10pt font size with the default settings. You can
modify this setting by inserting the following LaTeX command into
\texttt{tex/include-in-header.tex}:

\begin{Shaded}
\begin{Highlighting}[numbers=left,,]
\FunctionTok{\textbackslash{}renewcommand}\NormalTok{\{}\ExtensionTok{\textbackslash{}ABNTEXfontereduzida}\NormalTok{\}\{[NEW SIZE (e.g., }\FunctionTok{\textbackslash{}small}\NormalTok{)]\}}
\end{Highlighting}
\end{Shaded}

\section{Language and Hyphenation}\label{language-and-hyphenation}

\section{Document Sections}\label{document-sections}

\subsection{Editing Pre-Textual
Sections}\label{editing-pre-textual-sections}

\texttt{abnt} uses a system of tags to transfer and render the content
of Quarto files (\texttt{.qmd}) to LaTeX. These tags look like this:

\begin{Shaded}
\begin{Highlighting}[numbers=left,,]
\NormalTok{\textasciigrave{}\textasciigrave{}\textasciigrave{}\{=latex\}}
\CommentTok{\%:::\% class attribute begin/end \%:::\%}
\NormalTok{\textasciigrave{}\textasciigrave{}\textasciigrave{}}
\end{Highlighting}
\end{Shaded}

Unless you want to customize the template, you don't need to modify the
\texttt{.tex} files. You can write directly in the \texttt{.qmd} files.
Just ensure that you preserve all the tags.

\subsection{\texorpdfstring{How to Include LaTeX Commands in Quarto
files
(\texttt{.qmd})}{How to Include LaTeX Commands in Quarto files (.qmd)}}\label{how-to-include-latex-commands-in-quarto-files-.qmd}

To add LaTeX commands in your writing use a \texttt{\{=latex\}} chunk.

\begin{Shaded}
\begin{Highlighting}[numbers=left,,]
\NormalTok{\textasciigrave{}\textasciigrave{}\textasciigrave{}\{=latex\}}
\CommentTok{\% Some LaTeX code.}
\NormalTok{\textasciigrave{}\textasciigrave{}\textasciigrave{}}
\end{Highlighting}
\end{Shaded}

\subsection{How to Add or Remove
Sections}\label{how-to-add-or-remove-sections}

For pre-textual sections (e.g., list of symbols, abstract), remove them
from \texttt{tex/include-before-body.tex} and from
\texttt{R/quarto-pre-render.R}.

For textual sections (e.g., chapters), remove them from
\texttt{.quarto-{[}format{]}.yml} file.

For post-textual sections (e.g., appendices, annexes):

\begin{itemize}
\tightlist
\item
  If it's the Glossary, remove it from \texttt{.quarto-{[}format{]}.yml}
  and copy the the LaTeX code after
  \texttt{\textless{}!-\/-\ glossary\ end\ -\/-\textgreater{}} in
  \texttt{glossary.qmd} to the bottom of the last chapter;
\item
  If it's not the last appendix chapter, simply remove it from
  \texttt{.quarto-{[}format{]}.yml}; else remove it from
  \texttt{.quarto-{[}format{]}.yml}, remove
  \texttt{\textbackslash{}begin\{apendicesenv\}} from the bottom of
  \texttt{glossary.qmd} and add the code the code after
  \texttt{\textless{}!-\/-\ appendices\ end\ -\/-\textgreater{}} of the
  appendice file to the bottom of \texttt{glossary.qmd};
\item
  {[}Annexes{]};
\item
  {[}Index{]}.
\end{itemize}

It's important to note that, at this moment, the transition between
sections of the document are made inserting LaTeX code at the end of
specific sections. These are:

\begin{itemize}
\tightlist
\item
  Between the last chapter and the Glossary section.
\item
  Between the Glossary section and the Appendices section.
\item
  Between the Appendices and Annexes section.
\item
  After the Annexes section.
\end{itemize}

\section{Citation Management}\label{citation-management}

\subsection{Citation Method}\label{citation-method}

\index{BibLaTeX}

This Quarto format was specifically designed to be compatible with
\href{https://www.ctan.org/pkg/biblatex}{BibLaTeX}, which is a
comprehensive re-implementation of
\href{https://www.bibtex.org/}{BiBTeX}. At first glance, these two
systems may appear very similar.

To get started, simply insert your references into the
\texttt{references.bib}file. However, this task can be somewhat tedious
and demanding. To simplify the process, we recommend exploring the
integration of \href{https://www.zotero.org/}{Zotero} along with
\href{https://github.com/retorquere/zotero-better-bibtex}{Better
BiBTeX}, as demonstrated in a section below.

For detailed guidance on handling citations in Quarto, please refer to
Quarto's
\href{https://quarto.org/docs/authoring/footnotes-and-citations.html}{Citation
\& Footnotes} documentation.

\subsection{Citation Style}\label{citation-style}

\index{ABNT}
\index{APA}

There are two built-in citation styles:

\begin{itemize}
\tightlist
\item
  \href{https://www.abnt.org.br/}{ABNT} (Brazilian Association of
  Technical Standards);
\item
  \href{https://apastyle.apa.org/}{APA} (American Psychological
  Association).
\end{itemize}

To use one of them, simply change the \texttt{biblio-style} option in
your \texttt{yml} file with the style of you preference.

\begin{Shaded}
\begin{Highlighting}[numbers=left,,]
\FunctionTok{format}\KeywordTok{:}
\AttributeTok{  }\FunctionTok{abnt{-}pdf}\KeywordTok{:}
\AttributeTok{    }\FunctionTok{biblio{-}style}\KeywordTok{:}\AttributeTok{ abnt}\CommentTok{ \# options: [abnt, abnt{-}ibid, abnt{-}numeric, apa]}
\end{Highlighting}
\end{Shaded}

There are other options related to the citation style; some are shown
below. Please refer to
\href{https://www.ctan.org/pkg/biblatex}{\texttt{biblatex}},
\href{https://www.ctan.org/pkg/biblatex-abnt}{\texttt{biblatex-abnt}}
and \href{https://www.ctan.org/pkg/biblatex-apa}{\texttt{biblatex-apa}}
manuals to learn more about them.

\begin{Shaded}
\begin{Highlighting}[numbers=left,,]
\FunctionTok{format}\KeywordTok{:}
\AttributeTok{  }\FunctionTok{abnt{-}pdf}\KeywordTok{:}
\FunctionTok{    biblio{-}footnote}\KeywordTok{: }\CharTok{\textgreater{}}
\NormalTok{      According to the Brazilian Association of Technical Standards}
\NormalTok{      (ABNT NBR 6023).}
\AttributeTok{    }\FunctionTok{biblatexoptions}\KeywordTok{:}
\AttributeTok{      }\KeywordTok{{-}}\AttributeTok{ backend=biber,}
\AttributeTok{      }\KeywordTok{{-}}\AttributeTok{ language=english,}\CommentTok{ \# [options: english, brazil, spanish, french]}
\AttributeTok{      }\KeywordTok{{-}}\AttributeTok{ url=true,}
\AttributeTok{      }\KeywordTok{{-}}\AttributeTok{ useprefix=false,}
\AttributeTok{      }\KeywordTok{{-}}\AttributeTok{ giveninits=true,}
\AttributeTok{      }\KeywordTok{{-}}\AttributeTok{ extrayear=true}
\AttributeTok{    }\FunctionTok{bibhang}\KeywordTok{:}\AttributeTok{ 0cm}\CommentTok{ \# Use 0.5cm if \textasciigrave{}biblio{-}style: apa\textasciigrave{}.}
\AttributeTok{    }\FunctionTok{bibparsep}\KeywordTok{:}\AttributeTok{ 0ex}
\end{Highlighting}
\end{Shaded}

\section{ABNT Figures and Tables}\label{abnt-figures-and-tables}

Thanks for the incredible work of
\href{https://github.com/cscheid}{Carlos Scheidegger} and other Quarto
developers we now have a built-in solution for figures and tables that
require two captions (one at the top and the other at the bottom, or a
caption and a legend), as required by the ABNT norms. Please note that
this feature is only available for Quarto versions \textgreater=v1.4.

The procedure for adding these captions is the same for figures and
tables. Enclose your figure/table/code in figure \texttt{divs}, as shown
in the example below. The first paragraph after the figure content will
be rendered as the source (bottom caption), and the last one will be the
top caption.

The formatting options for this bottom caption/legend is still matter of
debate (see
\href{https://github.com/quarto-dev/quarto-cli/discussions/6103\#discussioncomment-7494661}{here}).
That's why is important to add Quarto's
\href{https://github.com/quarto-ext/latex-environment}{LaTeX
Environment} filter in your
\texttt{\_quarto-pdf.yml"\ with\ the\ command}legend\texttt{and\ use}{[}SOURCE
TEXT GOES HERE{]}.legend\}` when defining legends for figures/tables,
like the example below.

\begin{Shaded}
\begin{Highlighting}[numbers=left,,]
\NormalTok{::: \{\#fig{-}1\}}
\NormalTok{::: \{.figure{-}content\}}
\NormalTok{This is the figure content.}
\NormalTok{:::}

\CommentTok{[}\OtherTok{Source: My source.}\CommentTok{]}\NormalTok{\{.legend\}}

\NormalTok{This is a caption.}
\NormalTok{:::}
\end{Highlighting}
\end{Shaded}

Please note that, like all cross-reference elements, these \texttt{divs}
must follow a naming pattern. Always use the prefixes \texttt{\#fig-}
for figures and \texttt{\#tbl-} for tables.

Visit the showcase chapter ``Introduction''
(\texttt{qmd/introduction.qmd}) of this Quarto format to see this
feature in action. For more detailed information, please refer to
Quarto's
\href{https://quarto.org/docs/prerelease/1.4/crossref.html}{Cross-referenceable
elements} article.

\section{Cross-Referenceable
Elements}\label{cross-referenceable-elements}

Quarto allow you to create and reference almost anything by using div
enclosures.

Example: See Theorem~\ref{thm-line}.

\begin{theorem}[Line]\protect\hypertarget{thm-line}{}\label{thm-line}

The equation of any straight line, called a linear equation, can be
written as:

\[
y = mx + b
\]

\end{theorem}

Although, it's important to note that for this to work, each type of
\texttt{div} must use pre-defined prefixes. If you don't follow these
rules your document will not be rendered.

Here are most of the the label prefixes.

\begin{figure}

\begin{minipage}{0.33\linewidth}

\begin{itemize}
\tightlist
\item
  \texttt{cnj-}: Conjecture
\item
  \texttt{cor-}: Corollary
\item
  \texttt{def-}: Definition
\item
  \texttt{eq-}: Equation
\item
  \texttt{exm-}: Example
\end{itemize}

\end{minipage}%
%
\begin{minipage}{0.33\linewidth}

\begin{itemize}
\tightlist
\item
  \texttt{exr-}: Exercise
\item
  \texttt{fig-}: Figure
\item
  \texttt{lem-}: Lemma
\item
  \texttt{lst-}: Listings
\end{itemize}

\end{minipage}%
%
\begin{minipage}{0.33\linewidth}

\begin{itemize}
\tightlist
\item
  \texttt{prp-}: Proposition
\item
  \texttt{sec-}: Section
\item
  \texttt{tbl-}: Table
\item
  \texttt{thm-}: Theorem
\end{itemize}

\end{minipage}%

\end{figure}%

For more information about cross-reference elements, see Quarto's guide
\href{https://quarto.org/docs/books/book-crossrefs.html}{Book
Crossrefs},
\href{https://quarto.org/docs/authoring/cross-references.html}{Cross
References} and
\href{https://quarto.org/docs/prerelease/1.4/crossref.html}{Cross-referenceable
elements} articles.

\section{Freezing and cache}\label{freezing-and-cache}

See
\href{https://quarto.org/docs/projects/code-execution.html\#freeze}{Freeze}.

\section{How to Customize this Quarto
format}\label{how-to-customize-this-quarto-format}

\subsection{Quarto System}\label{quarto-system}

See \href{https://quarto.org/docs/guide/}{Quarto's guide}.

\subsection{Template Partials}\label{template-partials}

See
\href{https://quarto.org/docs/journals/templates.html\#template-partials}{Template
partials}.

\begin{itemize}
\tightlist
\item
  Set fixed dimensions (e.g., page dimensions) in \texttt{cm} or
  \texttt{pt}. \texttt{cm} is the prefer unit for margins.
\item
  Set line spacing as a proportion of
  \texttt{\textbackslash{}baselineskip} (e.g.,
  \texttt{1.5\textbackslash{}baselineskip}).
\item
  Use the settings \texttt{\textbackslash{}tinyskipamount},
  \texttt{\textbackslash{}smallskipamount},
  \texttt{\textbackslash{}midskipamount},
  \texttt{\textbackslash{}bigskipamount},
  \texttt{\textbackslash{}hugeskipamount} and their counterparts
  \texttt{\textbackslash{}tinyskip}, \texttt{\textbackslash{}smallskip},
  \texttt{\textbackslash{}midskip}, \texttt{\textbackslash{}bigskip},
  \texttt{\textbackslash{}hugeskip}. You can find them in the
  \texttt{lengths.tex} template partial.
\item
  For other kinds of relative vertical spacing, use the \texttt{ex}
  unit.
\item
  For relative horizontal spacing, use the \texttt{em} unit.
\end{itemize}

See \textcite[section 7.5]{oetiker2023} to learn more about LaTeX
spacing features. The articles on
\href{https://www.overleaf.com}{Overleaf} are also a great source of
information. Check
\href{https://www.overleaf.com/learn/latex/Lengths_in_LaTeX}{Lengths in
LaTeX} and
\href{https://www.overleaf.com/learn/latex/Articles/How_to_change_paragraph_spacing_in_LaTeX}{How
to change paragraph spacing in LaTeX} to get a sense of the subject.

The following are the equivalences for a Arial typeface with size
12\texttt{pt}.

\subsubsection{Unit equivalences}\label{unit-equivalences}

\begin{itemize}
\tightlist
\item
  1\texttt{em} \(==\) 12\texttt{pt} or \(\approx\) 0.423333\texttt{cm}.
\item
  1\texttt{ex} \(==\) \(\approx\) 6.22266\texttt{pt} or \(\approx\)
  0.219521\texttt{cm}.
\end{itemize}

\paragraph{\texorpdfstring{\texttt{\textbackslash{}baselineskip}}{\textbackslash baselineskip}}\label{baselineskip}

Use \texttt{\textbackslash{}the\textbackslash{}baselineskip} and
\texttt{\textbackslash{}getvalue\{\}} to figure out the exact value.
Note that \texttt{\textbackslash{}getvalue\{\}} will return the value in
\texttt{pt}.

Example of using \texttt{\textbackslash{}getvalue\{\}}:

\begin{Shaded}
\begin{Highlighting}[numbers=left,,]
\FunctionTok{\textbackslash{}begingroup}
\FunctionTok{\textbackslash{}setlength}\NormalTok{\{}\FunctionTok{\textbackslash{}parskip}\NormalTok{\}\{1em\}}
\FunctionTok{\textbackslash{}getlength}\NormalTok{\{}\FunctionTok{\textbackslash{}parskip}\NormalTok{\}}
\FunctionTok{\textbackslash{}endgroup}
\end{Highlighting}
\end{Shaded}

\begin{itemize}
\tightlist
\item
  \texttt{\textbackslash{}linestretch=1}

  \begin{itemize}
  \tightlist
  \item
    \texttt{1\textbackslash{}baselineskip} \(==\) 14.5\texttt{pt}.
    That's about 1.2x (or (\(\approx\) 1.208333x) the font size
    (standard procedure).
  \end{itemize}
\item
  \texttt{\textbackslash{}linestretch=1.5}

  \begin{itemize}
  \tightlist
  \item
    \texttt{0.25\textbackslash{}baselineskip} \(==\) 5.4375\texttt{pt}
    or \(\approx\) 0.191822917\texttt{cm};
  \item
    \texttt{0.5\textbackslash{}baselineskip} \(==\) 10.875\texttt{pt} or
    \(\approx\) 0.383645833\texttt{cm};
  \item
    \texttt{0.75\textbackslash{}baselineskip} \(==\) 16.3125\texttt{pt}
    or \(\approx\) 0.57546875\texttt{cm};
  \item
    \texttt{1\textbackslash{}baselineskip} \(==\) 21.75\texttt{pt} or
    \(\approx\) 0.76729167\texttt{cm};
  \item
    \texttt{1.5\textbackslash{}baselineskip} \(==\) 32.625\texttt{pt} or
    \(\approx\) 1.1509375\texttt{cm};
  \item
    \texttt{2\textbackslash{}baselineskip} \(==\) 43.5\texttt{pt} or
    \(\approx\) 1.534583\texttt{cm};
  \item
    \texttt{2.5\textbackslash{}baselineskip} \(==\) 54.375\texttt{pt} or
    \(\approx\) 1.91822917\texttt{cm};
  \item
    \texttt{3\textbackslash{}baselineskip} \(==\) 65.25\texttt{pt} or
    \(\approx\) 2.301875\texttt{cm}.
  \end{itemize}
\end{itemize}

\subsection{How to Add New Citation
Styles}\label{how-to-add-new-citation-styles}

\subsection{Must See References}\label{must-see-references}

To learn the basics about LaTeX, see \textcite{oetiker2023}. To delve
deeper into the LaTeX system, see \textcite{lamport1994} and
\textcite{knuth1986}.

\subsubsection{Manuals}\label{manuals}

\begin{figure}

\begin{minipage}{0.50\linewidth}

\begin{itemize}
\tightlist
\item
  \href{https://quarto.org/docs/guide/}{Quarto}
\item
  \href{https://www.ctan.org/pkg/abntex2}{\texttt{abntex2}}
\item
  \href{https://www.ctan.org/pkg/memoir}{\texttt{memoir}}
\item
  \href{https://www.ctan.org/pkg/biblatex}{\texttt{biblatex}}
\item
  \href{https://www.ctan.org/pkg/biblatex-abnt}{\texttt{biblatex-abnt}}
\end{itemize}

\end{minipage}%
%
\begin{minipage}{0.50\linewidth}

\begin{itemize}
\tightlist
\item
  \href{https://www.ctan.org/pkg/biblatex-apa}{\texttt{biblatex-apa}}
\item
  \href{https://ctan.org/pkg/babel}{\texttt{babel}}
\item
  \href{https://ctan.org/pkg/fontspec}{\texttt{fontspec}}
\item
  \href{https://www.ctan.org/pkg/makeidx}{\texttt{makeidx}}
\end{itemize}

\end{minipage}%

\end{figure}%

\subsubsection{R packages}\label{r-packages}

\index{R packages}

\begin{itemize}
\tightlist
\item
  \href{https://gt.rstudio.com}{\texttt{gt}}
\end{itemize}



\tocskipone
\tocprintchapternonum

\begingroup
\ABNTEXfontereduzida
\renewcommand{\baselinestretch}{1}
\setlength{\parindent}{0pt}
\setlength{\parskip}{\tinyskipamount}
\setlength{\afterchapskip}{\hugeskipamount}
\phantompart
\printindex
\endgroup
\end{document}
