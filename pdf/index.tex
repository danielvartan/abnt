% Author: Daniel Vartanian.
% License: GNU General Public License v3 or later
%
% Based on template.tex, developed by the Quarto team and
% abtex2-modelo-trabalho-academico.tex, v-1.9.7, developed by
% Lauro César Araujo and the team behind `abnt2tex`, with additional guidance
% from the theses and dissertations regulations of the University of São Paulo
% (USP). For more information, please visit <http://www.abntex.net.br/>.

% For help, see:
%
% - <https://quarto.org/docs/reference/formats/pdf.html>
% - <https://github.com/abntex/abntex2/wiki/ComoCustomizar>
% - <https://www.ctan.org/pkg/abntex2>
% - <https://www.ctan.org/pkg/memoir>
% - <https://www.ctan.org/pkg/hyperref>

% -----
% Preamble
% -----

% Don´t move the `\PassOptionsToPackage` macros.
% See why: <https://tex.stackexchange.com/a/433605/234832>.
\PassOptionsToPackage{
unicode,bookmarksnumbered
}{hyperref}

\PassOptionsToPackage{hyphens}{url}

\PassOptionsToPackage{dvipsnames,svgnames,x11names}{xcolor}


\documentclass[
12pt,
openright,
oneside,
a4paper,
chapter=TITLE,
section=TITLE,
french,
spanish,
brazil,
english
]{abntex2}
% \usepackage{showframe}

% The order in which the packages are loaded is important!

\usepackage{array}
\usepackage{calc}
\usepackage{caption}
\usepackage{color}
\usepackage{colortbl}
\usepackage{amsmath}
\usepackage{amssymb}
\usepackage{booktabs}
\usepackage{enumitem}
\usepackage{etoolbox}
\usepackage{epigraph}
\usepackage{float}
\usepackage[T1]{fontenc}
\usepackage[hang,multiple]{footmisc}
\usepackage{graphicx}
\usepackage{iftex}
\usepackage{indentfirst}
\usepackage[hyphenation,lastparline,nosingleletter]{impnattypo}
\usepackage[utf8]{inputenc}
\usepackage{lastpage}
\usepackage{lipsum}
\usepackage{longtable}
\usepackage{luacode}
\usepackage{luatexbase}
\usepackage{microtype}
\usepackage{multirow}
\usepackage{parskip}
\usepackage{pdflscape}
\usepackage{pdfpages}
\usepackage{titlesec}
\usepackage[table]{xcolor}
\usepackage{xparse}
\usepackage{xstring}
\usepackage{hyperref}
\usepackage{url}

\ifPDFTeX
  \usepackage{textcomp} % provide euro and other symbols
\else % if luatex or xetex
\usepackage{unicode-math}
\fi
% Set Lengths -----

\newlength{\microskipamount}
\newlength{\tinyskipamount}
\newlength{\hugeskipamount}

\setlength{\microskipamount}{0.25\baselineskip}
\setlength{\tinyskipamount}{0.5\baselineskip}
\setlength{\smallskipamount}{0.75\baselineskip}
\setlength{\medskipamount}{1\baselineskip}
\setlength{\bigskipamount}{1.5\baselineskip}
\setlength{\hugeskipamount}{2\baselineskip}

\newcommand{\microskip}{\vspace{\microskipamount}}
\newcommand{\tinyskip}{\vspace{\tinyskipamount}}
\newcommand{\hugeskip}{\vspace{\hugeskipamount}}

% Set Skips -----

\setlength{\beforechapskip}{\bigskipamount}
\setlength{\afterchapskip}{\smallskipamount}

\titlespacing*{\chapter}{0pt}{\beforechapskip}{\afterchapskip}
\titlespacing*{\section}{0pt}{\medskipamount}{\smallskipamount}
\titlespacing*{\subsection}{0pt}{\medskipamount}{\smallskipamount}
\titlespacing*{\subsubsection}{0pt}{\medskipamount}{\smallskipamount}
\titlespacing*{\paragraph}{0pt}{\medskipamount}{\smallskipamount}

% Set Epigraph -----

\setlength\epigraphwidth{0.6\textwidth}
\setlength\epigraphrule{0pt}
% Set Page -----

\setlength{\headsep}{1cm}
\setlength{\footskip}{1cm}
\checkandfixthelayout[fixed]

% Set Text Spacing -----

\renewcommand{\familydefault}{\rmdefault}

\renewcommand{\baselinestretch}{1.5}

\setlength{\parindent}{1cm}

\setlength{\parskip}{0ex}

% Set Text Font -----


\ifPDFTeX\else
    % xetex/luatex font selection
  \setmainfont[]{Noto-Sans}
  \setsansfont[]{Noto-Sans}
  \setmonofont[Scale=0.75]{Noto-Sans-Mono}




\fi


% Set Footnote -----

\setlength{\footnotemargin}{0.5em} % Equal to `\footmarkwidth`
\let\svfootnoterule\footnoterule % Equal to `\footmarksep`
\renewcommand\footnoterule{\vspace{1ex}\svfootnoterule\vspace{1ex}}
% Change `abntex2` Default Names -----

\newcommand{\capaname}{Capa}
\newcommand{\fichacatalograficaname}{Ficha catalográfica}
\newcommand{\resumoestrangeironame}{Resumo}
\newcommand{\glossarioname}{Glossário}

% Force Bibliography Title Based on `lang' or User Input -----

\ifstrequal{en}{en}{
  \newcommand{\newbibname}{References}
}{
  \ifstrequal{en}{pt-BR}{
    \newcommand{\newbibname}{Referências}
  }{
    \ifstrequal{en}{es}{
      \newcommand{\newbibname}{Referencias}
    }{
      \ifstrequal{en}{fr}{
        \newcommand{\newbibname}{Références}
      }{
        \newcommand{\newbibname}{\bibname}
      }
    }
  }
}

% Add Parameters for Language-Specific Names to `babel` -----

\addto\captionsenglish{
  \renewcommand{\capaname}{Cover}
  \renewcommand{\folhaderostoname}{Title Page}
  \renewcommand{\fichacatalograficaname}{Cataloging Record}
  \renewcommand{\errataname}{Errata}
  \renewcommand{\folhadeaprovacaoname}{Approval Sheet}
  \renewcommand{\dedicatorianame}{Inscription}
  \renewcommand{\agradecimentosname}{Acknowledgements}
  \renewcommand{\epigraphname}{Epigraph}
  \renewcommand{\resumoname}{Abstract}
  \renewcommand{\resumoestrangeironame}{Resumo}
  \renewcommand{\listfigurename}{List of Figures}
  \renewcommand{\listtablename}{List of Tables}
  \renewcommand{\listadesiglasname}{List of Abbreviations and Acronyms}
  \renewcommand{\listadesimbolosname}{List of Symbols}
  \renewcommand{\contentsname}{Contents}
  \renewcommand{\bibname}{References}
  \renewcommand{\glossarioname}{Glossary}
  \renewcommand{\apendicename}{APPENDIX}
  \renewcommand{\apendicesname}{Appendices}
  \renewcommand{\anexoname}{ANNEX}
  \renewcommand{\anexosname}{Annexes}
  \renewcommand{\indexname}{Index}
  \renewcommand{\orientadorname}{Supervisor:}
  \renewcommand{\coorientadorname}{Co-supervisor:}
  \renewcommand{\fontename}{Source}
  \renewcommand{\notaname}{Note}
  \renewcommand{\pageautorefname}{page}
  \renewcommand{\sectionautorefname}{section}
  \renewcommand{\subsectionautorefname}{subsection}
  \renewcommand{\subsubsectionautorefname}{subsubsection}
  \renewcommand{\paragraphautorefname}{subsubsubsection}
}

\addto\captionsbrazil{
  \renewcommand{\capaname}{Capa}
  \renewcommand{\folhaderostoname}{Folha de Rosto}
  \renewcommand{\fichacatalograficaname}{Ficha Catalográfica}
  \renewcommand{\errataname}{Errata}
  \renewcommand{\folhadeaprovacaoname}{Folha de Aprovação}
  \renewcommand{\dedicatorianame}{Dedicatória}
  \renewcommand{\agradecimentosname}{Agradecimentos}
  \renewcommand{\epigraphname}{Epígrafe}
  \renewcommand{\resumoname}{Resumo}
  \renewcommand{\resumoestrangeironame}{Abstract}
  \renewcommand{\listfigurename}{Lista de Figuras}
  \renewcommand{\listtablename}{Lista de Tabelas}
  \renewcommand{\listadesiglasname}{Lista de Abreviaturas e Siglas}
  \renewcommand{\listadesimbolosname}{Lista de Símbolos}
  \renewcommand{\contentsname}{Sumário}
  \renewcommand{\bibname}{Referências}
  \renewcommand{\glossarioname}{Glossário}
  \renewcommand{\apendicename}{APÊNDICE}
  \renewcommand{\apendicesname}{Apêndices}
  \renewcommand{\anexoname}{ANEXO}
  \renewcommand{\anexosname}{Anexos}
  \renewcommand{\indexname}{Índice}
  \renewcommand{\orientadorname}{Orientador:}
  \renewcommand{\coorientadorname}{Coorientador:}
  \renewcommand{\fontename}{Fonte}
  \renewcommand{\notaname}{Nota}
  \renewcommand{\pageautorefname}{página}
  \renewcommand{\sectionautorefname}{seção}
  \renewcommand{\subsectionautorefname}{subseção}
  \renewcommand{\subsubsectionautorefname}{subsubseção}
  \renewcommand{\paragraphautorefname}{subsubsubseção}
}

\addto\captionsspanish{
  \renewcommand{\capaname}{Portada}
  \renewcommand{\folhaderostoname}{Página de título}
  \renewcommand{\fichacatalograficaname}{Ficha Catalográfica}
  \renewcommand{\errataname}{Errata}
  \renewcommand{\folhadeaprovacaoname}{Hoja de aprobación}
  \renewcommand{\dedicatorianame}{Dedicatoria}
  \renewcommand{\agradecimentosname}{Agradecimientos}
  \renewcommand{\epigraphname}{Epígrafe}
  \renewcommand{\resumoname}{Resumen}
  \renewcommand{\resumoestrangeironame}{Resumo}
  \renewcommand{\listfigurename}{Lista de Figuras}
  \renewcommand{\listtablename}{Lista de Tablas}
  \renewcommand{\listadesiglasname}{Lista de Abreviaturas y Siglas}
  \renewcommand{\listadesimbolosname}{Lista de Símbolos}
  \renewcommand{\contentsname}{Sumario}
  \renewcommand{\bibname}{Referencias}
  \renewcommand{\glossarioname}{Glosario}
  \renewcommand{\apendicename}{APÉNDICE}
  \renewcommand{\apendicesname}{Apéndices}
  \renewcommand{\anexoname}{ANEXO}
  \renewcommand{\anexosname}{Anexos}
  \renewcommand{\indexname}{Índice}
  \renewcommand{\orientadorname}{Asesor:}
  \renewcommand{\coorientadorname}{Coasesor:}
  \renewcommand{\fontename}{Fuente}
  \renewcommand{\notaname}{Nota}
  \renewcommand{\pageautorefname}{página}
  \renewcommand{\sectionautorefname}{sección}
  \renewcommand{\subsectionautorefname}{subsección}
  \renewcommand{\subsubsectionautorefname}{subsubsección}
  \renewcommand{\paragraphautorefname}{subsubsubsección}
}

\addto\captionsfrench{
  \renewcommand{\capaname}{Couverture}
  \renewcommand{\folhaderostoname}{Page de Titre}
  \renewcommand{\fichacatalograficaname}{Fiche Cataloguée}
  \renewcommand{\errataname}{Errata}
  \renewcommand{\folhadeaprovacaoname}{Feuille d'Approbation}
  \renewcommand{\dedicatorianame}{Dédiace
  \renewcommand{\agradecimentosname}{Remerciements}}
  \renewcommand{\epigraphname}{Épigraphe}
  \renewcommand{\resumoname}{Résumé}
  \renewcommand{\resumoestrangeironame}{Resumo}
  \renewcommand{\listfigurename}{Liste des Figures}
  \renewcommand{\listtablename}{Liste des Tableaux}
  \renewcommand{\listadesiglasname}{Liste des Abréviations et Sigles}
  \renewcommand{\listadesimbolosname}{Liste des Symboles}
  \renewcommand{\contentsname}{Sommaire}
  \renewcommand{\bibname}{Références}
  \renewcommand{\glossarioname}{Glossaire}
  \renewcommand{\apendicename}{APPENDICE}
  \renewcommand{\apendicesname}{Appendices}
  \renewcommand{\anexoname}{ANNEXE}
  \renewcommand{\anexosname}{Annexes}
  \renewcommand{\indexname}{Index}
  \renewcommand{\orientadorname}{Conseiller:}
  \renewcommand{\coorientadorname}{Co-conseiller:}
  \renewcommand{\fontename}{Source}
  \renewcommand{\notaname}{Note}
  \renewcommand{\pageautorefname}{page}
  \renewcommand{\sectionautorefname}{section}
  \renewcommand{\subsectionautorefname}{sous-section}
  \renewcommand{\subsubsectionautorefname}{sous-sous-section}
  \renewcommand{\paragraphautorefname}{sous-sous-sous-section}
}
% See `babel.tex` for language changes.
% See `toc.text` for changes related to the ToC.

% Set Page Numbering -----

\makepagestyle{abntheadings}
\makeevenhead{abntheadings}{\ABNTEXfontereduzida\thepage}{}{}
\makeoddhead{abntheadings}{}{}{\ABNTEXfontereduzida\thepage}

% Set Text Variables -----

\renewcommand{\ABNTEXpartfont}{\sffamily\bfseries}
\renewcommand{\ABNTEXpartfontsize}{\normalsize}
\renewcommand{\ABNTEXchapterfont}{\sffamily\bfseries}
\renewcommand{\ABNTEXchapterfontsize}{\normalsize}
\renewcommand{\ABNTEXsectionfont}{\sffamily}
\renewcommand{\ABNTEXsectionfontsize}{\normalsize}
\renewcommand{\ABNTEXsubsectionfont}{\sffamily}
\renewcommand{\ABNTEXsubsectionfontsize}{\normalsize}
\renewcommand{\ABNTEXsubsubsectionfont}{\sffamily}
\renewcommand{\ABNTEXsubsubsectionfontsize}{\normalsize}
\renewcommand{\ABNTEXsubsubsubsectionfont}{\sffamily}
\renewcommand{\ABNTEXsubsubsubsectionfontsize}{\normalsize\itshape}
\renewcommand{\ABNTEXfontereduzida}{\footnotesize}
\renewcommand{\ABNTEXcaptiondelim}{~\textendash~}
\renewcommand{\ABNTEXcaptionfontedelim}{:~}

\renewcommand{\captiontitlefont}{\ABNTEXfontereduzida}

% Set New Commands -----

\providecommand{\imprimiruniversidade}{}
\newcommand{\universidade}[1]{\renewcommand{\imprimiruniversidade}{#1}}

\providecommand{\imprimirescola}{}
\newcommand{\escola}[1]{\renewcommand{\imprimirescola}{#1}}

\providecommand{\imprimirprograma}{}
\newcommand{\programa}[1]{\renewcommand{\imprimirprograma}{#1}}

\newcommand{\imprimirtipodetrabalho}{\imprimirtipotrabalho}

\providecommand{\imprimirtipodetituloacademico}{}
\newcommand{\tipodetituloacademico}[1]{\renewcommand{\imprimirtipodetituloacademico}{#1}}

\providecommand{\imprimirtituloacademico}{}
\newcommand{\tituloacademico}[1]{\renewcommand{\imprimirtituloacademico}{#1}}

\providecommand{\imprimirareadeconcentracao}{}
\newcommand{\areadeconcentracao}[1]{\renewcommand{\imprimirareadeconcentracao}{#1}}

\providecommand{\imprimirnotadeversao}{}
\newcommand{\notadeversao}[1]{\renewcommand{\imprimirnotadeversao}{#1}}

% Set Chapter Style -----

\renewcommand{\chapnamefont}{\ABNTEXchapterfont\ABNTEXchapterfontsize\mdseries}
\renewcommand{\chapnumfont}{\ABNTEXchapterfont\ABNTEXchapterfontsize\mdseries}

\setsecnumformat{\chapnumfont\csname the#1\endcsname\quad}

\renewcommand{\printchaptername}{
  \ifthenelse{\boolean{abntex@apendiceousecao}}{
    \chapnamefont \ABNTEXchapterupperifneeded{\appendixname} % [Changed]
  }{}
}

\renewcommand{\chapternamenum}{
  \ifthenelse{\boolean{abntex@apendiceousecao}}{
    \hspace{-2em} \space
  }{}
}

\renewcommand{\printchapternum}{
  \tocprintchapter
  \setboolean{abntex@innonumchapter}{false}
  \chapnumfont
  \thechapter % [Changed]
  % \ifthenelse{\boolean{abntex@apendiceousecao}}{ % [Removed]
  %   \tocinnonumchapter
  %   \ABNTEXcaptiondelim
  % }{}
}

\renewcommand{\afterchapternum}{
  \ifthenelse{\boolean{abntex@apendiceousecao}}{ % [Added]
    \ABNTEXchapterfont\mdseries \hspace{-1em} \space\ABNTEXcaptiondelim\space \hspace{-1.5em}
  }{
    \hspace{-0.875em}
  }
}

\renewcommand{\printchapternonum}{
  \tocprintchapternonum
  \setlength{\afterchapskip}{\hugeskipamount} % [Added]
  \setboolean{abntex@innonumchapter}{true}
}

\renewcommand{\printchaptertitle}[1]{
  \chaptitlefont
  \ifthenelse{\boolean{abntex@innonumchapter}}{
    \centering \ABNTEXchapterupperifneeded{#1}
  }{
    \ifthenelse{\boolean{abntex@apendiceousecao}}{
      \ABNTEXchapterfont\mdseries\ABNTEXchapterupperifneeded{#1}
    }{
      \ABNTEXchapterupperifneeded{#1}
    }
  }
}

% Set `citation' environment -----

\renewenvironment{citacao}[1][default]{
   \list{}
   \ABNTEXfontereduzida
   \addtolength{\leftskip}{\ABNTEXcitacaorecuo}
   \item[]
   \begin{SingleSpace}
   \ifthenelse{\not\equal{#1}{default}}{\selectlanguage{#1}}{} % [Changed]
 }{
   \end{SingleSpace}
   \endlist
  }

% Set `\textual` -----

\renewcommand{\textual}{
  \pagestyle{abntheadings}
  \aliaspagestyle{chapter}{abntheadings}
}

% Set Cover -----

\renewcommand{\imprimircapa}{
  \phantomsection\pdfbookmark[0]{\capaname}{}
  \begin{capa}%
  \begin{adjustwidth}{-1cm}{0cm}
  \center
  \imprimirinstituicao

  \vfill
  \imprimirautor

  \vfill
  {\ABNTEXchapterfont\imprimirtitulo}

  \vfill
  \vspace{6.5cm}
  \imprimirlocal

  \imprimirdata
  \vspace{1.5cm}
  \end{adjustwidth}
  \end{capa}
}

% Set Title Page -----

\makeatletter
\renewcommand{\folhaderostocontent}{
  \begin{center}
  \imprimirautor

  \vfill
  {\ABNTEXchapterfont\imprimirtitulo}

  \vfill
  \textbf{\imprimirnotadeversao}

  \vfill
  \abntex@ifnotempty{
    \imprimirpreambulo
  }{
    \hspace{0.35\textwidth}
    \begin{minipage}{.6\textwidth}
    \SingleSpacing
    \imprimirpreambulo
    \end{minipage}
  }

  \vfill
  \imprimirlocal

  \imprimirdata
  \vspace{1cm}
  \end{center}
}
\makeatother

% Set Cataloging Record -----

\renewenvironment{fichacatalografica}{
  \PRIVATEbookmarkthis{\fichacatalograficaname}
  \setlength{\parindent}{0cm}
  \begin{SingleSpacing}
}{
  \end{SingleSpacing}
}

% Set Errata -----

\renewenvironment{errata}[1][\errataname]{
  \newpage
  \phantomsection
  \pretextualchapter{#1}
}{
  \cleardoublepage
}

% Set Approval Sheet -----

\renewenvironment{folhadeaprovacao}[1][\folhadeaprovacaoname]{
  \clearpage
  \PRIVATEbookmarkthis{#1}
  \setlength\parindent{0cm}
  \AtBeginEnvironment{tabular}{\normalsize}
  \begin{SingleSpace}
}{
  \end{SingleSpace}
  \cleardoublepage
}

% Set Abstract -----

\newenvironment{resumoenv}[1][\resumoname]{
  \pretextualchapter{#1}
  \begingroup
  \setlength{\parindent}{0cm}
  \setlength{\parskip}{\smallskipamount} % The troublemaker.
  \AtBeginEnvironment{tabular}{\normalsize}
  \renewcommand{\arraystretch}{1}
  \setlength{\aboverulesep}{0ex}
  \setlength{\belowrulesep}{0ex}
  \setlength{\arrayrulewidth}{0pt}
  \setlength{\tabcolsep}{0cm}
  \vspace{-\smallskipamount} % !
  \begin{SingleSpace}
}{
  \end{SingleSpace}
  \cleardoublepage
  \endgroup
}

% Set List of Abbreviations and Acronyms -----

\renewenvironment{siglas}{
  \pretextualchapter{\listadesiglasname}
}{
  \cleardoublepage
}

% Set List of Symbols -----

\renewenvironment{simbolos}{
  \pretextualchapter{\listadesimbolosname}
}{
  \cleardoublepage
}

% Set Glossary -----

\newenvironment{glossario}{
  \tocprintchapternonum
}{
  \cleardoublepage
}

% Set Appendices and Annexes -----

\renewcommand{\PRIVATEapendiceconfig}[2]{
  \setboolean{abntex@apendiceousecao}{true}
  \renewcommand{\appendixname}{#1}

  \ifthenelse{\boolean{ABNTEXsumario-abnt-6027-2012}}{
    \renewcommand{\appendixtocname}{\uppercase{#2}}
  }{
    \renewcommand{\appendixtocname}{#2}
  }

  \renewcommand{\appendixpagename}{#2}
  \renewcommand{\appendixtocname}{#2}
  \renewcommand{\cftappendixname}{} % [Altered]

  % Note:
  %
  % \cleardoublepage
  % \phantomsection
  % \addcontentsline{toc}{part}{Appendices}
  % \appendix
  %
  % is automatically add by the Quarto render.
}

\newcommand{\PRIVATEapendiceconfigafter}[1]{
    \chapterstyle{apendice}

    % Quarto adds \appendix automatically,
    % so we just need to add the chapter title.
    \chapter*{#1}
    \markboth{#1}{#1}

    \let\clearpage\relax
}

\renewcommand{\apendices}{
  \clearpage
  \PRIVATEapendiceconfig{\apendicename}{\apendicesname}
  \appendix
  \PRIVATEapendiceconfigafter{\apendicesname}
}

\renewenvironment{apendicesenv}{
  \clearpage
  \PRIVATEapendiceconfig{\apendicename}{\apendicesname}
  \begin{appendix}
  \PRIVATEapendiceconfigafter{\apendicesname}
}{
  \end{appendix}
  \setboolean{abntex@apendiceousecao}{false}
  \bookmarksetup{startatroot}
}

\renewcommand{\anexos}{
  \clearpage
  \PRIVATEapendiceconfig{\anexoname}{\anexosname}

  \newpage
  \appendix
  \renewcommand\theHchapter{anexochapback.\arabic{chapter}}
  \PRIVATEapendiceconfigafter{\anexosname}
}

\renewenvironment{anexosenv}{
  \clearpage
  \PRIVATEapendiceconfig{\anexoname}{\anexosname}

  \newpage
  \phantomsection
  \addcontentsline{toc}{part}{\appendixtocname}

  \begin{appendix}
  \renewcommand\theHchapter{anexochapback.\arabic{chapter}}
  \PRIVATEapendiceconfigafter{\anexosname}
}{
  \end{appendix}
  \setboolean{abntex@apendiceousecao}{false}
  \bookmarksetup{startatroot}
}
% Set New Colors -----

\definecolor{blue}{HTML}{2905C3}

% See <https://getbootstrap.com/docs/5.0/utilities/colors/>.
\definecolor{quarto-blue}{HTML}{2780E3}
\definecolor{quarto-lighter-blue}{HTML}{ECF4FC}
\definecolor{quarto-orange}{HTML}{FF7518}
\definecolor{quarto-ligther-orange}{HTML}{FFF3EB}
\definecolor{quarto-red}{HTML}{D9534F}
\definecolor{quarto-ligther-red}{HTML}{FCF1F1}
\definecolor{quarto-green}{HTML}{3FB618}
\definecolor{quarto-ligther-green}{HTML}{EFF9EB}
\definecolor{quarto-purple}{HTML}{7D12BA}
\definecolor{quarto-gray}{HTML}{A3A3A3}
\definecolor{quarto-medium-gray}{HTML}{CFD0D1}
\definecolor{quarto-ligther-gray}{HTML}{F1F3F5}

\definecolor{bs-link-color}{HTML}{39729E}

% Set Body Color -----


% Quarto's Default Settings -----

% \usepackage{graphicx} % Already loaded in `packages.tex`.
\makeatletter
\newsavebox\pandoc@box
\newcommand*\pandocbounded[1]{% scales image to fit in text height/width
  \sbox\pandoc@box{#1}%
  \Gscale@div\@tempa{\textheight}{\dimexpr\ht\pandoc@box+\dp\pandoc@box\relax}%
  \Gscale@div\@tempb{\linewidth}{\wd\pandoc@box}%
  \ifdim\@tempb\p@<\@tempa\p@\let\@tempa\@tempb\fi% select the smaller of both
  \ifdim\@tempa\p@<\p@\scalebox{\@tempa}{\usebox\pandoc@box}%
  \else\usebox{\pandoc@box}%
  \fi%
}
% Set default figure placement to `htbp`
\def\fps@figure{htbp}
\makeatother

% Set Distance from Top of the Page to First Float -----

\makeatletter
\setlength{\@fptop}{5pt}
\makeatother

% Set Captions and Legends -----

\DeclareCaptionFont{ABNTEXfontereduzida}{\ABNTEXfontereduzida}

% For customization, see `\DeclareCaptionFormat` in the `caption` package.
\captionsetup{
  font=ABNTEXfontereduzida
  ,justification=justified
}

\renewcommand{\abovecaptionskip}{\smallskipamount}
\renewcommand{\belowcaptionskip}{\smallskipamount}

\renewcommand{\legend}[1]{
  \hyphenpenalty=100000
  \ABNTEXfontereduzida
  \addvspace{\smallskipamount}
  Source: #1
}

\newcommand{\legendleft}[1]{
  \addvspace{\smallskipamount}
  \begin{tabular}{@{}l@{ } p{14cm}}
    \hyphenpenalty=100000
    \ABNTEXfontereduzida Source: &
    \ABNTEXfontereduzida #1
  \end{tabular}
}

% Credits: <https://tex.stackexchange.com/a/611556/234832>.
\AddToHook{cmd/caption/before}{\hyphenpenalty=100000}

% Set Figure Environment -----

\AtBeginEnvironment{figure}{
  \ABNTEXfontereduzida
  \addvspace{\tinyskipamount}
}

\AtEndEnvironment{figure}{
  \addvspace{\smallskipamount}
}
\renewcommand{\arraystretch}{1.5}
\setlength{\aboverulesep}{0ex}
\setlength{\belowrulesep}{0ex}

% Allow Footnotes in `longtable` Head/Footer ----

\IfFileExists{footnotehyper.sty}{\usepackage{footnotehyper}}{\usepackage{footnote}}
\makesavenoteenv{longtable}

% Set Tabular Environment -----

\AtBeginEnvironment{table}{\ABNTEXfontereduzida}
\AtBeginEnvironment{tabular}{\ABNTEXfontereduzida}

\AtBeginEnvironment{longtable}{\ABNTEXfontereduzida \addvspace{\tinyskipamount}}
\AtBeginEnvironment{longtable*}{\ABNTEXfontereduzida \addvspace{\tinyskipamount}}

\floatplacement{table}{H}
\providecommand{\tightlist}{
\setlength{\itemsep}{0ex}\setlength{\parskip}{0\baselineskip}}

% \setlist[enumerate]{leftmargin=1cm)}
% \setlist[itemize]{leftmargin=2cm}
\makeatletter
\newcommand*{\getlength}[1]{\strip@pt#1}
\makeatother
\title{
abnt: Quarto Format for ABNT Theses and Dissertations

}

\titulo{
abnt: Quarto Format for ABNT Theses and Dissertations
}


\author{Daniel Vartanian}
\autor{Daniel Vartanian}

\local{{[}City{]}}

\date{2026}
\data{2026}

\orientador{{[}Supervisor's full name{]}}

\coorientador{{[}Co-supervisor's full name{]}}

\tipodetituloacademico{{[}Master/PhD{]}}

\tituloacademico{{[}Master of Science/Doctor of Science{]}}

\tipotrabalho{{[}Dissertation/Thesis{]}}

\areadeconcentracao{{[}Area of concentration{]}}

\instituicao{\MakeUppercase{{[}University{]}}}
\universidade{{[}University{]}}

\instituicao{
  \MakeUppercase{{[}University{]}}
  \par
  \MakeUppercase{{[}School/Department{]}}
}

\escola{{[}School/Department{]}}

\instituicao{
  \MakeUppercase{{[}University{]}}
  \par
  \MakeUppercase{{[}School/Department{]}}
  \par
  \MakeUppercase{{[}Graduate program{]}}
}

\programa{{[}Graduate program{]}}

\notadeversao{{[}Original/Revised version{]}}

\hypersetup{
pdftitle={abnt: Quarto Format for ABNT Theses and Dissertations},
pdfauthor={Daniel Vartanian},
pdflang={en},
pdfsubject={{[}Dissertation/Thesis{]}},
linktoc={section},
colorlinks=true,
linkcolor={blue},
filecolor={blue},
citecolor={blue},
urlcolor={blue},
pdfcreator={LaTeX via pandoc},
bookmarksdepth=5
}
% Set Sections Skips (`\cftchapterpresnum`) -----

\setlength{\cftbeforebookskip}{0\baselineskip}
\setlength{\cftbeforepartskip}{\medskipamount}
\setlength{\cftbeforechapterskip}{\microskipamount}
\setlength{\cftbeforesectionskip}{0\baselineskip}
\setlength{\cftbeforesubsectionskip}{0\baselineskip}
\setlength{\cftbeforesubsubsectionskip}{0\baselineskip}
\setlength{\cftbeforeparagraphskip}{0\baselineskip}

% Set Section Numbers Fonts (`\cftchapterpresnum`) -----

\renewcommand{\cftchapterpresnum}{\normalfont}
\renewcommand{\cftsectionpresnum}{\normalfont}
\renewcommand{\cftsubsectionpresnum}{\normalfont}
\renewcommand{\cftsubsubsectionpresnum}{\normalfont}
\renewcommand{\cftparagraphpresnum}{\normalfont}

% Set Section Names Fonts (`\cftpartfont`) -----

\renewcommand{\cftpartfont}[1]{
  \ABNTEXchapterupperifneeded{\normalfont\bfseries #1}
}

\renewcommand{\cftchapterfont}[1]{
  \ABNTEXchapterupperifneeded{\normalfont\bfseries #1}
}

\renewcommand{\cftsectionfont}[1]{
  \ABNTEXsectionupperifneeded{\normalfont #1}
}

\renewcommand{\cftsubsectionfont}[1]{
  \ABNTEXsubsectionupperifneeded{\normalfont\bfseries #1}
}

\renewcommand{\cftsubsubsectionfont}[1]{
  \ABNTEXsubsubsectionupperifneeded{\normalfont #1}
}

\renewcommand{\cftparagraphfont}[1]{
  \ABNTEXsubsubsubsectionupperifneeded{\normalfont\itshape #1}
}

% Set Section Page Numbers Fonts (`\cftpartpagefont`) -----

\renewcommand{\cftpartpagefont}{\normalfont}
\renewcommand{\cftchapterpagefont}{\normalfont}
\renewcommand{\cftsectionpagefont}{\normalfont}
\renewcommand{\cftsubsectionpagefont}{\normalfont}
\renewcommand{\cftsubsubsectionpagefont}{\normalfont}
\renewcommand{\cftparagraphpagefont}{\normalfont}
\renewcommand{\cftfigurepagefont}{\normalfont}
\renewcommand{\cfttablepagefont}{\normalfont}

% Renew `abntex2` ToC Commands -----

\cftinsertcode{A}{} % [Changed]

% This is not right. Create an issue about it.
\renewcommand{\tocprintchapternonum}{
  \addtocontents{toc}{\setlength{\cftchapterindent}{5.65em}}
  \addtocontents{toc}{\setlength{\cftchapternumwidth}{0em}}
}

\renewcommand{\tocpartapendices}{
  \addtocontents{toc}{\setlength{\cftpartindent}{5.65em}}
  \addtocontents{toc}{\setlength{\cftpartnumwidth}{0em}}
}

% Set ToC Skip -----

\newcommand{\tocskipone}{
  \addtocontents{toc}{\protect\vspace{\smallskipamount}}
}

% \setlength{\cftbeforepartskip}{\bigskipamount}
\newcommand{\tocskiptwo}{
  % \addtocontents{toc}{\protect\vspace{\tinyskipamount}}
}
% Set `biblatex` Options -----

\usepackage[
,backend=biber,url=true,useprefix=false,giveninits=true,style=abnt
]{biblatex}

\usepackage{csquotes}

% Set Bibliography Resource -----

\addbibresource{references.bib}

% Set Reference Formatting -----

\AtBeginBibliography{\vspace{0.5\baselineskip}}
\AtEveryBibitem{\clearfield{annotation}}
\renewcommand{\bibfont}{\ABNTEXfontereduzida}

\setlength{\bibhang}{0cm}

\setlength{\bibparsep}{-2ex}

% Set Footnote -----

\newcommand{\bibnamewithfootnote}{
  \MakeUppercase{\newbibname}\protect\footnote{According to the
Brazilian Association of Technical Standards (ABNT NBR 6023).}
}


\defbibheading{bibheading}[\bibnamewithfootnote]{
  \ifthenelse{\boolean{ABNTEXupperchapter}}{
    \setboolean{ABNTEXupperchapter}{false}
    \chapter*{#1}
    \markboth{#1}{#1}
    \setboolean{ABNTEXupperchapter}{true}
  }{
    \chapter*{#1}
    \markboth{#1}{#1}
  }
}
\usepackage{makeidx}
\makeindex

% Set Sections Skips (`\cftchapterpresnum`) -----

% Credits: https://tex.stackexchange.com/a/28361/234832
% \begin{luacode}
% local PENALTY=node.id("penalty")
% last_line_twice_parindent = function (head)
%   while head do
%     local _w,_h,_d = node.dimensions(head)
%     if head.id == PENALTY and head.subtype ~= 15 and (_w < 2 * tex.parindent) then
%
%         -- we are at a glue and have less than 2*\parindent to go
%         local p = node.new("penalty")
%         p.penalty = 10000
%         p.next = head
%         head.prev.next = p
%         p.prev = head.prev
%         head.prev = p
%     end
%
%     head = head.next
%   end
%   return true
% end
%
% luatexbase.add_to_callback("pre_linebreak_filter",last_line_twice_parindent,"Raphink")
% \end{luacode}
% Use upquote if available, for straight quotes in verbatim environments
\IfFileExists{upquote.sty}{\usepackage{upquote}}{}
\IfFileExists{microtype.sty}{% use microtype if available
  \usepackage[]{microtype}
  \UseMicrotypeSet[protrusion]{basicmath} % disable protrusion for tt fonts
}{}





\setlength{\emergencystretch}{3em} % Prevent overfull lines

\setcounter{secnumdepth}{5}

% Make \paragraph and \subparagraph free-standing
\ifx\paragraph\undefined\else
  \let\oldparagraph\paragraph
  \renewcommand{\paragraph}[1]{\oldparagraph{#1}}
\fi
\ifx\subparagraph\undefined\else
  \let\oldsubparagraph\subparagraph
  \renewcommand{\subparagraph}[1]{\oldsubparagraph{#1}}
\fi

\usepackage{color}
\usepackage{fancyvrb}
\newcommand{\VerbBar}{|}
\newcommand{\VERB}{\Verb[commandchars=\\\{\}]}
\DefineVerbatimEnvironment{Highlighting}{Verbatim}{commandchars=\\\{\}}
% Add ',fontsize=\small' for more characters per line
\usepackage{framed}
\definecolor{shadecolor}{RGB}{241,243,245}
\newenvironment{Shaded}{\begin{snugshade}}{\end{snugshade}}
\newcommand{\AlertTok}[1]{\textcolor[rgb]{0.68,0.00,0.00}{#1}}
\newcommand{\AnnotationTok}[1]{\textcolor[rgb]{0.37,0.37,0.37}{#1}}
\newcommand{\AttributeTok}[1]{\textcolor[rgb]{0.40,0.45,0.13}{#1}}
\newcommand{\BaseNTok}[1]{\textcolor[rgb]{0.68,0.00,0.00}{#1}}
\newcommand{\BuiltInTok}[1]{\textcolor[rgb]{0.00,0.23,0.31}{#1}}
\newcommand{\CharTok}[1]{\textcolor[rgb]{0.13,0.47,0.30}{#1}}
\newcommand{\CommentTok}[1]{\textcolor[rgb]{0.37,0.37,0.37}{#1}}
\newcommand{\CommentVarTok}[1]{\textcolor[rgb]{0.37,0.37,0.37}{\textit{#1}}}
\newcommand{\ConstantTok}[1]{\textcolor[rgb]{0.56,0.35,0.01}{#1}}
\newcommand{\ControlFlowTok}[1]{\textcolor[rgb]{0.00,0.23,0.31}{\textbf{#1}}}
\newcommand{\DataTypeTok}[1]{\textcolor[rgb]{0.68,0.00,0.00}{#1}}
\newcommand{\DecValTok}[1]{\textcolor[rgb]{0.68,0.00,0.00}{#1}}
\newcommand{\DocumentationTok}[1]{\textcolor[rgb]{0.37,0.37,0.37}{\textit{#1}}}
\newcommand{\ErrorTok}[1]{\textcolor[rgb]{0.68,0.00,0.00}{#1}}
\newcommand{\ExtensionTok}[1]{\textcolor[rgb]{0.00,0.23,0.31}{#1}}
\newcommand{\FloatTok}[1]{\textcolor[rgb]{0.68,0.00,0.00}{#1}}
\newcommand{\FunctionTok}[1]{\textcolor[rgb]{0.28,0.35,0.67}{#1}}
\newcommand{\ImportTok}[1]{\textcolor[rgb]{0.00,0.46,0.62}{#1}}
\newcommand{\InformationTok}[1]{\textcolor[rgb]{0.37,0.37,0.37}{#1}}
\newcommand{\KeywordTok}[1]{\textcolor[rgb]{0.00,0.23,0.31}{\textbf{#1}}}
\newcommand{\NormalTok}[1]{\textcolor[rgb]{0.00,0.23,0.31}{#1}}
\newcommand{\OperatorTok}[1]{\textcolor[rgb]{0.37,0.37,0.37}{#1}}
\newcommand{\OtherTok}[1]{\textcolor[rgb]{0.00,0.23,0.31}{#1}}
\newcommand{\PreprocessorTok}[1]{\textcolor[rgb]{0.68,0.00,0.00}{#1}}
\newcommand{\RegionMarkerTok}[1]{\textcolor[rgb]{0.00,0.23,0.31}{#1}}
\newcommand{\SpecialCharTok}[1]{\textcolor[rgb]{0.37,0.37,0.37}{#1}}
\newcommand{\SpecialStringTok}[1]{\textcolor[rgb]{0.13,0.47,0.30}{#1}}
\newcommand{\StringTok}[1]{\textcolor[rgb]{0.13,0.47,0.30}{#1}}
\newcommand{\VariableTok}[1]{\textcolor[rgb]{0.07,0.07,0.07}{#1}}
\newcommand{\VerbatimStringTok}[1]{\textcolor[rgb]{0.13,0.47,0.30}{#1}}
\newcommand{\WarningTok}[1]{\textcolor[rgb]{0.37,0.37,0.37}{\textit{#1}}}

\newcolumntype{P}[1]{>{\centering\arraybackslash}p{#1}}

\clubpenalty10000
\widowpenalty10000
\displaywidowpenalty10000

\ifLuaTeX
  \usepackage{selnolig}  % disable illegal ligatures
\fi


\IfFileExists{xurl.sty}{\usepackage{xurl}}{} % add URL line breaks if available
\urlstyle{same} % disable monospaced font for URLs


% Set Theorem Environment -----

\AtEndEnvironment{theorem}{\vspace{\tinyskipamount}}

% Set Custom Functions -----

% Credits: <https://tex.stackexchange.com/a/300215/234832>.

\usepackage{xparse}

\ExplSyntaxOn
\NewExpandableDocumentCommand{\repeatntimes}{O{}mm}
 {
  \int_compare:nT { #2 > 0 }
   {
    #3 \prg_replicate:nn { #2 - 1 } { #1#3 }
   }
 }
\ExplSyntaxOff
% -----
% Title Page
% -----

\preambulo{
\hyphenpenalty=100000

%:::% title-page body begin %:::%
{\imprimirtipotrabalho} presented to the {\imprimirescola} at the {\imprimiruniversidade}, as a requirement for the degree of {\imprimirtituloacademico} by the {\imprimirprograma}.

\smallskip

Area of concentration: {\imprimirareadeconcentracao}

\smallskip

Supervisor: Prof. Dr. {\imprimirorientador}

\vspace{\tinyskipamount}

Co-Supervisor: Prof. Dr. {\imprimircoorientador}
%:::% title-page body end %:::%
}
\usepackage{booktabs}
\usepackage{caption}
\usepackage{longtable}
\usepackage{colortbl}
\usepackage{array}
\usepackage{anyfontsize}
\usepackage{multirow}
\makeatletter
\@ifpackageloaded{tcolorbox}{}{\usepackage[skins,breakable]{tcolorbox}}
\@ifpackageloaded{fontawesome5}{}{\usepackage{fontawesome5}}
\definecolor{quarto-callout-color}{HTML}{909090}
\definecolor{quarto-callout-note-color}{HTML}{0758E5}
\definecolor{quarto-callout-important-color}{HTML}{CC1914}
\definecolor{quarto-callout-warning-color}{HTML}{EB9113}
\definecolor{quarto-callout-tip-color}{HTML}{00A047}
\definecolor{quarto-callout-caution-color}{HTML}{FC5300}
\definecolor{quarto-callout-color-frame}{HTML}{acacac}
\definecolor{quarto-callout-note-color-frame}{HTML}{4582ec}
\definecolor{quarto-callout-important-color-frame}{HTML}{d9534f}
\definecolor{quarto-callout-warning-color-frame}{HTML}{f0ad4e}
\definecolor{quarto-callout-tip-color-frame}{HTML}{02b875}
\definecolor{quarto-callout-caution-color-frame}{HTML}{fd7e14}
\makeatother
\makeatletter
\@ifpackageloaded{bookmark}{}{\usepackage{bookmark}}
\makeatother
\makeatletter
\@ifpackageloaded{caption}{}{\usepackage{caption}}
\AtBeginDocument{%
\ifdefined\contentsname
  \renewcommand*\contentsname{Table of Contents}
\else
  \newcommand\contentsname{Table of Contents}
\fi
\ifdefined\listfigurename
  \renewcommand*\listfigurename{List of Figures}
\else
  \newcommand\listfigurename{List of Figures}
\fi
\ifdefined\listtablename
  \renewcommand*\listtablename{List of Tables}
\else
  \newcommand\listtablename{List of Tables}
\fi
\ifdefined\figurename
  \renewcommand*\figurename{Figure}
\else
  \newcommand\figurename{Figure}
\fi
\ifdefined\tablename
  \renewcommand*\tablename{Table}
\else
  \newcommand\tablename{Table}
\fi
}
\@ifpackageloaded{float}{}{\usepackage{float}}
\floatstyle{ruled}
\@ifundefined{c@chapter}{\newfloat{codelisting}{h}{lop}}{\newfloat{codelisting}{h}{lop}[chapter]}
\floatname{codelisting}{Listing}
\newcommand*\listoflistings{\listof{codelisting}{List of Listings}}
\usepackage{amsthm}
\theoremstyle{plain}
\newtheorem{theorem}{Theorem}[chapter]
\theoremstyle{definition}
\newtheorem{definition}{Definition}[chapter]
\theoremstyle{plain}
\newtheorem{conjecture}{Conjecture}[chapter]
\theoremstyle{remark}
\AtBeginDocument{\renewcommand*{\proofname}{Proof}}
\newtheorem*{remark}{Remark}
\newtheorem*{solution}{Solution}
\newtheorem{refremark}{Remark}[chapter]
\newtheorem{refsolution}{Solution}[chapter]
\makeatother
\makeatletter
\makeatother
\makeatletter
\@ifpackageloaded{caption}{}{\usepackage{caption}}
\@ifpackageloaded{subcaption}{}{\usepackage{subcaption}}
\makeatother
\makeatletter
\@ifpackageloaded{tcolorbox}{}{\usepackage[skins,breakable]{tcolorbox}}
\makeatother
\makeatletter
\@ifundefined{shadecolor}{\definecolor{shadecolor}{HTML}{CFD0D1}}{}
\makeatother
\makeatletter
\@ifundefined{codebgcolor}{\definecolor{codebgcolor}{HTML}{F1F3F5}}{}
\makeatother
\makeatletter
\ifdefined\Shaded\renewenvironment{Shaded}{\begin{tcolorbox}[boxrule=0pt, borderline west={3pt}{0pt}{shadecolor}, breakable, enhanced, sharp corners, frame hidden, colback={codebgcolor}]}{\end{tcolorbox}}\fi
\makeatother

% -----
% Body
% -----

\begin{document}

% Top matter -----

\pretextual

\frenchspacing

\selectlanguage{english}

%:::% class attribute begin/end %:::%

% -----
% Cover (Mandatory)
% -----

%:::% cover begin %:::%
\imprimircapa
%:::% cover end %:::%

% -----
% Title Page (Mandatory)
% -----

%:::% approval-sheet begin %:::%
\imprimirfolhaderosto
%:::% approval-sheet end %:::%

% -----
% Cataloging Record (Mandatory)
% -----

%:::% cataloging-record begin %:::%
\begin{fichacatalografica}
\hyphenpenalty=100000
%:::% cataloging-record body begin %:::%
I authorize and grant permission for the full or partial reproduction and distribution of this work by any conventional or electronic means for the  purposes of study and research, provided that the source is properly cited, the integrity of the original work is maintained, and such reproduction is not intended for commercial gain.

\vfill

\begin{center}
Cataloging in publication

[University. School. Library]

[CRB-8]

\medskip
\ABNTEXfontereduzida

\setlength{\fboxsep}{1cm}
\fbox{\begin{minipage}[c][6.5cm]{12.5cm}
[Author's surname], [Author's forename(s)]

\smallskip

\hspace{0.5cm} {\imprimirtitulo}  / {\imprimirautor}; supervisor, {\imprimirorientador}; co-supervisor {\imprimircoorientador}. {\imprimirlocal}, {\imprimirdata}

\smallskip

{\thelastpage}p. : il

\smallskip

\hspace{0.5cm} {\imprimirtipotrabalho} (\imprimirtituloacademico) -- {\imprimirprograma}, {\imprimirescola}, {\imprimiruniversidade}, {\imprimirdata}.

\smallskip

\hspace{0.5cm} {\imprimirnotadeversao}

\smallskip

\hspace{0.5cm} 1. [Subject A]. 2. [Subject B]. 3. [Subject C]. I. [Supervisor's surname], [Supervisor's forename(s)], super. II. [Co-supervisor's surname], [Co-supervisor's forename(s)] co-super. III. Title.
\end{minipage}}
\end{center}
\vspace{\hugeskipamount}
%:::% cataloging-record body end %:::%
\end{fichacatalografica}
%:::% cataloging-record end %:::%

% -----
% Errata (Optional)
% -----

%:::% errata begin %:::%
\begin{errata}[\errataname]
%:::% errata reference begin %:::%
\noindent [AUTHOR’S SURNAME], [Author’s forename(s) initial(s)]. \textbf{\imprimirtitulo}. {\imprimirdata}. {\thelastpage}p. {\imprimirtipotrabalho}  ({\imprimirtituloacademico}) -- {\imprimirescola}, {\imprimiruniversidade}, {\imprimirlocal}, {\imprimirdata}.
%:::% errata reference end %:::%
\smallskip
%:::% errata body begin %:::%

This is the preliminary version of this thesis. Any required corrections
will be listed here upon approval.

%:::% errata body end %:::%
\end{errata}
%:::% errata end %:::%

% -----
% Approval Sheet (Mandatory)
% -----

%:::% approval-sheet begin %:::%
\begin{folhadeaprovacao}[\folhadeaprovacaoname]
\hyphenpenalty=100000
%:::% approval-sheet body begin %:::%
{\imprimirtipotrabalho} by {\imprimirautor}, under the title \textbf{\imprimirtitulo}, presented to the {\imprimirescola} at the {\imprimiruniversidade}, as a requirement for the degree of {\imprimirtituloacademico} by the {\imprimirprograma}, in the concentration area of {\imprimirareadeconcentracao}.

\vspace{\hugeskipamount}
Approved on [Month] [Day], [Year].

\vspace{\hugeskipamount}
\begin{center}
  Examination Committee
\end{center}

\vspace{\smallskipamount}
Committee Chair:

\vspace{\tinyskipamount}
\begingroup

\AtBeginEnvironment{tabular}{
  \normalsize\raggedright
  \renewcommand{\arraystretch}{2}
}

\setlength{\arrayrulewidth}{0pt}
\setlength{\tabcolsep}{0cm}
\begin{tabular}{m{2.5cm} m{13.5cm}}
  Prof. Dr. & [Full name] \\
  Institution & [School], [University] \\
\end{tabular}

\vspace{\bigskipamount}
Examiners:

\vspace{\tinyskipamount}
\begin{tabular}{m{2.5cm} m{13.5cm}}
  Prof. Dr. & [Full name] \\
  Institution & [School], [University] \\
  Evaluation & [Approved/Rejected] \\
\end{tabular}

\vspace{\smallskipamount}
\begin{tabular}{m{2.5cm} m{13.5cm}}
  Prof. Dr. & [Full name] \\
  Institution & [School], [University] \\
  Evaluation & [Approved/Rejected] \\
\end{tabular}

\vspace{\smallskipamount}
\begin{tabular}{m{2.5cm} m{13.5cm}}
  Prof. Dr. & [Full name] \\
  Institution & [School], [University] \\
  Evaluation & [Approved/Rejected] \\
\end{tabular}
\endgroup
%:::% approval-sheet body end %:::%
\end{folhadeaprovacao}
%:::% approval-sheet end %:::%

% -----
% Inscription (Optional)
% -----

%:::% inscription begin %:::%
\begin{dedicatoria}[] % \dedicatorianame | Keep #1 empty.
\vspace*{\fill} % Don't change it.
\centering
%:::% inscription body begin %:::%
\textit{To the worm that first gnawed on the cold flesh of my corpse,}

\textit{I dedicate, as a fond remembrance, these posthumous memories.}\footnotemark{}

\footnotetext{
	ASSIS, M. \textbf{Memórias póstumas de Brás Cubas} [The Posthumous Memoirs of Brás Cubas]. São Paulo: Companhia das Letras, 2014. ISBN 978-85-438-0163-6.
}
%:::% inscription body end %:::%
\vspace*{\fill} % Don't change it.
\vspace{4.5cm}
% \vspace{13cm}
\end{dedicatoria}
%:::% inscription end %:::%

% -----
% Acknowledgments (Optional)
% -----

%:::% acknowledgments begin %:::%
\begin{agradecimentos}[\agradecimentosname]
  %:::% acknowledgments body begin %:::%

I would like to acknowledge this awesome
\href{https://github.com/danielvartan/abnt}{Quarto format}! :)

  %:::% acknowledgments body end %:::%
\end{agradecimentos}
%:::% acknowledgments end %:::%

% -----
% Epigraph (Optional)
% -----

%:::% epigraph begin %:::%
\begin{epigrafe}[] % \epigraphname | Keep #1 empty.
\vspace*{\fill} % Don't change it.
\begin{flushright}
%:::% epigraph body begin %:::%
\textit{Nullius in verba}\footnotemark{}

\footnotetext{
	THE ROYAL SOCIETY. \textbf{History of the Royal Society}. Available from: <\href{https://royalsociety.org/about-us/history/}{https://royalsociety.org/about-us/history/}>. Visited on: 9 Sept. 2023.
}
%:::% epigraph body end %:::%
\end{flushright}
\end{epigrafe}
%:::% epigraph end %:::%

% -----
% Abstract in Vernacular Language (Mandatory)
% -----

%:::% vernacular-abstract begin %:::%
\begin{resumoenv}[\resumoname]
 %:::% vernacular-abstract reference begin %:::%
[AUTHOR’S SURNAME], [Author’s forename(s) initial(s)]. \textbf{\imprimirtitulo}. {\imprimirdata}. {\thelastpage}p. {\imprimirtipotrabalho}  (\imprimirtituloacademico) -- {\imprimirescola}, {\imprimiruniversidade}, {\imprimirlocal}, {\imprimirdata}.
%:::% vernacular-abstract reference end %:::%

%:::% vernacular-abstract body begin %:::%

\href{https://github.com/danielvartan/abnt}{abnt} is a
\href{https://quarto.org}{Quarto} format designed for creating theses
and dissertations that comply with guidelines established by the
Brazilian Association of Technical Standards
(\href{https://www.abnt.org.br/}{ABNT}). Learn more at
\url{https://github.com/danielvartan/abnt}.

%:::% vernacular-abstract body end %:::%

%:::% vernacular-abstract keywords begin %:::%
\begin{tabular}{p{2.3cm} p{13.6cm}}
  \textbf{Keywords}: & [Keyword 1]. [Keyword 2]. [Keyword 3].
\end{tabular}
%:::% vernacular-abstract keywords end %:::%
\end{resumoenv}
%:::% vernacular-abstract end %:::%

% -----
% Abstract in Foreign Language (Mandatory)
% -----

%:::% foreign-abstract begin %:::%
\begin{resumoenv}[\resumoestrangeironame]
\begin{otherlanguage*}{brazil}
%:::% foreign-abstract reference begin %:::%
[SOBRENOME DO AUTOR], [Inicial(is) do(s) prenome(s) do autor].  \textbf{[Título]}. {\imprimirdata}. {\thelastpage}p. [Dissertação/Tese]  ([Título acadêmico]) -- [Escola/Faculdade], [Universidade], [Cidade/Local], {\imprimirdata}.
%:::% foreign-abstract reference end %:::%

%:::% foreign-abstract body begin %:::%

\href{https://github.com/danielvartan/abnt}{abnt} é um formato
\href{https://quarto.org}{Quarto} projetado para criar teses e
dissertações que atendem às diretrizes estabelecidas pela Associação
Brasileira de Normas Técnicas (\href{https://www.abnt.org.br/}{ABNT}).
Saiba mais em \url{https://github.com/danielvartan/abnt}.

%:::% foreign-abstract body end %:::%

%:::% foreign-abstract keywords begin %:::%
\begin{tabular}{p{3.6cm} p{12.3cm}}
  \textbf{Palavras-chaves}: & [Palavra-chave 1]. [Palavra-chave 2]. [Palavra-chave 3].
\end{tabular}
%:::% foreign-abstract keywords end %:::%
\end{otherlanguage*}
\end{resumoenv}
%:::% foreign-abstract end %:::%

% -----
% List of Figures (Optional)
% -----

%:::% list-of-figures begin %:::%
\pdfbookmark[0]{\listfigurename}{lof}
\listoffigures*
\cleardoublepage
%:::% list-of-figures end %:::%

% -----
% List of Tables (Optional)
% -----

%:::% list-of-tables begin %:::%
\pdfbookmark[0]{\listtablename}{lot}
\listoftables*
\cleardoublepage
%:::% list-of-tables end %:::%

% -----
% List of Abbreviations and Acronyms (Optional)
% -----

%:::% list-of-abbreviations begin %:::%
\begin{siglas}
%:::% list-of-abbreviations body begin %:::%

\begin{description}
\item[\textsubscript{F}]
\hspace{20cm}

Subscript indicating a relation with work-free days.
\item[\textsubscript{W}]
\hspace{20cm}

Subscript indicating a relation with workdays.
\item[MCTQ]
\hspace{20cm}

Munich ChronoType Questionnaire.
\item[MCTQ\textsuperscript{PT}]
\hspace{20cm}

Portuguese version of the MCTQ.
\item[MEQ]
\hspace{20cm}

Morningness-Eveningness Questionnaire.
\item[MSF]
\hspace{20cm}

Local time of mid-sleep on work-free days.
\item[MSF\textsubscript{sc}]
\hspace{20cm}

Chronotype proxy. The midpoint between sleep onset and sleep end on
work-free days. A sleep correction (\textsubscript{SC}) is made when a
possible sleep compensation related to a lack of sleep on workdays is
identified.
\item[MSW]
\hspace{20cm}

Local time of mid-sleep on workdays.
\end{description}

%:::% list-of-abbreviations body end %:::%
\end{siglas}
%:::% list-of-abbreviations end %:::%

% -----
% List of Symbols (Optional)
% -----

%:::% list-of-symbols begin %:::%
\begin{simbolos}
%:::% list-of-symbols body begin %:::%

For an extensive list of chronobiology related symbols, please refer to
\textcite{aschoff1965} and \textcite{marques2012}.

\begin{description}
\item[\(\tau\)]
\hspace{20cm}

Period of a rhythm in free flow; only revealed under constant
environmental conditions.
\item[\(T\)]
\hspace{20cm}

Zeitgeber period.
\item[\(\phi\)]
\hspace{20cm}

Phase.
\item[\(\Delta\phi\)]
\hspace{20cm}

Phase shift.
\item[\(+\Delta\phi\)]
\hspace{20cm}

Phase advance.
\item[\(-\Delta\phi\)]
\hspace{20cm}

Phase delay.
\item[\(\Psi\)]
\hspace{20cm}

Phase relation.
\end{description}

%:::% list-of-symbols body end %:::%
\end{simbolos}
%:::% list-of-symbols end %:::%

% -----
% Table of Contents (Mandatory)
% -----

%:::% table-of-contents begin %:::%
\pdfbookmark[0]{\contentsname}{toc}
\tableofcontents*
\cleardoublepage
%:::% table-of-contents end %:::%

% -----
% Other Additions
% -----


% Main and back matter -----

\textual
\bookmarksetup{startatroot}

\chapter{{[}Showcase{]} Introduction}\label{sec-introduction}

Consectetur maecenas aliquet facilisis, porta volutpat, porta nulla
cursus montes? Blandit gravida interdum turpis, dignissim maecenas
metus, aptent sem vehicula quis inceptos. Accumsan nascetur cubilia,
laoreet sed quis magna tempor dui cras. Hendrerit tortor sapien mauris
aptent imperdiet rutrum in primis augue tristique felis. Lacinia
suspendisse purus tortor natoque sollicitudin euismod -- nulla tempus
dui posuere. Cras rhoncus vel, aliquam vivamus nec semper torquent
praesent \autocite{einstein1905}. See Figure~\ref{fig-karl-popper}.

\index{Karl Popper}

\begingroup
\setlength{\ABNTEXcitacaorecuo}{4.25cm} % Default: 4cm

\begin{citacao}[english]
The activity can be represented by a \textit{general schema of problem-solving by the method of imaginative conjectures and criticism}, or, as I have often called it, by \textit{the method of conjecture and refutation}. The schema (in its simplest form) is this:

\hspace{3.65cm} $\text{P}_{1} \to \text{TT} \to \text{EE} \to \text{P}_{2}$

Here $\text{P}_{1}$ is the \textit{problem} from which we start,  $\text{TT}$ (the ‘tentative theory’) is the imaginative conjectural solution which we first reach, for example our first \textit{tentative interpretation}. $\text{EE}$ (\textit{‘error- elimination’}) consists of a severe critical examination of our conjecture, our tentative interpretation: it consists, for example, of the critical use of documentary evidence and, if we have at this early stage more than one conjecture at our disposal, it will also consist of a critical discussion and comparative evaluation of the competing conjectures. $\text{P}_{2}$ is the problem situation as it emerges from our first critical attempt to solve our problems.

It leads up to our second attempt (\textit{and so on}). A satisfactory understanding will be reached if the interpretation, the conjectural theory, finds support in the fact that it can throw new light on new problems --- on more problems than we expected; or if it finds support in the fact that it explains many sub-problems, some of which were not seen to start with. Thus we may say that we can gauge the progress we have made by comparing $\text{P}_{1}$ with some of our later problems ($\text{P}_{n}$, say).

\noindent \hspace*{\fill} \autocite[164]{popper1979a}
\end{citacao}
\endgroup

Adipiscing nostra aenean vivamus varius donec eleifend a ac feugiat
mauris. Fermentum nunc suscipit dictumst aenean posuere ad massa fusce?
Egestas dignissim, suscipit scelerisque blandit facilisi bibendum odio
habitant? Commodo tincidunt aptent velit lobortis ut penatibus
parturient sollicitudin gravida odio facilisis, imperdiet nibh vulputate
faucibus cras gravida ad urna lobortis in tempor fusce scelerisque;
magnis hendrerit -- tempus auctor aenean, venenatis malesuada torquent,
dictum, diam nullam ad sagittis facilisi gravida metus cursus pulvinar
rutrum torquent curae aliquam dui, id torquent duis tincidunt ornare
torquent nam magna sapien suscipit
\autocite{krakauer2019,mitchell2009a}.

\index{figures}

\begin{figure}[H]

\caption{\label{fig-karl-popper}Karl Popper (July 25, 1902 -- September
17, 1994).\\
One of the 20th century's most influential philosophers of science.}

\centering{

\includegraphics[width=0.5\linewidth,height=\textheight,keepaspectratio]{images/karl-popper.png}

\legend{\href{https://www.npg.org.uk/collections/search/portrait/mw08238/Sir-Karl-Raimund-Popper?}{Steve
Pyke}.}

}

\end{figure}%

\section{Heading 2}\label{heading-2}

Adipiscing aliquet luctus dapibus hac mattis urna metus eleifend felis?
Penatibus suspendisse lacinia tristique -- porttitor quisque est.
Integer consequat donec dui faucibus netus fames eget porta fames.
Montes aliquam vivamus magnis nascetur ad non dui magnis magnis
condimentum! Tortor hendrerit orci feugiat mi, tristique; ante accumsan
venenatis. Nisi posuere et ut nulla dictumst metus dui etiam vel
ultricies velit \autocite{pearson1900}. See Table~\ref{tbl-penguins}.

\index{tables}

\begin{table}

\caption{\label{tbl-penguins}A sample of the penguins dataset}

\centering{

\fontsize{9.0pt}{11.0pt}\selectfont
\begin{tabular*}{\linewidth}{@{\extracolsep{\fill}}ccrrr}
\toprule
\multicolumn{1}{l}{{\bfseries Species}} & \multicolumn{1}{l}{{\bfseries Island}} & {\bfseries Bill length (mm)} & {\bfseries Bill depth (mm)} & {\bfseries Flipper length (mm)} \\ 
\midrule\addlinespace[2.5pt]
\multicolumn{1}{l}{{Gentoo}} & \multicolumn{1}{l}{{Biscoe}} & 43.5 & 14.2 & 220 \\ 
\multicolumn{1}{l}{{Adelie}} & \multicolumn{1}{l}{{Torgersen}} & 36.2 & 17.2 & 187 \\ 
\multicolumn{1}{l}{{Chinstrap}} & \multicolumn{1}{l}{{Dream}} & 58.0 & 17.8 & 181 \\ 
\multicolumn{1}{l}{{Adelie}} & \multicolumn{1}{l}{{Dream}} & 37.0 & 16.9 & 185 \\ 
\multicolumn{1}{l}{{Adelie}} & \multicolumn{1}{l}{{Biscoe}} & 38.1 & 16.5 & 198 \\ 
\bottomrule
\end{tabular*}

\legend{Based on \textcite{horst2020} penguin dataset.}

}

\end{table}%

\subsection{Heading 3}\label{heading-3}

Consectetur facilisis montes, conubia eu justo luctus ante pulvinar ut
vehicula. Tempor nam in non, ridiculus proin et molestie fames. Praesent
curabitur est eu massa senectus proin; consequat malesuada nascetur.
Vestibulum ut vel fermentum commodo ac suscipit sagittis; mi cubilia
cursus inceptos in, dignissim penatibus inceptos mauris
\autocite{neyman1928}. See Table~\ref{tbl-cohens-benchmark}.

Amet mollis -- egestas risus cubilia enim duis? Ultricies id mus nec
integer; mollis libero. Facilisis maecenas tellus euismod, inceptos
consequat morbi. Leo viverra ultricies condimentum venenatis consequat
taciti viverra? At suspendisse volutpat phasellus quisque vitae
porttitor enim. Placerat quisque rutrum nisl praesent; aliquam pharetra
fringilla. Na fringilla \autocite{neyman1928a}. See
Figure~\ref{fig-gini-index}.

\index{figures!chart}

\begin{figure}[H]

\caption{\label{fig-gini-index}Gini index by Brazilian municipality in
2010. Values range from 0 (perfect equality) to 1 (total inequality).}

\centering{

\pandocbounded{\includegraphics[keepaspectratio]{index_files/figure-pdf/unnamed-chunk-9-1.png}}

\legendleft{Data extracted from \textcite{ibge}, processed by
\textcite{vartanian2025r}.}

}

\end{figure}%

\index{tables}

\clearpage
\begingroup
\begin{landscape}
\renewcommand{\arraystretch}{2.25}

\begin{table}

\caption{\label{tbl-cohens-benchmark}Cohen's benchmark for effect sizes}

\centering{

\fontsize{9.0pt}{11.0pt}\selectfont
\begin{tabular*}{\linewidth}{@{\extracolsep{\fill}}llccc}
\toprule
 &  & \multicolumn{3}{c}{{\bfseries Effect Size Classes}} \\ 
\cmidrule(lr){3-5}
{\bfseries Test} & {\bfseries Relevant Effect Size} & {\bfseries Small} & {\bfseries Medium} & {\bfseries Large} \\ 
\midrule\addlinespace[2.5pt]
Comparison of independent means & \(d\), \(\Delta\), Hedges' \(g\) & 0.20 & 0.50 & 0.80 \\ 
Comparison of two correlations & \(q\) & 0.10 & 0.30 & 0.50 \\ 
Difference between proportions & Cohen's \(g\) & 0.05 & 0.15 & 0.25 \\ 
Correlation & \(r\) & 0.10 & 0.30 & 0.50 \\ 
 & \(r^{2}\) & 0.01 & 0.09 & 0.25 \\ 
Crosstabulation & \(w\), \(\varphi\), \(V\), \(C\) & 0.10 & 0.30 & 0.50 \\ 
ANOVA & \(f\) & 0.10 & 0.25 & 0.40 \\ 
 & \(\eta^{2}\) & 0.01 & 0.06 & 0.14 \\ 
Multiple regression & \(R^{2}\) & 0.02 & 0.13 & 0.26 \\ 
 & \(f^{2}\) & 0.02 & 0.15 & 0.35 \\ 
\bottomrule
\end{tabular*}
\begin{minipage}{\linewidth}
\vspace{.05em}
 \\
 \\
Notes: The rationale for most of these benchmarks can be found in Cohen (1988) at the following pages: Cohen's \(d\) (p. 40), \(q\) (p. 115), Cohen's \(g\) (pp. 147--149), \(r\) and \(r^{2}\) (pp. 79--80), Cohen's \(w\) (pp. 224--227), \(f\) and \(\eta^{2}\) (pp. 285--287), \(R^{2}\) and \(f^{2}\) (pp. 413--414).\\
\end{minipage}

\legend{Adapted from \textcite{ellis2010}. Based on
\textcite{cohen1988a}.}

}

\end{table}%

\end{landscape}
\endgroup
\clearpage

\subsubsection{Heading 4}\label{heading-4}

Lorem volutpat hendrerit porta sem vehicula aliquam sed phasellus. Ante
cursus primis taciti interdum augue elementum imperdiet eu. Placerat
arcu nisi tristique sodales. Convallis enim lacus class suspendisse
porttitor litora ante conubia risus turpis! Quis nisl
\autocite{markov2006}. See Figure~\ref{fig-mpg}.

\index{figures!chart}

\begin{figure}[H]

\caption{\label{fig-mpg}Marginal distributions of weight (\emph{1000
lbs}) and fuel efficiency (\emph{miles per gallon}) for combustion
engine vehicles}

\centering{

\pandocbounded{\includegraphics[keepaspectratio]{index_files/figure-pdf/unnamed-chunk-12-1.png}}

\legendleft{Data extracted from the 1974 Motor Trend magazine, published
by \textcite{henderson1981}. Visualization by \textcite{holtz},
available at
\href{https://r-graph-gallery.com/277-marginal-histogram-for-ggplot2.html}{The
R Graph Gallery}.}

}

\end{figure}%

\paragraph{Heading 5}\label{heading-5}

Sit est tincidunt; augue quisque senectus natoque ullamcorper, convallis
curabitur vitae nunc. Tristique nullam sed, scelerisque egestas commodo
facilisis feugiat vitae natoque lobortis praesent? Ad pulvinar quisque
hendrerit magnis dictumst? Scelerisque dictumst augue per non cras,
aliquet aliquam posuere facilisi varius porta auctor. Laoreet erat
class, parturient hendrerit integer nibh, curabitur pulvinar accumsan
tortor, sociosqu sodales, mollis, viverra mi bibendum, nulla rutrum,
condimentum quam malesuada torquent ultricies penatibus non viverra
mattis, mollis morbi sagittis \autocite{perezgonzalez2015}. See
Definition~\ref{def-multiple-linear-regression}.

\begin{definition}[Multiple Linear
Regression]\protect\hypertarget{def-multiple-linear-regression}{}\label{def-multiple-linear-regression}

A linear regression of \(Y\) on \(k\) variables \(X_{1}, \dots, X_{k}\),
rather than on just a single variable \(X\). In a problem of multiple
linear regressions, we obtain \(n\) vectors of observations
(\(x_{i1}. \dots, x_{ik}, Y_{i}\)), for \(i = 1, \dots, n\). Here
\(x_{ij}\) is the observed value of the variable \(X_{j}\) for the
\(i\)th observation. The \(E(Y_{i})\) is given by the relation:
\autocite[738]{degroot2012a}

\begin{equation}\phantomsection\label{eq-multiple-linear-regression-function}{
E(Y_{i}) = \beta_{0} + \beta_{1} x_{i1} + \dots + \beta_{k} x_{ik}
}\end{equation}

\end{definition}

Adipiscing auctor sagittis, netus, mollis pulvinar interdum inceptos.
Primis nascetur congue nam class, leo nunc -- faucibus fermentum
tincidunt fames vestibulum curabitur? Auctor justo arcu, posuere id
dignissim risus sapien. Tristique inceptos mi arcu viverra gravida
malesuada arcu nam dictum nisl. Fermentum pretium lobortis dignissim
sociosqu, vulputate nisi cubilia vehicula laoreet. Eros libero dui
ligula diam cubilia pharetra vivamus commodo sociosqu pellentesque, ut
nulla nibh -- tincidunt quisque semper cum, eros bibendum gravida libero
aliquet netus, montes, turpis: rutrum, eu mollis, gravida diam fames
fames, himenaeos viverra pharetra, sed congue, sapien integer
ullamcorper euismod auctor cras nec aliquam sociis lectus dui porta
fames aptent \autocite{epstein2013a}. See Table~\ref{tbl-error-types}.

\index{tables}

\begin{table}

\caption{\label{tbl-error-types}Type I and type II errors}

\centering{

\fontsize{9.0pt}{11.0pt}\selectfont
\begin{tabular*}{\linewidth}{@{\extracolsep{\fill}}>{\centering\arraybackslash}p{\dimexpr 0.33\linewidth -2\tabcolsep-1.5\arrayrulewidth}>{\centering\arraybackslash}p{\dimexpr 0.33\linewidth -2\tabcolsep-1.5\arrayrulewidth}>{\centering\arraybackslash}p{\dimexpr 0.33\linewidth -2\tabcolsep-1.5\arrayrulewidth}}
\toprule
{\bfseries Decision about \(\text{H}_{0}\)} & {\bfseries \(\text{H}_{0}\) True} & {\bfseries \(\text{H}_{0}\) False} \\ 
\midrule\addlinespace[2.5pt]
 & {\cellcolor[HTML]{F1FFF1}{\textcolor[HTML]{000000}{Correct inference}}} & {\cellcolor[HTML]{FFEEEE}{\textcolor[HTML]{000000}{Type II error}}} \\ 
Accept & {\cellcolor[HTML]{F1FFF1}{\textcolor[HTML]{000000}{(True negative)}}} & {\cellcolor[HTML]{FFEEEE}{\textcolor[HTML]{000000}{(False negative)}}} \\ 
 & {\cellcolor[HTML]{F1FFF1}{\textcolor[HTML]{000000}{(\(1 - \alpha\))}}} & {\cellcolor[HTML]{FFEEEE}{\textcolor[HTML]{000000}{(\(\beta\))}}} \\ 
 & {\cellcolor[HTML]{FFEEEE}{\textcolor[HTML]{000000}{Type I error}}} & {\cellcolor[HTML]{F1FFF1}{\textcolor[HTML]{000000}{Correct inference}}} \\ 
Reject & {\cellcolor[HTML]{FFEEEE}{\textcolor[HTML]{000000}{(False positive)}}} & {\cellcolor[HTML]{F1FFF1}{\textcolor[HTML]{000000}{(True positive)}}} \\ 
 & {\cellcolor[HTML]{FFEEEE}{\textcolor[HTML]{000000}{(\(\alpha\))}}} & {\cellcolor[HTML]{F1FFF1}{\textcolor[HTML]{000000}{(\(1 - \beta\))}}} \\ 
\bottomrule
\end{tabular*}

\legend{Based on \textcite[383]{casella2002}}

}

\end{table}%

\bookmarksetup{startatroot}

\chapter{{[}Showcase{]} Development}\label{sec-development}

Ipsum mauris nec turpis ultricies dapibus montes, habitasse proin nostra
posuere nulla! Facilisis magnis iaculis nunc dis facilisi tristique
fusce phasellus inceptos senectus! Sem imperdiet pharetra rutrum
interdum metus primis tempus posuere est libero. Fermentum purus viverra
porttitor, bibendum, placerat pulvinar nostra volutpat nulla neque,
venenatis parturient: convallis mi, ultricies nam viverra quisque neque
natoque duis suscipit turpis \autocite{watson1953}.

\section{Heading 2}\label{heading-2-1}

the \href{https://en.wikipedia.org/wiki/Lorenz_system}{Lorenz system},
originally introduced by Edward N. Lorenz \autocite*{lorenz1963} in his
seminal paper, comprises three coupled, nonlinear ordinary differential
equations that model atmospheric convection, effectively illustrating
the chaotic nature of weather patterns.

The dynamics of the model are represented by the following set of
first-order, nonlinear differential equations:

\begin{equation}\phantomsection\label{eq-lorenz-system}{
\begin{aligned}
\frac{dx}{dt} &= \sigma(y - x), \\
\frac{dy}{dt} &= x(\rho - z) - y \\
\frac{dz}{dt} &= xy - \beta z
\end{aligned}
}\end{equation}

In these \hyperref[eq-lorenz-system]{equations}:

\begin{itemize}
\tightlist
\item
  \(x\) represents the rate of convection;
\item
  \(y\) denotes the horizontal temperature variation;
\item
  \(z\) indicates the vertical temperature variation;
\item
  \(\sigma\), \(\rho\), and \(\beta\) are system parameters
  corresponding to the
  \href{https://en.wikipedia.org/wiki/Prandtl_number}{Prandtl number},
  \href{https://en.wikipedia.org/wiki/Rayleigh_number}{Rayleigh number},
  and specific physical dimensions of the fluid layer.
\end{itemize}

The phase space visualization of this system
(Figure~\ref{fig-lorenz-system}) reveals intricate patterns and
behaviors, including the presence of strange attractors, which are
indicative of chaotic dynamics. To learn more about the Lorenz system,
see \textcite{lorenz2008}.

\index{figures!chart}

\index{Lorenz system}
\index{figures!chart}

\begin{figure}[H]

\caption{\label{fig-lorenz-system}Phase space visualization of the
Lorenz system}

\centering{

\pandocbounded{\includegraphics[keepaspectratio]{qmd/development_files/figure-pdf/unnamed-chunk-5-1.png}}

\legend{Reproduced from \textcite{vartanian2024d}. Based on
\textcite{lorenz1963}.}

}

\end{figure}%

\subsection{Heading 3}\label{heading-3-1}

Sit dui aliquet eu felis ut morbi sociis cubilia tincidunt vehicula nibh
pretium nec ac cubilia vulputate risus pretium cursus risus dictumst
placerat volutpat quisque nisl massa ac elementum curae malesuada
blandit sollicitudin curabitur congue cum scelerisque congue neque orci
\autocite{freire2013a}.

Adipiscing rhoncus vel class magna viverra imperdiet nullam, penatibus
erat torquent ullamcorper? Sollicitudin nam sociosqu neque eget non
aptent ut consequat facilisi hendrerit montes. Tincidunt morbi quisque
venenatis molestie neque ligula pulvinar senectus. Pharetra convallis
nisi dui justo in vulputate non est habitant inceptos maecenas taciti
inceptos metus, ultricies lacus montes vehicula morbi velit condimentum;
platea, litora sollicitudin volutpat cras ultrices eleifend dapibus orci
penatibus tristique cubilia, velit imperdiet vivamus tellus dignissim
purus himenaeos bibendum \autocite{walker2024a,peirce2011}. See
Figure~\ref{fig-sao-paulo-tracts}.

\index{figures!chart}

\begin{figure}[H]

\caption{\label{fig-sao-paulo-tracts}Population by census tract in 2022
in the city of São Paulo, Brazil}

\centering{

\pandocbounded{\includegraphics[keepaspectratio]{qmd/development_files/figure-pdf/unnamed-chunk-13-1.png}}

\legendleft{Tract boundary data from the
\href{https://www.ibge.gov.br/en/statistics/social/labor/22836-2022-census-3.html}{Brazilian
2022 Census} and municipal boundary data imported via the
\href{https://ipeagit.github.io/censobr/}{censobr} \autocite{pereirab}
and \href{https://ipeagit.github.io/geobr/}{geobr} \autocite{pereirab} R
packages.}

}

\end{figure}%

\subsubsection{Heading 4}\label{heading-4-1}

Lorem gravida elementum augue ante auctor venenatis felis ante enim
augue? Maecenas et vel ut nisi -- consequat quisque pellentesque
ridiculus: molestie mollis! Ridiculus magna accumsan ut cum, sapien
tempus. Cras sociis phasellus metus lobortis neque pretium quisque fusce
posuere velit conubia vestibulum tellus platea lacinia class est quis a
placerat est ultrices enim sociis facilisis nisi dis laoreet, torquent,
duis himenaeos viverra ad, tortor -- et mi, duis, velit, erat gravida
imperdiet cursus vitae justo, rhoncus; dis taciti erat ut quisque
facilisi risus fusce taciti, molestie faucibus phasellus, auctor porta
platea enim malesuada venenatis nullam nibh sollicitudin dictumst
condimentum habitant quis arcu malesuada etiam
\autocite{hopfield1984,darwin2017}. See Figure~\ref{fig-eruption}.

\index{figures!chart}

\begin{figure}[H]

\caption{\label{fig-eruption}Relation between \emph{waiting time to next
eruption} (minutes) and \emph{eruption time} (minutes) at Old Faithful
Geyser, Yellowstone National Park, Wyoming, USA}

\centering{

\pandocbounded{\includegraphics[keepaspectratio]{qmd/development_files/figure-pdf/unnamed-chunk-15-1.png}}

\legend{Reproduced from the
\href{https://ggplot2.tidyverse.org/reference/geom_density_2d.html}{ggplot2}
R package documentation \autocite{wickham2016a}.}

}

\end{figure}%

Elit ad ligula enim sodales, elementum risus erat metus eget mollis.
Ridiculus iaculis nam dictum eros eros -- at rutrum nostra ornare? Nisl
pellentesque consequat, praesent id aenean, elementum ante. Luctus porta
ornare at commodo penatibus lectus congue leo lacus aliquam a egestas
erat ligula aliquam himenaeos penatibus ante phasellus vestibulum erat
ac curae consequat natoque tempor vivamus potenti nunc neque cum
dictumst ornare proin ornare faucibus pretium netus maecenas consequat
nibh diam aliquet tristique, velit platea dictum hendrerit volutpat,
tellus blandit morbi varius massa ad fames -- vitae, lectus neque diam
eget, aptent tempor elementum sodales, nunc nostra; dictumst molestie
hendrerit imperdiet, erat, pellentesque conubia felis montes per dui
mattis donec proin curae conubia enim sagittis ullamcorper molestie
mattis eleifend purus per bibendum a ridiculus odio nec vitae morbi
gravida ridiculus tincidunt per posuere lectus ornare bibendum tempor
risus pharetra platea aptent lectus congue viverra libero tempus
maecenas per \autocite{holland2012}.

\paragraph{Heading 5}\label{heading-5-1}

The hypothesis can be outlined as follows:

\begin{description}
\tightlist
\item[\textbf{Null Hypothesis} (\(\text{H}_{0}\))]
Including \emph{latitude} as a predictor does not result in a meaningful
improvement in the model's fit, as evidenced by a change in the adjusted
\(\text{R}^{2}\) that is less than the Minimum Effect Size (MES).
\item[\textbf{Alternative Hypothesis} (\(\text{H}_{a}\))]
Including \emph{latitude} as a predictor results in a meaningful
improvement in the model's fit, as evidenced by a change in the adjusted
\(\text{R}^{2}\) that meets or exceeds the Minimum Effect Size (MES).
\end{description}

Formally:

\begin{conjecture}[Data
Test]\protect\hypertarget{cnj-hypothesis-test}{}\label{cnj-hypothesis-test}

\[
\begin{cases}
  \text{H}_{0}: \Delta \ \text{Adjusted} \ \text{R}^{2} < \text{MES} \\
  \text{H}_{a}: \Delta \ \text{Adjusted} \ \text{R}^{2} \geq \text{MES}
\end{cases}
\]

\end{conjecture}

Where:

\begin{equation}\phantomsection\label{eq-adjusted-r-squared-delta}{
\Delta \ \text{Adjusted} \ \text{R}^{2} = \text{Adjusted} \ \text{R}^{2}_{f} - \text{Adjusted} \ \text{R}^{2}_{r}
}\end{equation}

Lorem arcu egestas cum facilisi pharetra sapien litora. Egestas faucibus
libero ullamcorper nisi ac pellentesque, rutrum sed rutrum at --
elementum mi? Phasellus sapien ornare vulputate molestie lacinia sodales
mollis tristique placerat lectus iaculis molestie eu porta suscipit
phasellus phasellus eu nibh fames mus molestie eros cum suspendisse cum
faucibus nunc ultrices facilisi id a porttitor aliquet leo vivamus
pellentesque enim interdum diam ornare placerat porta malesuada quisque
fermentum arcu, eros dictum mattis, dui, fringilla ante hendrerit
ultricies, sociosqu dictumst sollicitudin, pulvinar purus, penatibus
cubilia, a, nullam turpis, placerat mauris, sociosqu dictum dis, magnis
nibh lobortis potenti, non, phasellus, urna class, sed fusce dui, enim
imperdiet duis nascetur bibendum curabitur class sem nullam diam, curae
mus eleifend accumsan ut phasellus molestie habitasse commodo
pellentesque mattis tempor magna pulvinar per turpis velit magna
eleifend dignissim dictumst ante cum nullam dui dignissim massa
himenaeos phasellus fusce sapien est platea congue gravida convallis
praesent tempus cras porta justo nam diam fusce eros aliquet
\autocite{thoreau2017}.

\bookmarksetup{startatroot}

\chapter{{[}Showcase{]} Conclusion}\label{sec-conclusion}

\epigraph{Every genuine test of a theory is an attempt to falsify it, or to refute it.}{--- \textcite{popper2002}}

\vspace*{-1\baselineskip}

\epigraph{I suggest that it is the aim of science to find satisfactory explanations, of whatever strikes us as being in need of explanation.}{--- \textcite[193]{popper1979a}}

Ipsum ad curae dapibus -- nulla scelerisque magna condimentum dapibus
commodo ac. Nisi mi lectus porttitor vestibulum enim proin sociosqu
magna egestas imperdiet viverra conubia placerat faucibus id facilisi,
ante neque magna tempus scelerisque vehicula, himenaeos lacinia, nibh eu
mattis lacinia hendrerit, conubia integer donec vestibulum mollis at
tortor semper dui rutrum facilisis orci scelerisque lacus nibh ut
quisque vestibulum elementum \autocite{gell-mann2023}.

\section{Heading 2}\label{heading-2-2}

Consectetur tempor eros luctus tempor sagittis nam iaculis. Torquent
torquent lacinia et gravida. Scelerisque condimentum nulla primis diam
varius, mauris etiam viverra rutrum pharetra consequat arcu malesuada.
Ut faucibus pulvinar habitasse per sem; convallis mauris tristique est
semper nam mollis. Semper conubia justo commodo quis auctor; augue vel
pellentesque sociis mattis. Massa blandit id tincidunt et, fringilla
magna per cum ullamcorper -- nec morbi congue, id justo senectus aptent
et rhoncus consequat, torquent interdum ante scelerisque aenean hac
magna \autocite{lecun2015}.

Elit cursus maecenas nullam quisque, etiam -- maecenas aenean nulla
tempus. Eu risus, vulputate turpis risus; nisl tincidunt integer
faucibus viverra? Pretium turpis scelerisque mollis dictum cum? Nisi
mollis per ac in vel rutrum. Himenaeos molestie primis non placerat
morbi laoreet tincidunt enim, scelerisque justo venenatis montes nostra
curae, sollicitudin ornare scelerisque quisque est rutrum ante tincidunt
vehicula commodo viverra aenean rhoncus penatibus iaculis, faucibus dui
mattis, sagittis dictum elementum \autocite{foster1972}.

\subsection{Heading 3}\label{heading-3-2}

Amet sollicitudin pellentesque leo -- luctus, habitant proin sodales
morbi. Sed convallis turpis, netus fusce sodales ut cubilia arcu
pulvinar diam, semper nostra rhoncus! Odio non potenti tortor integer
molestie eros nisl. Dictumst nostra proin venenatis faucibus ac commodo
turpis, himenaeos mi netus. Vivamus congue semper cum curabitur, nam --
quisque neque euismod \autocite{more2014}.

\begin{itemize}
\tightlist
\item
  Bullet point

  \begin{itemize}
  \tightlist
  \item
    Bullet point

    \begin{itemize}
    \tightlist
    \item
      Bullet point
    \end{itemize}
  \end{itemize}
\end{itemize}

\subsubsection{Heading 4}\label{heading-4-2}

Amet erat id ultricies, lacus malesuada platea varius cras! Himenaeos et
eu sociosqu natoque laoreet vel egestas. Laoreet donec convallis
scelerisque feugiat, duis, auctor potenti etiam, dictumst imperdiet cras
blandit. Vehicula sed tortor ridiculus, sodales dignissim parturient,
commodo aliquet pellentesque. Ac velit nec augue faucibus, scelerisque
semper suspendisse aliquam lacinia viverra nec molestie eleifend
suscipit, ultricies taciti: ullamcorper, lacinia ornare scelerisque dis,
ultricies est justo turpis ad magnis, senectus risus duis augue
phasellus euismod parturient posuere fusce \autocite{papert2020}. See
Table~\ref{tbl-error-types}.

\begin{enumerate}
\def\labelenumi{\arabic{enumi}.}
\tightlist
\item
  List
\item
  List
\item
  List
\end{enumerate}

\paragraph{Heading 5}\label{heading-5-2}

Elit convallis arcu -- nam dui penatibus. Blandit convallis natoque ut
donec nascetur. Phasellus sollicitudin fermentum interdum ornare velit.
Praesent nibh quisque odio senectus tristique. Accumsan taciti ridiculus
rutrum accumsan morbi? Na morbi \autocite{tisue2004,tisue2004a}.

\postextual

\begingroup
\renewcommand{\baselinestretch}{1}
\setcounter{footnote}{0}
\renewcommand{\thefootnote}{\fnsymbol{footnote}}

\printbibliography[
  heading=bibheading
]

\endgroup

\tocskipone
\tocprintchapternonum
\addcontentsline{toc}{chapter}{\newbibname}

\begin{glossario}

\bookmarksetup{startatroot}

\chapter*{Glossary}\label{glossary}
\addcontentsline{toc}{chapter}{Glossary}

\markboth{Glossary}{Glossary}

For an extensive list of chronobiology related terms and definitions,
please refer to \textcite{aschoff1965} and \textcite{marques2012}.

\begin{description}
\item[Chronotype]
\hspace{20cm}

Any kind of temporal phenotype \autocite{ehret1974,pittendrigh1993}.
Usually, it refers to circadian phenotypes in a spectrum that goes from
morningness to eveningness \autocite{roenneberg2003b}. It can also be
seen as an organism's phase of entrainment \autocite{roenneberg2012a}.
\item[Circadian rhythm]
\hspace{20cm}

A rhythm with a period close to a day/24h, an approximation to the
period of the earth's rotation \autocite{pittendrigh1960}. From the
Latin \emph{circā}, around, and \emph{dĭes}, day \autocite{latinitium}.
Example: the sleep-wake cycle.
\item[Complex system]
\hspace{20cm}

Several definitions exist. The following are particularly useful:
\end{description}

\begin{itemize}
\tightlist
\item
  ``Systems that don't yield to compact forms of representation or
  description'' (David Krakauer apud \textcite{mitchell2013})
\item
  ``A system of many interacting parts where the system is more than
  just the sum of its parts'' (Mark Newman apud \textcite{mitchell2013})
\end{itemize}

\begin{description}
\item[Entrainment]
\hspace{20cm}

A shift and alignment of biological rhythms induced by a zeitgeber input
\autocite{kuhlman2018}. For example: a shift/alignment of an organism's
circadian rhythm when exposed to light.
\end{description}

\end{glossario}

\begin{apendicesenv}

\cleardoublepage
\phantomsection
\addcontentsline{toc}{part}{Appendices}
\appendix

\chapter{{[}Showcase{]}}\label{sec-appendix-a}

Ipsum molestie hendrerit potenti rhoncus sagittis. Et pulvinar massa
montes urna cras rhoncus lectus quam felis risus magna quisque torquent
luctus eros ultrices, tellus curae purus suscipit porttitor magnis
donec, aliquet orci dictum scelerisque, ac, sociis, volutpat euismod
bibendum, elementum, risus posuere non, quisque quisque: donec nisi nunc
auctor convallis scelerisque iaculis sodales mus praesent class mollis
ad ornare himenaeos nostra \autocite{vonneumann1993a}.

\section{Heading 2}\label{heading-2-3}

Ipsum condimentum sem ultricies class augue vestibulum congue nostra
cubilia ridiculus. Ullamcorper nostra dapibus sapien porta; inceptos
urna ultrices; nam facilisi conubia. Nibh per viverra, praesent morbi,
sapien aenean condimentum, penatibus in mollis, himenaeos ultricies
iaculis! Nec varius at placerat habitasse pellentesque penatibus:
eleifend sollicitudin curae magna. Neque rhoncus vulputate, dis,
convallis accumsan venenatis pretium cras. Quam sapien iaculis felis
scelerisque suspendisse dapibus, cum rhoncus dapibus quis sem, integer
rhoncus diam est gravida etiam facilisi, ultricies viverra luctus sed
arcu non scelerisque urna bibendum sed laoreet mi leo a quam tincidunt
\autocite{lorenz1963}.

\subsection{Heading 3}\label{heading-3-3}

Ipsum gravida elementum augue ante auctor venenatis! Felis ante enim,
augue maecenas et vel, ut nisi consequat quisque pellentesque ridiculus.
Molestie mollis ridiculus magna accumsan, ut cum sapien tempus. Cras
sociis phasellus metus lobortis neque pretium quisque fusce posuere
velit conubia vestibulum tellus platea lacinia class, est quis a
placerat est ultrices; enim sociis facilisis, nisi dis laoreet torquent
duis himenaeos, viverra ad tortor et mi duis velit erat gravida
imperdiet cursus vitae justo rhoncus \autocite{hinton1986}.

\end{apendicesenv}

\begin{anexosenv}

\chapter{{[}Showcase{]}}\label{sec-annex-a}

\clearpage
\includepdf{images/annex.pdf}

\chapter{Settings}\label{sec-settings}

\begin{tcolorbox}[enhanced jigsaw, leftrule=.75mm, breakable, left=2mm, toprule=.15mm, colback=white, opacitybacktitle=0.6, colframe=quarto-callout-note-color-frame, toptitle=1mm, arc=.35mm, colbacktitle=quarto-callout-note-color!10!white, opacityback=0, title=\textcolor{quarto-callout-note-color}{\faInfo}\hspace{0.5em}{Note}, coltitle=black, rightrule=.15mm, bottomrule=.15mm, bottomtitle=1mm, titlerule=0mm]

Please note that not all settings are documented. For issues or
discussions, use the
\href{https://github.com/danielvartan/abnt/issues}{\emph{Issues}} and
\href{https://github.com/danielvartan/abnt/discussions}{\emph{Discussions}}
tabs in the project 's GitHub repository.

For more information, check the format
\href{https://github.com/danielvartan/abnt/blob/main/_extensions/abnt/_extension.yml}{extension
YAML file} and the
\href{https://quarto.org/docs/reference/formats/pdf.html}{PDF Options}
section in the Quarto guide.

\end{tcolorbox}

\section{PDF Engine}\label{pdf-engine}

This format use \href{https://www.luatex.org/}{\texttt{lualatex}}. I
recommended always using this engine for better font handling and
feature support.

\section{Typography}\label{typography}

\index{tipography}

\subsection{Typefaces}\label{typefaces}

To change typefaces, use Quarto
\href{https://quarto.org/docs/reference/formats/pdf.html}{PDF options}
such as \texttt{mainfont}, \texttt{monofont}, and \texttt{sansfont}.

\begin{Shaded}
\begin{Highlighting}[]
\FunctionTok{format}\KeywordTok{:}
\AttributeTok{  }\FunctionTok{abnt{-}pdf}\KeywordTok{:}
\AttributeTok{    }\FunctionTok{mainfont}\KeywordTok{:}\AttributeTok{ Arial}
\end{Highlighting}
\end{Shaded}

The ABNT NBR 14724:2011 norm does not specify a required font. You can
choose any font, provided it is installed on your computer.

To see all the fonts installed in your system, run using R:

\begin{Shaded}
\begin{Highlighting}[]
\CommentTok{\# install.packages("systemfonts")}
\FunctionTok{library}\NormalTok{(systemfonts)}

\FunctionTok{system\_fonts}\NormalTok{()}
\end{Highlighting}
\end{Shaded}

\subsection{Line-Height and Font Size}\label{line-height-and-font-size}

Use \texttt{fontsize} and \texttt{linestretch} in \texttt{quarto.yml} to
adjust these options.

\begin{Shaded}
\begin{Highlighting}[]
\FunctionTok{format}\KeywordTok{:}
\AttributeTok{  }\FunctionTok{abnt{-}pdf}\KeywordTok{:}
\AttributeTok{    }\FunctionTok{fontsize}\KeywordTok{:}\AttributeTok{ 12pt}
\AttributeTok{    }\FunctionTok{linestretch}\KeywordTok{:}\AttributeTok{ }\FloatTok{1.5}
\end{Highlighting}
\end{Shaded}

It's important to note that the third paragraph of Section 5.1 of ABNT
NBR 14724:2011 norm establishes that the font size should be 12pt for
the entire document, including the cover, except for quotations longer
than three lines, footnotes, pagination, cataloging data, captions, and
sources of illustrations and tables, which should be in a smaller and
uniform size.

\section{Language and Hyphenation}\label{language-and-hyphenation}

This format already includes support for multiple languages and
hyphenation patterns. These patterns will be automatically loaded based
on the selected language.

\section{Document Sections}\label{document-sections}

\subsection{Pre-Textual Sections}\label{pre-textual-sections}

\texttt{abnt} uses a system of tags to transfer and render the content
of Quarto files to LaTeX. These tags look like this:

\begin{verbatim}
%:::% class attribute begin/end %:::%
\end{verbatim}

Unless you want to customize the template, you don't need to modify the
\texttt{.tex} files. You can write directly in the \texttt{.qmd} files.
Just ensure that you preserve all the tags.

\subsection{Including LaTeX Commands in Quarto
Files}\label{including-latex-commands-in-quarto-files}

To add LaTeX commands always use a code block with \texttt{\{=latex\}}:

\begin{Shaded}
\begin{Highlighting}[]
\NormalTok{\textasciigrave{}\textasciigrave{}\textasciigrave{}\{=latex\}}
\CommentTok{\% Some LaTeX code.}
\NormalTok{\textasciigrave{}\textasciigrave{}\textasciigrave{}}
\end{Highlighting}
\end{Shaded}

\subsection{Customizing the Sections}\label{customizing-the-sections}

I plan to implement this format more straightforwardly using
\href{https://quarto.org/docs/extensions/filters.html}{Lua filters} to
directly manipulate the LaTeX code generated by
\href{https://pandoc.org/lua-filters.html}{Pandoc}. Until then, you can
customize the sections using the following instructions.

\subsubsection{Removing Pre-Textual
Sections}\label{removing-pre-textual-sections}

All sections specified in the \href{https://teses.usp.br}{Teses USP}
guidelines are included in this format.

If you don't need a pre-textual section (e.g., Errata, Acknowledgments),
remove it from \texttt{tex/include-before-body.tex}, and
\texttt{R/.pre-render.R}.

\subsubsection{Removing Textual
Sections}\label{removing-textual-sections}

For textual sections (e.g., chapters), remove them from the
\texttt{\_quarto.yml} file, but be careful when removing the last
chapter. See the next section for more details.

\subsubsection{Removing Post-Textual
Sections}\label{removing-post-textual-sections}

You must be careful when removing post-textual sections (e.g.,
Appendices, Annexes). Each post-textual section starts and ends with a
specific LaTeX command (e.g.,
\texttt{\textbackslash{}begin\{anexosenv\}}).

For example, to remove the Glossary, remove it from
\texttt{\_quarto.yml} and copy the LaTeX code after
\texttt{\textless{}!-\/-\ glossary\ end\ -\/-\textgreater{}} in
\texttt{glossary.qmd} to the bottom of the last chapter.

The same applies when removing appendices. The last appendix must
include LaTeX code at the end to end the section and initialize the
Annexes section.

The Index section is the exception. To remove this section, add the
following to your \texttt{\_quarto.yml} file:

\begin{Shaded}
\begin{Highlighting}[]
\FunctionTok{format}\KeywordTok{:}
\AttributeTok{  }\FunctionTok{abnt{-}pdf}\KeywordTok{:}
\AttributeTok{    }\FunctionTok{index{-}page}\KeywordTok{:}\AttributeTok{ }\CharTok{false}
\end{Highlighting}
\end{Shaded}

In summary, transitions between document sections are created by
inserting LaTeX code at the end of specific sections:

\begin{itemize}
\tightlist
\item
  Between the last chapter and the Glossary section
\item
  Between the Glossary section and the Appendices section
\item
  Between the Appendices and Annexes sections
\item
  After the Annexes section
\end{itemize}

\section{Citation Management}\label{citation-management}

\subsection{Methods}\label{methods}

\index{BibLaTeX}

This Quarto format is specifically designed along with
\href{https://www.ctan.org/pkg/biblatex}{BibLaTeX}. For detailed
guidance on handling citations, refer to Quarto's
\href{https://quarto.org/docs/authoring/footnotes-and-citations.html}{Citation
\& Footnotes} documentation.

\subsection{Styles}\label{styles}

\index{ABNT}
\index{APA}

This format works with all
\href{https://www.ctan.org/pkg/biblatex}{BibLaTeX} built-in styles,
including \href{https://www.abnt.org.br/}{ABNT} (Brazilian Association
of Technical Standards) and \href{https://apastyle.apa.org/}{APA}
(American Psychological Association).

To use a specific style, change the \texttt{biblatexoptions} option in
the \texttt{\_quarto.yml} file. Several citation-related options are
available. Refer to the
\href{https://www.ctan.org/pkg/biblatex}{\texttt{biblatex}} manual for
complete documentation.

\begin{Shaded}
\begin{Highlighting}[]
\FunctionTok{format}\KeywordTok{:}
\AttributeTok{  }\FunctionTok{abnt{-}pdf}\KeywordTok{:}
\AttributeTok{    }\FunctionTok{biblatexoptions}\KeywordTok{:}
\AttributeTok{      }\KeywordTok{{-}}\AttributeTok{ backend=biber}
\AttributeTok{      }\KeywordTok{{-}}\AttributeTok{ url=true}
\AttributeTok{      }\KeywordTok{{-}}\AttributeTok{ useprefix=false}
\AttributeTok{      }\KeywordTok{{-}}\AttributeTok{ giveninits=true}
\AttributeTok{      }\KeywordTok{{-}}\AttributeTok{ style=abnt}
\AttributeTok{    }\FunctionTok{bibhang}\KeywordTok{:}\AttributeTok{ 0cm}\CommentTok{ \# Use 0.5cm if \textasciigrave{}style=apa\textasciigrave{}}
\AttributeTok{    }\FunctionTok{bibparsep}\KeywordTok{:}\AttributeTok{ {-}2ex}\CommentTok{ \# Use 1ex \textasciigrave{}style=apa\textasciigrave{}}
\FunctionTok{    biblio{-}footnote}\KeywordTok{: }\CharTok{\textgreater{}}
\NormalTok{      According to the Brazilian Association of Technical Standards}
\NormalTok{      (ABNT NBR 6023).}
\end{Highlighting}
\end{Shaded}

\subsection{Zotero \& Better BibTex}\label{zotero-better-bibtex}

If you use \href{https://www.zotero.org/}{Zotero} with
\href{https://retorque.re/zotero-better-bibtex/}{Better BibTex},
configure it to export URLs only when no DOI is available. This helps
declutter the References section. See
\url{https://tex.stackexchange.com/a/603358/234832} for instructions.

\section{Figures and Tables}\label{figures-and-tables}

To add top and bottom captions to figures and tables, enclose your
content in \texttt{divs} as shown below. The first paragraph after the
content will be rendered as the source (bottom caption), and the last
paragraph will be the top caption.

\begin{Shaded}
\begin{Highlighting}[]
\NormalTok{::: \{\#fig{-}1\}}
\NormalTok{::: \{.figure{-}content\}}
\NormalTok{This is the figure content.}
\NormalTok{:::}

\CommentTok{[}\OtherTok{This is the source.}\CommentTok{]}\NormalTok{\{.legend\}}

\NormalTok{This is the caption}
\NormalTok{:::}
\end{Highlighting}
\end{Shaded}

For figures/tables rendered in code chunks, use the following format:

\begin{Shaded}
\begin{Highlighting}[]
\NormalTok{::: \{\#fig{-}2\}}
\InformationTok{\textasciigrave{}\textasciigrave{}\textasciigrave{}\{r\}}
\FunctionTok{library}\NormalTok{(dplyr)}

\NormalTok{penguins }\SpecialCharTok{|\textgreater{}}
  \FunctionTok{pull}\NormalTok{(bill\_length\_mm) }\SpecialCharTok{|\textgreater{}}
  \FunctionTok{hist}\NormalTok{()}
\InformationTok{\textasciigrave{}\textasciigrave{}\textasciigrave{}}

\CommentTok{[}\OtherTok{This is the source.}\CommentTok{]}\NormalTok{\{.legend\}}

\NormalTok{This is the caption}
\NormalTok{:::}
\end{Highlighting}
\end{Shaded}

Note that when using \texttt{{[}{]}\{.legend\}} or
\texttt{{[}{]}\{.legendleft\}} the caption will always start with
\texttt{Source:}.

Like all cross-reference elements, these \texttt{divs} must follow a
naming pattern. Use the prefix \texttt{\#fig-} for figures and
\texttt{\#tbl-} for tables. For more information, see the
\hyperref[cross-referenceable-elements]{\emph{Cross-Referenceable
Elements}} section and refer to Quarto's
\href{https://quarto.org/docs/prerelease/1.4/crossref.html}{Cross-Referenceable
Elements} article.

\section{Cross-Referenceable
Elements}\label{cross-referenceable-elements}

Quarto allow you to create and reference almost anything by using
\texttt{div} enclosures. For example: See Theorem~\ref{thm-line}.

\begin{theorem}[Line]\protect\hypertarget{thm-line}{}\label{thm-line}

The equation of any straight line can be written as:

\[
y = mx + b
\]

\end{theorem}

However, it's important to note that for this to work, each type of
\texttt{div} must use pre-defined prefixes. If you don't follow these
rules your document can behave unexpectedly.

For more information about cross-reference elements, see Quarto's guide
\href{https://quarto.org/docs/books/book-crossrefs.html}{Book
Crossrefs},
\href{https://quarto.org/docs/authoring/cross-references.html}{Cross
References} and
\href{https://quarto.org/docs/prerelease/1.4/crossref.html}{Cross-Referenceable
Elements} articles.

\section{Freeze and Cache}\label{freeze-and-cache}

Avoid using the
\href{https://quarto.org/docs/projects/code-execution.html\#freeze}{\texttt{freeze}}
and
\href{https://quarto.org/docs/projects/code-execution.html\#cache}{\texttt{cache}}
execution options. If you're having issue in updating your document,
check if these options are enabled.

\end{anexosenv}



\tocskipone
\tocprintchapternonum

\begingroup
\ABNTEXfontereduzida
\renewcommand{\baselinestretch}{1}
\setlength{\parindent}{0pt}
\setlength{\parskip}{\tinyskipamount}
\setlength{\afterchapskip}{\hugeskipamount}
\phantompart
\printindex
\endgroup
\end{document}
