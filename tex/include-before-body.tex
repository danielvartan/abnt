%:::% class attribute begin/end %:::%

% -----
% Cover (mandatory)
% -----

%:::% cover begin %:::%
\imprimircapa
%:::% cover end %:::%

% -----
% Title page (mandatory)
% -----

% Use `\imprimirfolhaderosto*` when applying the option `two-side`.

%:::% approval-sheet begin %:::%
\imprimirfolhaderosto
%:::% approval-sheet end %:::%

% -----
% Cataloging record (mandatory)
% -----

%:::% cataloging-record begin %:::%
\begin{fichacatalografica}
%:::% cataloging-record body begin %:::%
\begingroup
\normalsize I authorize the full or partial reproduction of this work by any conventional or electronic means for the purposes of study and research, provided that the source is cited.
\endgroup

\vfill

\begin{center}
Cataloging record prepared by the [Library name] with the data \\ 
inserted by [Full name of the librarian] ([Librarian register number]).
\vspace{1em}

\setlength{\fboxsep}{1cm}
\fbox{\begin{minipage}[c][7.5cm]{12.5cm}
[Author's surname], [Author's forename(s)]

\hspace{0.5cm} {\imprimirtitulo}  / {\imprimirautor}; supervisor, {\imprimirorientador}; co-supervisor {\imprimircoorientador}. -- {\imprimirlocal}, {\imprimirdata}

\hspace{0.5cm} {\thelastpage}p. : il
\vspace{1em}

\hspace{0.5cm} {\imprimirtipotrabalho} (\imprimirtituloacademico) -- {\imprimirprograma}, {\imprimirescola}, {\imprimiruniversidade}, {\imprimirdata}.

\hspace{0.5cm} {\imprimirnotadeversao}
\vspace{1em}

\hspace{0.5cm} 1. [Subject A]. 2. [Subject B]. 3. [Subject C]. I. [Supervisor's surname], [Supervisor's forename(s)], super. II. [Co-supervisor's surname], [Co-supervisor's forename(s)] co-super. III. Title.
\end{minipage}}
\end{center}
%:::% cataloging-record body end %:::%
\end{fichacatalografica}
%:::% cataloging-record end %:::%

% -----
% Errata (optional)
% -----

%:::% errata begin %:::%
\begin{errata}
%:::% errata body begin %:::%

This is the development version of the thesis (version \textless1.0.0).
Any necessary corrections will be listed here after its approval.

%:::% errata body end %:::%
\end{errata}
%:::% errata end %:::%

% -----
% Approval sheet (mandatory)
% -----

%:::% approval-sheet begin %:::%
\begin{folhadeaprovacao}
  \noindent
%:::% approval-sheet header begin %:::%
{\imprimirtipotrabalho} by {\imprimirautor}, under the title \textbf{\imprimirtitulo}, presented to the {\imprimirescola} at the {\imprimiruniversidade}, as part of the requirements for the degree of {\imprimirtituloacademico} by the {\imprimirprograma}, in the concentration area of {\imprimirareadeconcentracao}.
%:::% approval-sheet header end %:::%
  \par\vspace*{1.5cm}
  \noindent Approved on \_\_\_\_\_\_\_\_\_\_\_\_\_\_\_\_\_\_\_\_ , \_\_\_\_\_\_\_\_\_\_ .

  \vspace*{1.5cm}

  \begin{center}
    \noindent Examination committee
  \end{center}

  \vspace*{0.5cm}

  \noindent Committee chair:

  \vspace*{0.25cm}

  \renewcommand{\arraystretch}{2}
  \setlength{\arrayrulewidth}{0pt}
  \setlength{\tabcolsep}{0pt}
  \noindent
  \begin{tabular}{m{2cm} P{14cm}}
    Prof. Dr. & \_\_\_\_\_\_\_\_\_\_\_\_\_\_\_\_\_\_\_\_\_\_\_\_\_\_\_\_\_\_\_\_\_\_\_\_\_\_\_\_\_\_\_\_\_\_\_\_\_\_\_\_\_\_\_ \\
    Institution & \_\_\_\_\_\_\_\_\_\_\_\_\_\_\_\_\_\_\_\_\_\_\_\_\_\_\_\_\_\_\_\_\_\_\_\_\_\_\_\_\_\_\_\_\_\_\_\_\_\_\_\_\_\_\_ \\
  \end{tabular}

  \vspace*{1cm}

  \noindent Examiners:

  \vspace*{0.25cm}

  \noindent
  \begin{tabular}{m{2cm} P{14cm}}
    Prof. Dr. & \_\_\_\_\_\_\_\_\_\_\_\_\_\_\_\_\_\_\_\_\_\_\_\_\_\_\_\_\_\_\_\_\_\_\_\_\_\_\_\_\_\_\_\_\_\_\_\_\_\_\_\_\_\_\_ \\
    Institution & \_\_\_\_\_\_\_\_\_\_\_\_\_\_\_\_\_\_\_\_\_\_\_\_\_\_\_\_\_\_\_\_\_\_\_\_\_\_\_\_\_\_\_\_\_\_\_\_\_\_\_\_\_\_\_ \\
    Evaluation & \_\_\_\_\_\_\_\_\_\_\_\_\_\_\_\_\_\_\_\_\_\_\_\_\_\_\_\_\_\_\_\_\_\_\_\_\_\_\_\_\_\_\_\_\_\_\_\_\_\_\_\_\_\_\_ \\
  \end{tabular}

  \vspace*{0.5cm}

  \noindent
  \begin{tabular}{m{2cm} P{14cm}}
    Prof. Dr. & \_\_\_\_\_\_\_\_\_\_\_\_\_\_\_\_\_\_\_\_\_\_\_\_\_\_\_\_\_\_\_\_\_\_\_\_\_\_\_\_\_\_\_\_\_\_\_\_\_\_\_\_\_\_\_ \\
    Institution & \_\_\_\_\_\_\_\_\_\_\_\_\_\_\_\_\_\_\_\_\_\_\_\_\_\_\_\_\_\_\_\_\_\_\_\_\_\_\_\_\_\_\_\_\_\_\_\_\_\_\_\_\_\_\_ \\
    Evaluation & \_\_\_\_\_\_\_\_\_\_\_\_\_\_\_\_\_\_\_\_\_\_\_\_\_\_\_\_\_\_\_\_\_\_\_\_\_\_\_\_\_\_\_\_\_\_\_\_\_\_\_\_\_\_\_ \\
  \end{tabular}

  \vspace*{0.5cm}

  \noindent
  \begin{tabular}{m{2cm} P{14cm}}
    Prof. Dr. & \_\_\_\_\_\_\_\_\_\_\_\_\_\_\_\_\_\_\_\_\_\_\_\_\_\_\_\_\_\_\_\_\_\_\_\_\_\_\_\_\_\_\_\_\_\_\_\_\_\_\_\_\_\_\_ \\
    Institution & \_\_\_\_\_\_\_\_\_\_\_\_\_\_\_\_\_\_\_\_\_\_\_\_\_\_\_\_\_\_\_\_\_\_\_\_\_\_\_\_\_\_\_\_\_\_\_\_\_\_\_\_\_\_\_ \\
    Evaluation & \_\_\_\_\_\_\_\_\_\_\_\_\_\_\_\_\_\_\_\_\_\_\_\_\_\_\_\_\_\_\_\_\_\_\_\_\_\_\_\_\_\_\_\_\_\_\_\_\_\_\_\_\_\_\_ \\
  \end{tabular}
\end{folhadeaprovacao}
%:::% approval-sheet end %:::%

% -----
% Inscription (optional)
% -----

%:::% inscription begin %:::%
\begin{dedicatoria}
  \vspace*{\fill}
  \centering
%:::% inscription body begin %:::%
\textit{To the worm that first gnawed on the cold flesh of my corpse,}

\textit{I dedicate, as a fond remembrance, these posthumous memories.}\footnotemark{}

\footnotetext{ASSIS, M. \textbf{Memórias póstumas de Brás Cubas} [The Posthumous Memoirs of Brás Cubas]. São Paulo: Companhia das Letras, 2014. ISBN 978-85-438-0163-6.}
%:::% inscription body end %:::%
	\vspace*{\fill}
\end{dedicatoria}
%:::% inscription end %:::%

% -----
% Acknowledgments (optional)
% -----

%:::% acknowledgments begin %:::%
\begin{agradecimentos}
  %:::% acknowledgments body begin %:::%

I would like to acknowledge this awesome
\href{https://github.com/danielvartan/abnt}{Quarto format}! :)

  %:::% acknowledgments body end %:::%
\end{agradecimentos}
%:::% acknowledgments end %:::%

% -----
% Epigraph (optional)
% -----

%:::% epigraph begin %:::%
\begin{epigrafe}
  \vspace*{\fill}
	\begin{flushright}
	  %:::% epigraph body begin %:::%
\textit{Nullius in verba}\footnotemark{}

\footnotetext{THE ROYAL SOCIETY. \textbf{History of the Royal Society}. Available from:
<\href{https://royalsociety.org/about-us/history/}{https://royalsociety.org/about-us/history/}>. Visited on: 9 Sept. 2023.}
		%:::% epigraph body end %:::%
	\end{flushright}
\end{epigrafe}
%:::% epigraph end %:::%

% -----
% Abstract in the vernacular language (mandatory)
% -----

%:::% vernacular-abstract begin %:::%
\begin{resumoenv}[Abstract]
 %:::% vernacular-abstract reference begin %:::%
[Author's surname], [Author's forename(s) initial(s)]. \textbf{\imprimirtitulo}. {\imprimirdata}. {\thelastpage}p. {\imprimirtipotrabalho}  (\imprimirtituloacademico) -- {\imprimirescola}, {\imprimiruniversidade}, {\imprimirlocal}, {\imprimirdata}.
%:::% vernacular-abstract reference end %:::%

%:::% vernacular-abstract body begin %:::%

\{abnt\} is a \href{https://quarto.org}{Quarto} format designed for
theses and dissertations that adhere to the standards of the Brazilian
Association of Technical Standards (ABNT). It is based on the
\href{https://www.abntex.net.br/}{\texttt{abntex2}} LaTeX class and on
\href{https://teses.usp.br/index.php?option=com_content\&view=article\&id=52\&Itemid=67\&lang=en}{USP
guidelines for creating thesis and dissertation documents}.

%:::% vernacular-abstract body end %:::%

%:::% vernacular-abstract keywords begin %:::%
\begin{tabular}{p{2.3cm} p{13.7cm}}
  \textbf{Keywords}: &  [Keyword 1]. [Keyword 2]. [Keyword 3].
\end{tabular}
%:::% vernacular-abstract keywords end %:::%
\end{resumoenv}
%:::% vernacular-abstract end %:::%

% -----
% Abstract in the foreign language (mandatory)
% -----

%:::% foreign-abstract begin %:::%
\begin{resumoenv}[Resumo]
\begin{otherlanguage*}{brazil}
%:::% foreign-abstract reference begin %:::%
[Sobrenome do autor], [Inicial(is) do(s) prenome(s) do autor].  \textbf{[Título]}. {\imprimirdata}. {\thelastpage}p. {\imprimirtipotrabalho}  (\imprimirtituloacademico) -- {\imprimirescola}, {\imprimiruniversidade}, {\imprimirlocal}, {\imprimirdata}.
%:::% foreign-abstract reference end %:::%

%:::% foreign-abstract body begin %:::%

\{abnt\} is a \href{https://quarto.org}{Quarto} format designed for
theses and dissertations that adhere to the standards of the Brazilian
Association of Technical Standards (ABNT). It is based on the
\href{https://www.abntex.net.br/}{\texttt{abntex2}} LaTeX class and on
\href{https://teses.usp.br/index.php?option=com_content\&view=article\&id=52\&Itemid=67\&lang=en}{USP
guidelines for creating thesis and dissertation documents}.

%:::% foreign-abstract body end %:::%

%:::% foreign-abstract keywords begin %:::%
\begin{tabular}{p{3.6cm} p{12.4cm}}
  \textbf{Palavras-chaves}: &  [Palavra-chave 1]. [Palavra-chave 2]. [Palavra-chave 3].
\end{tabular}
%:::% foreign-abstract keywords end %:::%
\end{otherlanguage*}
\end{resumoenv}
%:::% foreign-abstract end %:::%

% -----
% List of figures (optional)
% -----

%:::% list-of-figures begin %:::%
\pdfbookmark[0]{\listfigurename}{lof}
\listoffigures*
\cleardoublepage
%:::% list-of-figures end %:::%

% -----
% List of tables (optional)
% -----

%:::% list-of-tables begin %:::%
\pdfbookmark[0]{\listtablename}{lot}
\listoftables*
\cleardoublepage
%:::% list-of-tables end %:::%

% -----
% List of abbreviations and acronyms (optional)
% -----

%:::% list-of-abbreviations begin %:::%
\begin{siglas}
%:::% list-of-abbreviations body begin %:::%

\begin{description}
\item[\textsubscript{F}]
\hspace{20cm}

Subscript indicating a relation with work-free days
\item[\textsubscript{W}]
\hspace{20cm}

Subscript indicating a relation with workdays
\item[MCTQ]
\hspace{20cm}

Munich ChronoType Questionnaire
\item[MCTQ\textsuperscript{PT}]
\hspace{20cm}

Portuguese version of the MCTQ
\item[MEQ]
\hspace{20cm}

Morningness-Eveningness Questionnaire
\item[MSF]
\hspace{20cm}

Local time of mid-sleep on work-free days
\item[MSF\textsubscript{sc}]
\hspace{20cm}

Chronotype proxy. The midpoint between sleep onset and sleep end on
work-free days. A sleep correction (\textsubscript{SC}) is made when a
possible sleep compensation related to a lack of sleep on workdays is
identified.
\item[MSW]
\hspace{20cm}

Local time of mid-sleep on workdays
\end{description}

%:::% list-of-abbreviations body end %:::%
\end{siglas}
%:::% list-of-abbreviations end %:::%

% -----
% List of symbols (optional)
% -----

%:::% list-of-symbols begin %:::%
\begin{simbolos}
%:::% list-of-symbols body begin %:::%

For an extensive list of chronobiology related symbols, please refer to
\textcite{aschoff1965} and \textcite{marques2012}.

\begin{description}
\item[\(\tau\)]
\hspace{20cm}

Period of a rhythm in free flow; only revealed under constant
environmental conditions.
\item[\(T\)]
\hspace{20cm}

Zeitgeber period.
\item[\(\phi\)]
\hspace{20cm}

Phase
\item[\(\Delta\phi\)]
\hspace{20cm}

Phase shift
\item[\(+\Delta\phi\)]
\hspace{20cm}

Phase advance
\item[\(-\Delta\phi\)]
\hspace{20cm}

Phase delay
\item[\(\Psi\)]
\hspace{20cm}

Phase relation
\end{description}

%:::% list-of-symbols body end %:::%
\end{simbolos}
%:::% list-of-symbols end %:::%

% -----
% List of terms and definitions (optional)
% -----

%:::% list-of-terms begin %:::%
\begin{termos}
%:::% list-of-terms body begin %:::%

For an extensive list of chronobiology related terms and definitions,
please refer to \textcite{aschoff1965} and \textcite{marques2012}.

\begin{description}
\item[Chronotype]
\hspace{20cm}

Any kind of temporal phenotype \autocite{ehret1974,pittendrigh1993}.
Usually, it refers to circadian phenotypes in a spectrum that goes from
morningness to eveningness \autocite{horne1976,roenneberg2003}. It can
also be seen as an organism's phase of entrainment
\autocite{roenneberg2012a}.
\item[Circadian rhythm]
\hspace{20cm}

A rhythm with a period close to a day/24h, an approximation to the
period of the earth's rotation \autocite{pittendrigh1960}. From the
Latin \emph{circā}, around, and \emph{dĭes}, day \autocite{latinitium}.
Example: the sleep-wake cycle.
\item[Complex system]
\hspace{20cm}

There are several definitions. Here are some that I found to be of use:
\end{description}

\begin{itemize}
\tightlist
\item
  ``Systems that don't yield to compact forms of representation or
  description'' (David Krakauer apud \textcite{mitchell2013})
\item
  ``A system of many interacting parts where the system is more than
  just the sum of its parts'' (Mark Newman apud \textcite{mitchell2013})
\item
  Systems with many connected agents that interact and exhibit
  self-organization and emergence behavior, all without the need for a
  central controller (adapted from Camilo Rodrigues Neto's definition,
  supervisor of this thesis).
\item
  Dialectics at its finest (my working definition).
\end{itemize}

\begin{description}
\item[Entrainment]
\hspace{20cm}

A shift and alignment of biological rhythms induced by a zeitgeber input
\autocite{kuhlman2018}. For example: a shift/alignment of an organism's
circadian rhythm when exposed to light.
\end{description}

%:::% list-of-terms body end %:::%
\end{termos}
%:::% list-of-terms end %:::%

% -----
% Table of contents (mandatory)
% -----

%:::% table-of-contents begin %:::%
\pdfbookmark[0]{\contentsname}{toc}
\tableofcontents*
\cleardoublepage
%:::% table-of-contents end %:::%
