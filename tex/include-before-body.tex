%:::% class attribute begin/end %:::%

% -----
% Cover (Mandatory)
% -----

%:::% cover begin %:::%
\imprimircapa
%:::% cover end %:::%

% -----
% Title Page (Mandatory)
% -----

%:::% approval-sheet begin %:::%
\imprimirfolhaderosto
%:::% approval-sheet end %:::%

% -----
% Cataloging Record (Mandatory)
% -----

%:::% cataloging-record begin %:::%
\begin{fichacatalografica}
\hyphenpenalty=100000
%:::% cataloging-record body begin %:::%
I authorize and grant permission for the full or partial reproduction and distribution of this work by any conventional or electronic means for the  purposes of study and research, provided that the source is properly cited, the integrity of the original work is maintained, and such reproduction is not intended for commercial gain.

\vfill

\begin{center}
Cataloging in publication

[University. School. Library]

[CRB-8]

\medskip
\ABNTEXfontereduzida

\setlength{\fboxsep}{1cm}
\fbox{\begin{minipage}[c][6.5cm]{12.5cm}
[Author's surname], [Author's forename(s)]

\smallskip

\hspace{0.5cm} {\imprimirtitulo}  / {\imprimirautor}; supervisor, {\imprimirorientador}; co-supervisor {\imprimircoorientador}. {\imprimirlocal}, {\imprimirdata}

\smallskip

{\thelastpage}p. : il

\smallskip

\hspace{0.5cm} {\imprimirtipotrabalho} (\imprimirtituloacademico) -- {\imprimirprograma}, {\imprimirescola}, {\imprimiruniversidade}, {\imprimirdata}.

\smallskip

\hspace{0.5cm} {\imprimirnotadeversao}

\smallskip

\hspace{0.5cm} 1. [Subject A]. 2. [Subject B]. 3. [Subject C]. I. [Supervisor's surname], [Supervisor's forename(s)], super. II. [Co-supervisor's surname], [Co-supervisor's forename(s)] co-super. III. Title.
\end{minipage}}
\end{center}
\vspace{\hugeskipamount}
%:::% cataloging-record body end %:::%
\end{fichacatalografica}
%:::% cataloging-record end %:::%

% -----
% Errata (Optional)
% -----

%:::% errata begin %:::%
\begin{errata}[\errataname]
%:::% errata reference begin %:::%
\noindent [AUTHOR’S SURNAME], [Author’s forename(s) initial(s)]. \textbf{\imprimirtitulo}. {\imprimirdata}. {\thelastpage}p. {\imprimirtipotrabalho}  ({\imprimirtituloacademico}) -- {\imprimirescola}, {\imprimiruniversidade}, {\imprimirlocal}, {\imprimirdata}.
%:::% errata reference end %:::%
\smallskip
%:::% errata body begin %:::%

This is the preliminary version of this thesis. Any required corrections
will be listed here upon approval.

%:::% errata body end %:::%
\end{errata}
%:::% errata end %:::%

% -----
% Approval Sheet (Mandatory)
% -----

%:::% approval-sheet begin %:::%
\begin{folhadeaprovacao}[\folhadeaprovacaoname]
\hyphenpenalty=100000
%:::% approval-sheet body begin %:::%
{\imprimirtipotrabalho} by {\imprimirautor}, under the title \textbf{\imprimirtitulo}, presented to the {\imprimirescola} at the {\imprimiruniversidade}, as a requirement for the degree of {\imprimirtituloacademico} by the {\imprimirprograma}, in the concentration area of {\imprimirareadeconcentracao}.

\vspace{\hugeskipamount}
Approved on [Month] [Day], [Year].

\vspace{\hugeskipamount}
\begin{center}
  Examination Committee
\end{center}

\vspace{\smallskipamount}
Committee Chair:

\vspace{\tinyskipamount}
\begingroup

\AtBeginEnvironment{tabular}{
  \normalsize\raggedright
  \renewcommand{\arraystretch}{2}
}

\setlength{\arrayrulewidth}{0pt}
\setlength{\tabcolsep}{0cm}
\begin{tabular}{m{2.5cm} m{13.5cm}}
  Prof. Dr. & [Full name] \\
  Institution & [School], [University] \\
\end{tabular}

\vspace{\bigskipamount}
Examiners:

\vspace{\tinyskipamount}
\begin{tabular}{m{2.5cm} m{13.5cm}}
  Prof. Dr. & [Full name] \\
  Institution & [School], [University] \\
  Evaluation & [Approved/Rejected] \\
\end{tabular}

\vspace{\smallskipamount}
\begin{tabular}{m{2.5cm} m{13.5cm}}
  Prof. Dr. & [Full name] \\
  Institution & [School], [University] \\
  Evaluation & [Approved/Rejected] \\
\end{tabular}

\vspace{\smallskipamount}
\begin{tabular}{m{2.5cm} m{13.5cm}}
  Prof. Dr. & [Full name] \\
  Institution & [School], [University] \\
  Evaluation & [Approved/Rejected] \\
\end{tabular}
\endgroup
%:::% approval-sheet body end %:::%
\end{folhadeaprovacao}
%:::% approval-sheet end %:::%

% -----
% Inscription (Optional)
% -----

%:::% inscription begin %:::%
\begin{dedicatoria}[] % \dedicatorianame | Keep #1 empty.
\vspace*{\fill} % Don't change it.
\centering
%:::% inscription body begin %:::%
\textit{To the worm that first gnawed on the cold flesh of my corpse,}

\textit{I dedicate, as a fond remembrance, these posthumous memories.}\footnotemark{}

\footnotetext{
	ASSIS, M. \textbf{Memórias póstumas de Brás Cubas} [The Posthumous Memoirs of Brás Cubas]. São Paulo: Companhia das Letras, 2014. ISBN 978-85-438-0163-6.
}
%:::% inscription body end %:::%
\vspace*{\fill} % Don't change it.
\vspace{4.5cm}
% \vspace{13cm}
\end{dedicatoria}
%:::% inscription end %:::%

% -----
% Acknowledgments (Optional)
% -----

%:::% acknowledgments begin %:::%
\begin{agradecimentos}[\agradecimentosname]
  %:::% acknowledgments body begin %:::%

I would like to acknowledge this awesome
\href{https://github.com/danielvartan/abnt}{Quarto format}! :)

  %:::% acknowledgments body end %:::%
\end{agradecimentos}
%:::% acknowledgments end %:::%

% -----
% Epigraph (Optional)
% -----

%:::% epigraph begin %:::%
\begin{epigrafe}[] % \epigraphname | Keep #1 empty.
\vspace*{\fill} % Don't change it.
\begin{flushright}
%:::% epigraph body begin %:::%
\textit{Nullius in verba}\footnotemark{}

\footnotetext{
	THE ROYAL SOCIETY. \textbf{History of the Royal Society}. Available from: <\href{https://royalsociety.org/about-us/history/}{https://royalsociety.org/about-us/history/}>. Visited on: 9 Sept. 2023.
}
%:::% epigraph body end %:::%
\end{flushright}
\end{epigrafe}
%:::% epigraph end %:::%

% -----
% Abstract in Vernacular Language (Mandatory)
% -----

%:::% vernacular-abstract begin %:::%
\begin{resumoenv}[\resumoname]
 %:::% vernacular-abstract reference begin %:::%
[AUTHOR’S SURNAME], [Author’s forename(s) initial(s)]. \textbf{\imprimirtitulo}. {\imprimirdata}. {\thelastpage}p. {\imprimirtipotrabalho}  (\imprimirtituloacademico) -- {\imprimirescola}, {\imprimiruniversidade}, {\imprimirlocal}, {\imprimirdata}.
%:::% vernacular-abstract reference end %:::%

%:::% vernacular-abstract body begin %:::%

\href{https://github.com/danielvartan/abnt}{abnt} is a
\href{https://quarto.org}{Quarto} format designed for creating theses
and dissertations that comply with guidelines established by the
Brazilian Association of Technical Standards
(\href{https://www.abnt.org.br/}{ABNT}). Learn more at
\url{https://github.com/danielvartan/abnt}.

%:::% vernacular-abstract body end %:::%

%:::% vernacular-abstract keywords begin %:::%
\begin{tabular}{p{2.3cm} p{13.6cm}}
  \textbf{Keywords}: & [Keyword 1]. [Keyword 2]. [Keyword 3].
\end{tabular}
%:::% vernacular-abstract keywords end %:::%
\end{resumoenv}
%:::% vernacular-abstract end %:::%

% -----
% Abstract in Foreign Language (Mandatory)
% -----

%:::% foreign-abstract begin %:::%
\begin{resumoenv}[\resumoestrangeironame]
\begin{otherlanguage*}{brazil}
%:::% foreign-abstract reference begin %:::%
[SOBRENOME DO AUTOR], [Inicial(is) do(s) prenome(s) do autor].  \textbf{[Título]}. {\imprimirdata}. {\thelastpage}p. [Dissertação/Tese]  ([Título acadêmico]) -- [Escola/Faculdade], [Universidade], [Cidade/Local], {\imprimirdata}.
%:::% foreign-abstract reference end %:::%

%:::% foreign-abstract body begin %:::%

\href{https://github.com/danielvartan/abnt}{abnt} é um formato
\href{https://quarto.org}{Quarto} projetado para criar teses e
dissertações que atendem às diretrizes estabelecidas pela Associação
Brasileira de Normas Técnicas (\href{https://www.abnt.org.br/}{ABNT}).
Saiba mais em \url{https://github.com/danielvartan/abnt}.

%:::% foreign-abstract body end %:::%

%:::% foreign-abstract keywords begin %:::%
\begin{tabular}{p{3.6cm} p{12.3cm}}
  \textbf{Palavras-chaves}: & [Palavra-chave 1]. [Palavra-chave 2]. [Palavra-chave 3].
\end{tabular}
%:::% foreign-abstract keywords end %:::%
\end{otherlanguage*}
\end{resumoenv}
%:::% foreign-abstract end %:::%

% -----
% List of Figures (Optional)
% -----

%:::% list-of-figures begin %:::%
\pdfbookmark[0]{\listfigurename}{lof}
\listoffigures*
\cleardoublepage
%:::% list-of-figures end %:::%

% -----
% List of Tables (Optional)
% -----

%:::% list-of-tables begin %:::%
\pdfbookmark[0]{\listtablename}{lot}
\listoftables*
\cleardoublepage
%:::% list-of-tables end %:::%

% -----
% List of Abbreviations and Acronyms (Optional)
% -----

%:::% list-of-abbreviations begin %:::%
\begin{siglas}
%:::% list-of-abbreviations body begin %:::%

\begin{description}
\item[\textsubscript{F}]
\hspace{20cm}

Subscript indicating a relation with work-free days.
\item[\textsubscript{W}]
\hspace{20cm}

Subscript indicating a relation with workdays.
\item[MCTQ]
\hspace{20cm}

Munich ChronoType Questionnaire.
\item[MCTQ\textsuperscript{PT}]
\hspace{20cm}

Portuguese version of the MCTQ.
\item[MEQ]
\hspace{20cm}

Morningness-Eveningness Questionnaire.
\item[MSF]
\hspace{20cm}

Local time of mid-sleep on work-free days.
\item[MSF\textsubscript{sc}]
\hspace{20cm}

Chronotype proxy. The midpoint between sleep onset and sleep end on
work-free days. A sleep correction (\textsubscript{SC}) is made when a
possible sleep compensation related to a lack of sleep on workdays is
identified.
\item[MSW]
\hspace{20cm}

Local time of mid-sleep on workdays.
\end{description}

%:::% list-of-abbreviations body end %:::%
\end{siglas}
%:::% list-of-abbreviations end %:::%

% -----
% List of Symbols (Optional)
% -----

%:::% list-of-symbols begin %:::%
\begin{simbolos}
%:::% list-of-symbols body begin %:::%

For an extensive list of chronobiology related symbols, please refer to
\textcite{aschoff1965} and \textcite{marques2012}.

\begin{description}
\item[\(\tau\)]
\hspace{20cm}

Period of a rhythm in free flow; only revealed under constant
environmental conditions.
\item[\(T\)]
\hspace{20cm}

Zeitgeber period.
\item[\(\phi\)]
\hspace{20cm}

Phase.
\item[\(\Delta\phi\)]
\hspace{20cm}

Phase shift.
\item[\(+\Delta\phi\)]
\hspace{20cm}

Phase advance.
\item[\(-\Delta\phi\)]
\hspace{20cm}

Phase delay.
\item[\(\Psi\)]
\hspace{20cm}

Phase relation.
\end{description}

%:::% list-of-symbols body end %:::%
\end{simbolos}
%:::% list-of-symbols end %:::%

% -----
% Table of Contents (Mandatory)
% -----

%:::% table-of-contents begin %:::%
\pdfbookmark[0]{\contentsname}{toc}
\tableofcontents*
\cleardoublepage
%:::% table-of-contents end %:::%

% -----
% Other Additions
% -----

